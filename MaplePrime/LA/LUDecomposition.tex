線形代数の計算にはあらかじめ関数パッケージ(LinearAlgebra)を呼び出しておく.
\begin{MapleInput}
> with(LinearAlgebra):
\end{MapleInput}

\subsection{行列の基本操作}
行列の掃き出しに必要となる行列の基本操作はRowOperation, ColumnOperationを参照.

\subsection{掃き出し法,LU分解(LUDecomposition)}
掃き出し法の計算は,LUDecompositionでおこなう.まず拡大係数行列を作る.
\begin{MapleInput}
> A1:=<1,2;3,4>; b:=<2,3>; <A1|b>;
\end{MapleInput}
\begin{MapleOutputGather}
{\it A1}\, := \, \left[ \begin {array}{cc} 1&2\\ 3&4\end {array} \right] \notag \\
b\, := \, \left[ \begin {array}{c} 2\\ 3\end {array} \right] \notag \\
\left[ \begin {array}{ccc} 1&2&2\\ 3&4&3\end {array} \right]  \notag
\end{MapleOutputGather}
これにLU分解をかける.それぞれP(permutation,置換), L(lower triangle,下三角), U(upper
triangle,上三角)行列に代入している.
\begin{MapleInput}
> P,L,U:=LUDecomposition(<A1|b>);
\end{MapleInput}
\begin{MapleOutput}
P,\,L,\,U\, := \, \left[ \begin {array}{cc} 1&0\\ 0&1\end {array} \right] ,\, \left[ \begin {array}{cc} 1&0\\ 3&1\end {array} \right] ,\, \left[ \begin {array}{ccc} 1&2&2\\ 0&-2&-3\end {array} \right] 
\end{MapleOutput}
さらに被約階段行列(row reduced echelonmatrix;後退代入までおこなって,解まで求めた状態)を求めるには,output='R'を指定する.
\begin{MapleInput}
> LUDecomposition(<A1|b>, output='R');
\end{MapleInput}
\begin{MapleOutput}
\left[ \begin {array}{ccc} 1&0&-1\\ 0&1&3/2\end {array} \right]
\end{MapleOutput}

\subsection{階数(Rank)}
行列の性質の中でも特に重要な階数(Rank)は次のコマンドで求められる.
\begin{MapleInput}
> Rank(A1);
\end{MapleInput}
\begin{MapleOutput}
2
\end{MapleOutput}