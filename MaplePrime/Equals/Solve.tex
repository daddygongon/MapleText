\subsection{方程式の解(solve)}
\begin{MapleError}
3x=2を満たすxをもとめよ.
\end{MapleError}
という問題は,
\begin{MapleInput}
> solve(3*x=2,x);
\end{MapleInput}
\begin{MapleOutput}
\frac{2}{3}
\end{MapleOutput}
連立方程式は以下のとおり.
\begin{MapleInput}
> solve({x+y=1,x-y=2},{x,y});
\end{MapleInput}
\begin{MapleOutput}
\left\{x = \frac{3}{2}, y = -\frac{1}{2} \right\}
\end{MapleOutput}
ただし,solveだけでは,x,yに値は代入されない.
\begin{MapleInput}
> sol1:=solve({x+y=1,x-y=2},{x,y});
> assign(sol1);
\end{MapleInput}
\begin{MapleOutput}
sol1 := \left\{x = \frac{3}{2}, y = -\frac{1}{2} \right\}
\end{MapleOutput}
とする必要がある.確認してみると
\begin{MapleInput}
> x,y;
\end{MapleInput}
\begin{MapleOutput}
\frac{3}{2},-\frac{1}{2}
\end{MapleOutput}
となり,値が代入されていることがわかる.

\paragraph{方程式の数値解(fsolve)}
解析的に解けない場合は,数値的に解を求めるfsolveを使う.上でxにassignしているので,xを初期化している.
\begin{MapleInput}
> x:='x';
> fsolve(log(x)-exp(-x),x);
\end{MapleInput}
\begin{MapleOutputGather}
x := x \notag \\
1.309799586 \notag
\end{MapleOutputGather}

