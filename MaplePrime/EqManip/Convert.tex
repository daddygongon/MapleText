\subsection{形式の変換(convert(exp1,opt))}
数式の記述形式を変えるのに頻繁に使う.
\begin{table}[h]
\begin{tabular}{l|l}
\hline
opt&意味 \\ \hline 
polynom&級数を多項式(polynomial)に\\
trig&三角関数(trigonal)に\\
sincos&tanを含まない,sin,cosに\\
exp&指数関数形式に\\
parfrac&部分分数(partial fraction)に\\
rational&浮動小数点数を有理数形式に \\ \hline
\end{tabular}
\end{table}%

\begin{MapleInput}
> s1:=series(sin(x),x,4);
> convert(s1,polynom);
\end{MapleInput}
\begin{MapleOutputGather}
{\it s1}\, := \,x-\frac{1}{6}\,{x}^{3}+O \left( {x}^{4} \right) \notag \\
x-\frac{1}{6}\,{x}^{3} \notag
\end{MapleOutputGather}

\begin{MapleInput}
> convert(sin(x),exp);
\end{MapleInput}
\begin{MapleOutput}
-\frac{1}{2}\,I\left( \exp(Ix)-\exp(-Ix) \right) 
\end{MapleOutput}

\begin{MapleInput}
> convert(sinh(x),exp);
\end{MapleInput}
\begin{MapleOutput}
\frac{1}{2}\,\exp(x)-\frac{1}{2}\,\exp(x)
\end{MapleOutput}

\begin{MapleInput}
> convert(tan(x),sincos);
\end{MapleInput}
\begin{MapleOutput}
{\frac {\sin \left( x \right) }{\cos \left( x \right) }}
\end{MapleOutput}

\begin{MapleInput}
> convert(exp(I*x),trig);
\end{MapleInput}
\begin{MapleOutput}
\cos \left( x \right) +I\sin \left( x \right)
\end{MapleOutput}

\begin{MapleInput}
> convert(1/(x-1)/(x+3),parfrac);
\end{MapleInput}
\begin{MapleOutput}
- \frac{1}{4\left( x+3 \right) }+ \frac{1}{4 \left( x-1 \right) }
\end{MapleOutput}

\begin{MapleInput}
> convert(3.14,rational);
\end{MapleInput}
\begin{MapleOutput}
{\frac {157}{50}}
\end{MapleOutput}

\subsection{左辺,右辺(lhs, rhs)}
それぞれ,左辺,右辺を意味するleft hand side, right hand sideの略
\begin{MapleInput}
> lhs(sin(x)^2=1-1/x);
> rhs(sin(x)^2=1-1/x);
\end{MapleInput}
\begin{MapleOutputGather}
\sin \left( x \right)^{2} \notag \\
1-\frac{1}{x} \notag
\end{MapleOutputGather}

\subsection{分母,分子(denom,numer)} 
それぞれ,分母,分子を意味するdenominator, numeratorの略
\begin{MapleInput}
> numer(a*x/(x+y)^3);
> denom(a*x/(x+y)^3);
\end{MapleInput}
\begin{MapleOutputGather}
ax \notag \\
\left( x+y \right) ^{3} \notag
\end{MapleOutputGather}

\subsection{係数(coeff)} 
\begin{MapleInput}
> coeff(4*a*x^2-3*y^2/x+6*b*x*y+3*c*y+2*y^2,y^2);
\end{MapleInput}
\begin{MapleOutput}
-\frac{3}{x}+2
\end{MapleOutput}

\subsection{要素の取りだし,要素数(op, nops)}
op, nopsはlist配列から要素や要素数を取り出すのに頻繁に使われる.数式を含めた,より一般的な構造に対しても作用させることができる.
\begin{MapleInput}
> op(4*a*x^2-3*y^2/x+6*b*x*y+3*c*y+2*y^2);
\end{MapleInput}
\begin{MapleOutput}
4\,a{x}^{2},\,-{\frac {{3\,y}^{2}}{x}},\,6\,bxy,\,3\,cy,\,2\,{y}^{2}
\end{MapleOutput}

\begin{MapleInput}
> nops(4*a*x^2-3*y^2/x+6*b*x*y+3*c*y+2*y^2); #res: 5
\end{MapleInput}
