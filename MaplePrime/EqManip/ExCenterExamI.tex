\begin{enumerate}
\item
$P = x(x+3)(2x-3)$とする.
また,$a$を定数とする.
$x = a+1$のときの
$P$の値は
\begin{equation*}
2a^3+\fbox{  ア  }a^2+\fbox{  イ  }a-\fbox{  ウ  }
\end{equation*}
である.

$x=a+1$のときの$P$の値と,$x=a$のときの$P$の値が等しいとする.このとき,$a$は
\begin{equation*}
3a^2+\fbox{ エ }a-\fbox{ オ } = 0
\end{equation*}
を満たす.したがって
\begin{equation*}
a = \frac{\fbox{ カキ }\pm \sqrt{\fbox{ クケ }}}{\fbox{ コ }}
\end{equation*}
である.

\item
(例題1.に引き続いて,)
さらに,2次関数(1)のグラフの頂点のy座標が-2であるとする.このとき,aは
\begin{equation*}
\fbox{ キ }\,a^2-\fbox{ クケ }\,a+\fbox{ コ } = 0
\end{equation*}
を満たす.これより,aの値は
\begin{equation*}
a = \fbox{ サ }, \frac{\fbox{ シ }}{\fbox{ ス }}
\end{equation*}
である.
以下,$a = \frac{\fbox{ シ }}{\fbox{ ス }}$であるとする.

このとき,2次関数(1)のグラフの頂点のx座標は\fbox{ セ }であり,(1)のグラフとx軸の2交点のx座標は\fbox{ ソ },\fbox{ タ }である.

また,関数(1)は$0 \leqq x \leqq 9$において

$x$ = \fbox{ チ }のとき,最小値\fbox{ ツテ }をとり, 

$x$ = \fbox{ ト }のとき,最大値$\frac{\fbox{ ナニ }}{\fbox{ ヌ }}$をとる.


(2008 年度大学入試センター試験数学 I より抜粋).

\end{enumerate}