\subsection{studentパッケージいろいろ}
ちょっとぐらい難しい積分も,Mapleは単純にintコマンドだけで実行してくれる.しかし,時には,途中の計算法である部分積分,置換積分,部分分数展開が必要になる.このような計算はstudentパッケージに用意さている.
\begin{MapleInput}
> with(student):
\end{MapleInput}
\paragraph{部分積分(integration by parts)} 
\begin{MapleInput}
> intparts(Int(x*exp(x),x),x);
\end{MapleInput}
\begin{MapleOutput}
x\exp(x)-\int \exp(x){dx}
\end{MapleOutput}
\paragraph{置換(change of variables)による積分} 
\begin{MapleInput}
> Int((cos(x)+1)^3*sin(x), x);
> changevar(cos(x)+1=u, Int((cos(x)+1)^3*sin(x), x=a..b), u);
> changevar(cos(x)+1=u, int((cos(x)+1)^3*sin(x), x), u);
\end{MapleInput}
\begin{MapleOutputGather}
\int \left( \cos \left( x \right) +1 \right) ^{3}\sin \left( x \right) {dx} \notag \\
\int _{\cos \left( a \right) +1}^{\cos \left( b \right) +1}-{u}^{3}{du} \notag \\
 -\frac{1}{4}\,{u}^{4} \notag \\
\end{MapleOutputGather}
\paragraph{部分分数(partial fraction)展開による積分}
部分分数(partial fraction)展開による積分では,convertコマンドを用いる.
\begin{MapleInput}
> pf1:=convert(1/(1+x^3),parfrac,x);
  int(pf1,x);
\end{MapleInput}
\begin{MapleOutputGather}
 {\it pf1}\, := \frac{1}{3}\,{\frac {-x+2}{{x}^{2}-x+1}}+ \frac{1}{3\,\left( x+1 \right) } \notag \\
 -\frac{1}{6}\,\ln  \left( {x}^{2}-x+1 \right) +\frac{1}{3}\, \sqrt{3}\arctan \left( 1/3\, \left( 2\,x-1 \right)  \sqrt{3} \right) +\frac{1}{3}\,\ln  \left( x+1 \right) \notag
\end{MapleOutputGather}

