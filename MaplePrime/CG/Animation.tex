\subsection{animate関数}
plotsパッケージにあるanimate関数を使う.構文は以下の通りで,[]内に動画にしたい関数を定義し,tで時間を変えていく.
\begin{MapleInput}
> with(plots): animate(plot, [sin(x-t),x=0..5*Pi], t=0..10);
\end{MapleInput}
\MaplePlot{50mm}{./figures/MapleCGplot2d9.eps}

\subsection{リストに貯めて,display表示}
おなじ動作を,display関数でオプションとしてinsequence=trueとしても可能.この場合は第一引数に入れるリスト([])に一連の画像を用意し,コマ
送りで表示させる.
\begin{MapleInput}
> tmp:=[]: n:=10: for i from 0 to n do t:=i; tmp:=[op(tmp),
> plot(sin(x-t)+sin(x+t),x=0..5*Pi)]; end do:
> display(tmp,insequence=true);
\end{MapleInput}
\MaplePlot{50mm}{./figures/MapleCGplot2d10.eps}

\subsection{凝った例}
ヘルプにある凝った例.Fという動画のコマを吐く関数を用意する.これを,animate関数から適当な変数を入れて呼び出す.backgroundには動かない絵を指定
することができる.
\begin{MapleInput}
> with(plottools,line): F := proc(t) plots[display](
> line([-2,0],[cos(t)-2,sin(t)],color=blue),
> line([cos(t)-2,sin(t)],[t,sin(t)],color=blue),
> plot(sin(x),x=0..t,view=[-3..7,-5..5]) ); end:
> animate(F,[theta],theta=0..2*Pi, background=plot([cos(t)-2,sin(t),t=0..2*Pi]),
> scaling=constrained,axes=none);
\end{MapleInput}
\begin{MapleOutput}
\end{MapleOutput}
\MaplePlot{50mm}{./figures/MapleCGplot2d11.eps}

\subsection{ファイルへの保存}
animationなどのgif形式のplotを外部ファイルへ出力して表示させるには,以下の一連のコマンドのようにする.
\begin{MapleInput}
> plotsetup(gif,plotoutput=file2): display(tmp,insequence=true);
> plotsetup(default):
\end{MapleInput}
こいつをquicktimeなどに食わせれば,Maple以外のソフトで動画表示が可能となる.3次元図形の標準規格であるvrmlも同じようにして作成することが可能(?vrml;参照).
