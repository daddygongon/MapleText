\subsubsection{Functions}
\begin{enumerate}
\item evalfのヘルプを参照して,Piを1000桁まで表示せよ.
\begin{MapleInput}
> ?evalf;
> evalf[1000](Pi);      #省略
\end{MapleInput}

\item 正接関数(tan)とその逆関数arctanをx=-Pi/2..Pi/2,y=-Pi..Pi,scaling=constrainedで同時にプロットせよ.
\begin{MapleInput}
> plot([tan(x),arctan(x)],x=-Pi/2..Pi/2,y=-Pi..Pi,scaling=constrained);
\end{MapleInput}
\MaplePlot{50mm}{./figures/Functionsplot2d2.eps}

\item 対数関数はln(x)で与えられる.ヘルプを参照しながら次の値を求めよ.
\begin{equation*}
\log_{10}1000, \log_{2}\frac{1}{16}, \log_{\sqrt{5}}125
\end{equation*}

\begin{MapleInput}
> ?ln;
\end{MapleInput}

\begin{MapleInput}
> log10(1000);
\end{MapleInput}
\begin{MapleOutput}
3
\end{MapleOutput}
\begin{MapleInput}
> log[2](1/16);
\end{MapleInput}
\begin{MapleOutput}
-4
\end{MapleOutput}
\begin{MapleInput}
> log[sqrt(5)](125);
\end{MapleInput}
\begin{MapleOutput}
6
\end{MapleOutput}

\item 次の関数は$y=2^x$とどのような位置関係にあるかx=-5..5,y=-5..5で同時にプロットして観察せよ.
\begin{equation*}
y = -2^x, y = (1/2)^x,y = -(1/2)^x
\end{equation*}
\begin{MapleInput}
> plot([2^x,-2^x,(1/2)^x,-(1/2)^x],x=-5..5,y=-5..5);
\end{MapleInput}
注目している関数以外を消せばはっきりするが,i) x軸に対称,ii) y軸に対称, iii) 原点に対称.
\MaplePlot{50mm}{./figures/Functionsplot2d3.eps}

\item 指数関数はexpで与えられる.$\mbox{e}^x$と$\log x$関数および$y = x$を同時にx=-5..5,y=-5..5でplotせよ.またそれらの関数の位置関係を述べよ.
\begin{MapleInput}
> plot([exp(x),log(x),x],x=-5..5,y=-5..5);
\end{MapleInput}
\MaplePlot{30mm}{./figures/Functionsplot2d4.eps}

\item 階乗関数factorialに3を代入して何を求める関数か予測せよ.ヘルプを参照し,よりなじみの深い表記を試してみよ.
\begin{MapleInput}
> factorial(3)
\end{MapleInput}
\begin{MapleOutput}
6
\end{MapleOutput}
\begin{MapleInput}
> ?factorial;
> 3!;
\end{MapleInput}

\end{enumerate}
