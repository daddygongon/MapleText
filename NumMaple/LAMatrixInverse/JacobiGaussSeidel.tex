以下のような連立方程式を
\begin{equation*}
\left[ \begin {array}{c} 5\,x+y+z+u\\ x+3\,y+z+u\\ x-2\,y-9\,z+u\\ x+3\,y-2\,z+5\,u\end {array} \right] = \left[ \begin {array}{c} -6\\ 2\\ -7\\ 3\end {array} \right]
\end{equation*}
形式的に解くと
\begin{equation*}
x=\frac{-6-(y+z+u)}{5}
\end{equation*}
となる.他の未知数も,
\ifHIKI
||y=||         ||
||z=||         ||
||u=||         ||
\else
\begin{equation*}
\setlength{\unitlength}{1cm}
\begin{picture}(5,4.5)
\put(0,3.0){y=\framebox(4,1.3){}}
\put(0,1.5){z=\framebox(4,1.3){}}
\put(0,0.0){u=\framebox(4,1.3){}}
\end{picture}
\end{equation*}
\fi
となる.適当に初期値($x_0,y_0,z_0,u_0$)をとり,下側の方程式に代入すると,得られた出力($x_1,y_1,z_1,u_1$)はより正解に近い値となる.これを繰り返すことによって正解が得られる.これをヤコビ(Jacobi)法と呼び,係数行列の対角要素が非対角要素にくらべて大きいときに適用できる.多くの現実の問題ではこの状況が成り立っている.

Gauss-Seidel法はJacobi法の高速版である.$n$番目の解の組が得られた後に一度に次の解の組に入れ替えるのではなく,得られた解を順次改良した解として使っていく.これにより,収束が早まる.以下にはヤコビ法のコードを示した.x1[i]の配列を変数に換えるだけで,Gauss-Seidel法となる.
\begin{MapleInput}
> AA:=Matrix([[5,1,1,1],[1,3,1,1],[1,-2,-9,1],[1,3,-2,5]]):
  b:=Vector([-6,2,-7,3]): n:=4; 
  x0:=[0,0,0,0]: x1:=[0,0,0,0]: 
  for iter from 1 to 20 do
    for i from 1 to n do
      x1[i]:=b[i]; 
      for j from 1 to n do
        x1[i]:=x1[i]-AA[i,j]*x0[j];
      end do:
      x1[i]:=x1[i]+AA[i,i]*x0[i];
      x1[i]:=x1[i]/AA[i,i];
    end do:
    x0:=evalf(x1);
    print(iter,x0);
  end do:
\end{MapleInput}
\begin{MapleError}
                                      4
          1, [-1.200000000, 0.6666666667, 0.7777777778, 0.6000000000]
          2, [-1.608888889, 0.6074074073, 0.5629629630, 0.7511111112]
          3, [-1.584296296, 0.7649382717, 0.5474897119, 0.7825185186]
          4, [-1.618989300, 0.7514293553, 0.5187050756, 0.6768921810]
          5, [-1.589405322, 0.8077973477, 0.5061160189, 0.6804222770]
          6, [-1.598867129, 0.8009556753, 0.4972691400, 0.6356490634]
          7, [-1.586774776, 0.8219829753, 0.4927633981, 0.6381076766]
          8, [-1.590570810, 0.8186345670, 0.4897074389, 0.6212705292]
          9, [-1.585922507, 0.8265309473, 0.4881589539, 0.6228163974]
         10, [-1.587501260, 0.8249823853, 0.4870924439, 0.6165295146]
         11, [-1.585720869, 0.8279597673, 0.4865626093, 0.6173477984]
         12, [-1.586374035, 0.8272701537, 0.4861897104, 0.6149933572]
         13, [-1.585690644, 0.8283969890, 0.4860087794, 0.6153885990]
         14, [-1.585958873, 0.8280977553, 0.4858782197, 0.6145034472]
         15, [-1.585695884, 0.8285257353, 0.4858165626, 0.6146844092]
         16, [-1.585805341, 0.8283983040, 0.4857707838, 0.6143503606]
         17, [-1.585703890, 0.8285613990, 0.4857498236, 0.6144303994]
         18, [-1.585748324, 0.8285078890, 0.4857337457, 0.6143038680]
         19, [-1.585709101, 0.8285702367, 0.4857266407, 0.6143384296]
         20, [-1.585727061, 0.8285480103, 0.4857209840, 0.6142903344]
\end{MapleError}