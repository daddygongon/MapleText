係数行列(coefficient matrix)と定数項(b)との関係は以下の通りである.
\begin{MapleInput}
> restart; 
  A:=Matrix([[1,1,-2],[1,-2,1],[2,-2,-1]]): 
  X:=Vector([x,y,z]):
  #X:=Vector([1,-1,2]):
  b:=Vector([-4,5,2]): 
  A.X=b;
\end{MapleInput}
\begin{MapleOutput}
\left[ \begin {array}{c} x+y-2\,z\\ x-2\,y+z\\ 2\,x-2\,y-z\end {array} \right] = \left[ \begin {array}{c} -4\\ 5\\ 2\end {array} \right]
\end{MapleOutput}
単に逆行列を求める際は
\begin{MapleInput}
> with(LinearAlgebra): 
  MatrixInverse(A);
\end{MapleInput}
\begin{MapleOutput}
\left[ \begin {array}{ccc} 4/3&5/3&-1\\ 1&1&-1\\ 2/3&4/3&-1\end {array} \right] 
\end{MapleOutput}
である.Mapleでは行列を三角行列に分解するために,LUDecompositionコマンドが用意されている.\begin{MapleInput}
> P,L,U:=LUDecomposition(<A|b>);
\end{MapleInput}
\begin{MapleOutput}
P,\,L,\,U\, := \, \left[ \begin {array}{ccc} 1&0&0\\ 0&1&0\\ 0&0&1\end {array} \right] ,\, \left[ \begin {array}{ccc} 1&0&0\\ 1&1&0\\ 2&4/3&1\end {array} \right] ,\, \left[ \begin {array}{cccc} 1&1&-2&-4\\ 0&-3&3&9\\ 0&0&-1&-2\end {array} \right]
\end{MapleOutput}
係数と定数項から作られる行列を拡大係数行列(augmented matrix) といい,Mapleでは,$<A|b>$で作られる.後退代入までおこなって連立方程式の解を求めるには,以下の通りoptionにoutput='R'をつける.
\begin{MapleInput}
> LUDecomposition(<A|b>,output='R');
\end{MapleInput}
\begin{MapleOutput}
\left[ \begin {array}{cccc} 1&0&0&1\\ 0&1&0&-1\\ 0&0&1&2\end {array} \right] 
\end{MapleOutput}

