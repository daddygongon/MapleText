\begin{enumerate}
\item 補間と近似の違いについて,適切な図を描いて説明せよ.

\item 次の4点
\begin{MapleInput}
x y 
0 1 
1 2
2 3
3 -2
\end{MapleInput}
を通る多項式を以下のそれぞれの手法で求めよ.(a) 逆行列, (b)ラグランジュ補間, (c)ニュートンの差分商公式 
\item
$\tan(5^\circ)=0.08748866355$, 
$\tan(10^\circ)=.1763269807$,
$\tan(15^\circ)=.2679491924$の値を用いて,ラグランジュ補間法により,$\tan(17^\circ)$の値を推定せよ.(2008年度期末試験)

\item
(対数関数のニュートンの差分商補間:2014期末試験,25点) 

2を底とする対数関数(Mapleでは$\log[2](x)$)の$F(9.2)=2.219203$をニュートンの差分商補間を用いて求める.ニュートンの内挿公式は,
\begin{equation*}
\begin{array}{rc}
F (x )&=F (x _{0})+
(x -x _{0})f _{1}\lfloor x_0,x_1\rfloor+
(x -x _{0})(x -x _{1})
f _{2}\lfloor x_0,x_1,x_2\rfloor + \\
& \cdots +  \prod_{i=0}^{n-1} (x-x_i) \, 
f_n \lfloor x_0,x_1,\cdots,x_n \rfloor
\end{array}
\end{equation*}
である.ここで$f_i \lfloor\, \rfloor$ は次のような関数を意味していて,
\begin{eqnarray*}
%\begin{array}{rl}
f _{1}\lfloor x_0,x_1\rfloor &=&  \frac{y_1-y_0}{x_1-x_0} \\
f _{2}\lfloor x_0,x_1,x_2\rfloor &=&  \frac{f _{1}\lfloor x_1,x_2\rfloor-
f _{1}\lfloor x_0,x_1\rfloor}{x_2-x_0} \\
\vdots & \\
f _{n}\lfloor x_0,x_1,\cdots,x_n\rfloor &=&  \frac{f_{n-1}\lfloor x_1,x_2\cdots,x_{n}\rfloor-
f _{n-1}\lfloor x_0,x_1,\cdots,x_{n-1}\rfloor}{x_n-x_0} \\
%\end{array}
\end{eqnarray*}
差分商と呼ばれる.$x_k=8.0,9.0,10.0,11.0$をそれぞれ選ぶと,差分商補間のそれぞれの項は以下の通りとなる.
\begin{equation*}
	\begin{array}{ccl|lll}
	\hline
	k  &  x_k & y_k=F_0( x_k) &f_1\lfloor x_k,x_{k+1}\rfloor & f_2\lfloor x_k,x_{k+1},x_{k+2}\rfloor &  f_3\lfloor x_k,x_{k+1},x_{k+2},x_{k+3}\rfloor \\
	\hline
	0  &   8.0  &  2.079442  &          &              &\\
 	   &      &     &     0.117783     &              &\\ 
	1  &   9.0  &  2.197225  &           &       \fbox{  ア  }      &\\
 	   &      &     &     0.105360     &              & 0.0003955000 \\
 	2  &  10.0  &  2.302585  &           &       -0.0050250      &\\ 
 	   &      &     &     0.095310    &              &\\ 
 	3  &  11.0  &  2.397895 &           &              &\\ 
	\hline
      \end{array}
\end{equation*}
それぞれの項は,例えば,
\begin{equation*}
f_2\lfloor x_1,x_2,x_3 \rfloor =\frac{0.095310-0.105360}{11.0-9.0}=-0.0050250
\end{equation*}
で求められる.ニュートンの差分商の一次多項式は
\begin{equation*}
F(x)=F_0(8.0)+(x-x_0)f_1\lfloor x_1,x_0\rfloor =2.079442+0.117783(9.2-8.0)=2.220782
\end{equation*}
となる.

\begin{enumerate}
\item 差分商補間の表中の開いている箇所\fbox{ ア $f_2\lfloor x_0,x_1,x_2 \rfloor$ }を埋めよ.
\item ニュートンの二次多項式
\begin{equation*}
F (x )=F (x _{0})+(x -x _{0})f _{1}\lfloor x_0,x_1\rfloor+(x -x _{0})(x -x _{1})
f _{2}\lfloor x_0,x_1,x_2\rfloor
\end{equation*}
の値を求めよ.
\item ニュートンの三次多項式の値を求めよ.
\end{enumerate}
ただし,ここでは有効数字7桁程度はとるように.(E.クライツィグ著「数値解析」(培風館,2003), p.31, 例4改)

\item exp(0)=1.0, exp(0.1)=1.1052, exp(0.3)=1.3499の値を用いて,ラグランジュ補間法により,exp(0.2)の値を推定せよ.(2009年度期末試験)
\item 次の関数
\begin{equation*}
f(x) = \frac{4}{1+x^2}
\end{equation*}
を$x = 0..1$で数値積分する.
\begin{enumerate}
\item $N$を2,4,8,…256ととり,$N$個の等間隔な区間にわけて中点法で求めよ.(15)
\item 小数点以下10桁まで求めた値3.141592654との差をdXとする.dXと分割数Nとを両対数プロット(loglogplot)して比較せよ(10)
\end{enumerate}
(2008年度期末試験)
\item 次の関数
\begin{equation*}
y = \frac{1}{1+x^2}
\end{equation*}
を$x = 0..1$で等間隔に$N+1$点とり,$N$個の区間にわけて数値積分で求める.$N$を2, 4, 8, 16, 32, 64, 128, 256と取ったときの(a)中点法, (b)台形公式, (c)シンプソン公式それぞれの収束性を比較せよ.

ヒント:Maple script にあるそれぞれの数値積分法を関数 (procedure) に直して,for-loop
で回せば楽.出来なければ,一つ一つ手で変えても OK. 両対数プロット (loglogplot) すると見やすい.
\end{enumerate}