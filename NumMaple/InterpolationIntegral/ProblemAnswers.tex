\begin{itemize}
\item[解答例[2.]]
少し違うデータの場合の結果を以下に示す.データの組は以下のとおり.
\begin{MapleInput}
> restart;
X:=[0,1,3,2];
Y:=[1,2,-2,3];
with(LinearAlgebra):
list1:=[X,Y];
\end{MapleInput}

\begin{MapleOutputGather}
\displaystyle X\, := \,[0,1,3,2] \notag \\
\displaystyle Y\, := \,[1,2,-2,3] \notag \\
\displaystyle {\it list1}\, := \,[[0,1,3,2],[1,2,-2,3]] \notag
\end{MapleOutputGather}
グラフにすると次の通り.
\begin{MapleInput}
> with(plots):
l1p:=pointplot(Transpose(Matrix(list1))):
display(l1p);
\end{MapleInput}
\MaplePlot{50mm}{./figures/C6_NewtonsDividedDifferenceplot2d1.eps}
2次,3次までの関数をそれぞれF2,F3として,Newtonの差分商公式を使った関数を定義する.
\begin{MapleInput}
> F2:=y0+(x-x0)*f1_01+(x-x0)*(x-x1)*f2_012;
F3:=y0+(x-x0)*f1_01+(x-x0)*(x-x1)*f2_012+(x-x0)*(x-x1)*(x-x2)*f3_0123;
\end{MapleInput}

\begin{MapleOutputGather}
F2 := y0+ \left( x-x0 \right) f1\_01+ \left( x- x0 \right)  \left( x- x1 \right)f2\_012 \notag \\
F3 := y0+ \left( x- x0 \right)f1\_01+ \left( x- x0\right)  \left( x- x1 \right)f2\_012\notag \\
+ \left( x- x0 \right)  \left( x-x1 \right)  \left( x- x2 \right)f3\_0123 \notag
\end{MapleOutputGather}
差分商$f_1, f_2, f_3$の公式をそのまま書き下す.
\begin{MapleInput}
> f1_01:=(y1-y0)/(x1-x0);
f1_12:=(y2-y1)/(x2-x1);
f1_23:=(y3-y2)/(x3-x2);
\end{MapleInput}

\begin{MapleOutputGather}
{\it f1\_01}\, := \,{\frac {{\it y1}-{\it y0}}{{\it x1}-{\it x0}}} \notag \\
{\it f1\_12}\, := \,{\frac {{\it y2}-{\it y1}}{{\it x2}-{\it x1}}}\notag \\
{\it f1\_23}\, := \,{\frac {{\it y3}-{\it y2}}{{\it x3}-{\it x2}}}\notag 
\end{MapleOutputGather}

\begin{MapleInput}
> f2_012:=(f1_12-f1_01)/(x2-x0);
f2_123:=(f1_23-f1_12)/(x3-x1);
\end{MapleInput}

\begin{MapleOutputGather}
 {\it f2\_012}\, := \frac {1}{{\it x2}-{\it x0}} \left( {\frac {{\it y2}-{\it y1}}{{\it x2}-{\it x1}}}-{\frac {{\it y1}-{\it y0}}{{\it x1}-{\it x0}}} \right)  \notag \\
{\it f2\_123}\, := \frac {1}{{\it x3}-{\it x1}} \left( {\frac {{\it y3}-{\it y2}}{{\it x3}-{\it x2}}}-{\frac {{\it y2}-{\it y1}}{{\it x2}-{\it x1}}} \right)  \notag 
\end{MapleOutputGather}

\begin{MapleInput}
> f3_0123:=(f2_123-f2_012)/(x3-x0); 
\end{MapleInput}
\begin{MapleOutputGather}
{\it f3\_0123}\, := \,{\frac {1}{{\it x3}-{\it x0}} \left( {\frac {1}{{\it x3}-{\it x1}} \left( {\frac {{\it y3}-{\it y2}}{{\it x3}-{\it x2}}}-{\frac {{\it y2}-{\it y1}}{{\it x2}-{\it x1}}} 
 \right) }-{\frac {1}{{\it x2}-{\it x0}} \left( {\frac {{\it y2}-{\it y1}}{{\it x2}-{\it x1}}}-{\frac {{\it y1}-{\it y0}}{{\it x1}-{\it x0}}} \right) } \right) } \notag
\end{MapleOutputGather}
こうすると$F_2$は
\begin{MapleInput}
F2;
\end{MapleInput}

\begin{MapleOutputGather}
{\it y0}+{\frac { \left( x-{\it x0} \right)  \left( {\it y1}-{\it y0} \right) }{{\it x1}-{\it x0}}}+{\frac { \left( x-{\it x0} \right)  \left( x-{\it x1} \right) }{{\it x2}-{\it x0}} \left( {\frac {{\it y2}-{\it y1}}{{\it x2}-{\it x1}}}-{\frac {{\it y1}-{\it y0}}{{\it x1}-{\it x0}}} \right) } \notag
\end{MapleOutputGather}
となる.さて,データを追加.次の||はx0,x1,x2,x3を作る.
\begin{MapleInput}
> for i from 1 to 4 do
  x||(i-1):=X[i];
  y||(i-1):=Y[i];
end:
\end{MapleInput}

値を入れるとF2,F3は簡単な数式となる.

\begin{MapleInput}
F2;
F3;
\end{MapleInput}

\begin{MapleOutputGather}
1+x-x \left( x-1 \right) \notag \\
 1+x-x \left( x-1 \right) -x \left( x-1 \right)  \left( x-3 \right) \notag
\end{MapleOutputGather}

\begin{MapleInput}
> eq2:=expand(F2);
eq3:=expand(F3);
\end{MapleInput}

\begin{MapleOutputGather}
{\it eq2}\, := \,-{x}^{2}+2\,x+1 \notag \\
{\it eq3}\, := \,-{x}^{3}+3\,{x}^{2}-x+1 \notag
\end{MapleOutputGather}
これでNewtonの差分商公式による内挿関数が求まっている.プロットすると次の通り.
\begin{MapleInput}
> with(plots):
l1p:=pointplot(Transpose(Matrix(list1))):
pf2:=plot(eq2,x=0..3,color=blue):
pf3:=plot(eq3,x=0..3):
display(l1p,pf2,pf3);
\end{MapleInput}
\MaplePlot{50mm}{./figures/C6_NewtonsDividedDifferenceplot2d2.eps}


\item[解答例[6.]]
関数を定義.
\begin{MapleInput}
> restart;
f1:=unapply(4/(1+x^2),x);
\end{MapleInput}

\begin{MapleOutput}
x \rightarrow \frac{4}{x^2+1}
\end{MapleOutput}
plotしてみる.
\begin{MapleInput}
> plot(f1(x),x);
\end{MapleInput}
\MaplePlot{50mm}{./figures/C6_IntegralAppendixplot2d1.eps}
Mapleで答えを確認.
\begin{MapleInput}
> int(f1(x),x=0..1);
\end{MapleInput}
\begin{MapleOutput}
\pi
\end{MapleOutput}

中点公式をcopyする.幾つかの初期値を設定.
\begin{MapleInput}
> N:=8: x0:=0: xn:=1: Digits:=20:
> mid:=proc(N)
  global x0,xn;
  local S,i,dS,h,xi;
  h:=(xn-x0)/N: 
  S:=0:
  for i from 0 to N-1 do
    xi:=x0+(i+1/2)*h;
    dS:=h*f1(xi);
    S:=S+dS;
  end do:
  evalf(S);
end;
\end{MapleInput}
resultsという配列を用意して,分割点の数と中点公式で得られた結果をためていく.正しい答え($\pi$)との差をとるが,あとでloglogplotするので,誤差の絶対値をためておく.
\begin{MapleInput}
> results:=[];
for i from 1 to 8 do
 results:=[op(results),[2^i,evalf(abs(mid(2^i)-Pi))]];
end:
\end{MapleInput}
loglogplotの結果.正しい解への収束の様子が表示される.
\begin{MapleInput}
with(plots):
loglogplot([results]);
\end{MapleInput}
\MaplePlot{50mm}{./figures/C6_IntegralAppendixplot2d2.eps}

\end{itemize}