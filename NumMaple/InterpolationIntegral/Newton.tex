もう一つ有名なNewton(ニュートン) の内挿公式は,
\begin{equation*}
\begin{array}{rc}
F (x )&=F (x _{0})+
(x -x _{0})f _{1}\lfloor x_0,x_1\rfloor+
(x -x _{0})(x -x _{1})
f _{2}\lfloor x_0,x_1,x_2\rfloor + \\
& \cdots + \displaystyle \prod_{i=0}^{n-1} (x-x_i) \, 
f_n \lfloor x_0,x_1,\cdots,x_n \rfloor
\end{array}
\end{equation*}
となる.ここで$f_i \lfloor\, \rfloor$ は次のような関数を意味していて,
\begin{equation*}
\begin{array}{rl}
f _{1}\lfloor x_0,x_1\rfloor &= \displaystyle \frac{y_1-y_0}{x_1-x_0} \\
f _{2}\lfloor x_0,x_1,x_2\rfloor &= \displaystyle \frac{f _{1}\lfloor x_1,x_2\rfloor-
f _{1}\lfloor x_0,x_1\rfloor}{x_2-x_0} \\
\vdots & \\
f _{n}\lfloor x_0,x_1,\cdots,x_n\rfloor &= \displaystyle \frac{f_{n-1}\lfloor x_1,x_2\cdots,x_{n}\rfloor-
f _{n-1}\lfloor x_0,x_1,\cdots,x_{n-1}\rfloor}{x_n-x_0} \\
\end{array}
\end{equation*}
差分商と呼ばれる.得られた多項式は,Lagrange の内挿公式で得られたものと当然一致する.Newtonの内挿公式の利点は,新たなデータ点が増えたときに,新たな項を加えるだけで,内挿式が得られる点である.

\subsection{Newton補間と多項式補間の一致の検証}
関数$F(x)$を$x$の多項式として展開.その時の,係数の取るべき値と,差分商で得られる値が一致.
\begin{MapleInput}
> restart: F:=x->f0+(x-x0)*f1p+(x-x0)*(x-x1)*f2p;
\end{MapleInput}
\begin{MapleOutput}
F\, := \,x\mapsto f0+ ( x-x0 ) f1p+ ( x-x0 )  ( x-x1 ) f2p
\end{MapleOutput}
\begin{MapleInput}
> F(x1); 
  sf1p:=solve(F(x1)=f1,f1p);
\end{MapleInput}
\begin{MapleOutputGather}
f0+ ( x1-x0 ) f1p \notag \\
\displaystyle sf1p\, := \,{\frac {f0-f1}{-x1+x0}} \notag
\end{MapleOutputGather}
f20の取るべき値の導出
\begin{MapleInput}
> sf2p:=solve(F(x2)=f2,f2p); 
  fac_f2p:=factor(subs(f1p=sf1p,sf2p));
\end{MapleInput}
\begin{MapleOutputGather}
sf2p\, := \displaystyle -{\frac {f0+f1p\,x2-f1p\,x0-f2}{ ( -x2+x0 ) 
 ( -x2+x1 ) }} \notag \\
\displaystyle {\it fac\_f2p}\, := \displaystyle {\frac {f0\,x1-x2\,f0+x2\,f1-x0\,f1-f2\,x1+f2\,x0}{ ( -x1+x0 )  ( -x2+x0 )  ( -x2+x1 ) }} \notag
\end{MapleOutputGather}
ニュートンの差分商公式を変形
\begin{MapleInput}
> ff11:=(f0-f1)/(x0-x1); 
  ff12:=(f1-f2)/(x1-x2); 
  ff2:=(ff11-ff12)/(x0-x2);
  fac_newton:=factor(ff2);
\end{MapleInput}
\begin{MapleOutputGather}
ff11:= {\displaystyle \frac {f0-f1}{-x1+x0}} \notag \\
ff12 := {\displaystyle \frac {f1-f2}{-x2+x1}} \notag \\
ff2 :=  \frac{ {\displaystyle \frac {f0-f1}{-x1+x0}}-{\displaystyle \frac {f1-f2}{-x2+x1}} }{-x2+x0 }\notag \\
{\it fac\_newton} := {\displaystyle \frac {f0\,x1-x2\,f0+x2\,f1-x0\,f1-f2\,x1+f2\,x0}{ ( -x1+x0 )  ( -x2+x0 )  ( -x2+x1 ) }} \notag 
\end{MapleOutputGather}

二式が等しいかどうかをevalbで判定
\begin{MapleInput}
> evalb(fac_f2p=fac_newton);
\end{MapleInput}
\begin{MapleOutput}
true
\end{MapleOutput}
