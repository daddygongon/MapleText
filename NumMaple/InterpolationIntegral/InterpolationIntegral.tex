\documentclass[10pt,a4j]{jreport}
\usepackage[dvips]{graphicx,color}
\usepackage{verbatim}
\usepackage{amsmath,amsthm,amssymb}
\topmargin -15mm\oddsidemargin -4mm\evensidemargin\oddsidemargin
\textwidth 170mm\textheight 257mm\columnsep 7mm
\setlength{\fboxrule}{0.2ex}
\setlength{\fboxsep}{0.6ex}

\pagestyle{empty}

\newcommand{\MaplePlot}[2]{{\begin{center}
    \includegraphics[width=#1,clip]{#2}
                     \end{center}
%
} }

\newenvironment{MapleInput}{%
    \color{red}\verbatim
}{%
    \endverbatim
}

\newenvironment{MapleError}{%
    \color{blue}\verbatim
}{%
    \endverbatim
}

\newenvironment{MapleOutput}{%
    \color{blue}\begin{equation*}
}{%
    \end{equation*}
}

\newenvironment{MapleOutputGather}{%
    \color{blue}\gather
}{%
    \endgather
}
\newif\ifHIKI
%\HIKItrue % TRUEの設定
\HIKIfalse % FALSEの設定
\begin{document}
\chapter{補間(interpolation)と数値積分(Integral)}
\section{概要:補間と近似}
数値計算の中心課題の一つである,行列に関する演算について見ていく.多次元,大規模な行列に対する効率のよい計算法が多数開発されており,多くの既存のライブラリが用意
されている.本章ではそれらの中心をなす,逆行列(matrix inverse)と固有値(Eigen
values)に関して具体的な計算方法を示す.現実的な問題には既存のライブラリを使うのが上策であるが,それでも基礎となる原理の理解や,ちょっとした計算,ライブラ
リの結果の検証に使えるルーチンを示す.

逆行列は連立一次方程式を解くことと等価である.ルーチン的なやり方にガウスの消去法がある.これは上三角行列になれば代入を適宜おこなうことで解が容易に求まることを利
用する.さらに,初期値から始めて次々に解に近づけていく反復法がある.この代表例であるJacobi(ヤコビ)法と,収束性を高めたGauss-Seidel(ガウス-
ザイデル)法を紹介する.

上記の手法をより高速にした修正コレスキー分解と共役傾斜(共役勾配)法があるが,少し複雑になるので割愛する.必要ならばNumRecipeを読め.
\section{多項式補間(polynomial interpolation)}
データを単純に多項式で補間する方法を先ず示そう.$N+1$点をN次の多項式でつなぐ.この場合の補間関数は,
\begin{equation*}
F \left(x \right)={\sum_{i=0}^{N} } a _{i }x ^{i }=a_{0}+a_{1}x +a_{2}x^{2}+\cdots +a_{N}x^{N}
\end{equation*}
である.データの点を$(x_{i},\,y_{i}),i=0..N$とすると
\begin{equation*}
\begin{array}{cl}
a _{0}+a _{1}x _{0}+a _{2}x _{0}^{2}+\cdots +a _{N }x _{0}^{N }& =y _{0}\\
a _{0}+a _{1}x _{1}+a _{2}x _{1}^{2}+\cdots +a _{N }x _{1}^{N }& =y _{1}\\
\vdots& \\
a _{0}+a _{1}x _{N}+a _{2}x _{N}^{2}+\cdots +a _{N }x _{N}^{N }& =y _{N}
\end{array}
\end{equation*}
が,係数 $a_i$を未知数と見なした線形の連立方程式となっている.係数行列は
\begin{equation*}
A=\left[
\begin{array}{ccccc}
1&x_0&x_0^2&\cdots&x_0^N \\
1&x_1&x_1^2&\cdots&x_1^N \\
\vdots& & & & \vdots \\
1&x_N&x_N^2&\cdots&x_N^N 
\end{array} \right]
\end{equation*}
となる.$a_i$と$y_i$をそれぞれベクトルとみなすと
\ifHIKI %%%%
||               ||
\else %%%%
\begin{equation*}
\setlength{\unitlength}{1cm}
\begin{picture}(10,3.5)
\put(0,0){\framebox(10,3.5){}}
\end{picture}
\end{equation*}
\fi %%%%
により未知数ベクトル$a_i$が求まる.これは単純に,前に紹介した Gauss の消去法や LU 分解で解ける.

\subsection{Mapleによる多項式補間の実例}
\begin{MapleInput}
> restart; X:=[0,1,2,3]: Y:=[1,2,3,-2]:
> with(LinearAlgebra):
> list1:=[X,Y];
\end{MapleInput}
\begin{MapleOutput}
{\it list1}\, := \,[[0,1,2,3],[1,2,3,-2]]
\end{MapleOutput}
\begin{MapleInput}
> with(plots):
  l1p:=listplot(Transpose(Matrix(list1))):
  display(l1p);
\end{MapleInput}
\MaplePlot{50mm}{./figures/C7_InterpolationIntegralplot2d1.eps}

\begin{MapleInput}
> A:=Matrix(4,4): 
  for i from 1 to 4 do 
    for j from 1 to 4 do 
      A[i,j]:=X[i]^(j-1);
    end do; 
  end do:
  A;
\end{MapleInput}
\begin{MapleOutput}
\left[ \begin {array}{cccc} 1&0&0&0\\ 1&1&1&1\\ 1&2&4&8\\ 1&3&9&27\end {array} \right]
\end{MapleOutput}

\begin{MapleInput}
> a1:=MatrixInverse(A).Vector(Y);
\end{MapleInput}
\begin{MapleOutput}
{\it a1}\, := \, \left[ \begin {array}{c} 1\\ -1\\ 3\\ -1\end {array} \right]
\end{MapleOutput}
\begin{MapleInput}
> f1:=unapply(add(a1[ii]*x^(ii-1),ii=1..4),x);
\end{MapleInput}
\begin{MapleOutput}
{\it f1}\, := \,x\mapsto 1-x+3\,{x}^{2}-{x}^{3}
\end{MapleOutput}
\begin{MapleInput}
> f1p:=plot(f1(x),x=0..3): 
  l1p:=listplot(Transpose(Matrix(list1))):
  display(f1p,l1p);
\end{MapleInput}
\MaplePlot{50mm}{./figures/C7_InterpolationIntegralplot2d2.eps}

\section{Lagrange(ラグランジュ) の内挿公式}
多項式補間は手続きが簡単であるため,計算間違いが少なく,プログラムとして組むのに適している.しかし,あまり"みとうし"のよい方法とはいえない.その点,Lagrange(ラグランジュ)の内挿公式は見通しがよい.これは
\begin{equation*}
F(x)= \sum_{k=0}^N \frac{\displaystyle \prod_{j \ne k} (x-x_j)}
{\displaystyle \prod_{j \ne k} (x_k-x_j)} y_k
=\sum_{k=0}^N \frac{ \frac{\displaystyle (x-x_0)(x-x_1)\cdots(x-x_N)}{\displaystyle (x-x_k)}}
{\frac{\displaystyle (x_k-x_0)(x_k-x_1)\cdots(x_k-x_N)}{\displaystyle (x_k-x_k)}} y_k
\end{equation*}
と表わされる.数学的に 2つ目の表記は間違っているが,先に割り算を実行すると読み取って欲しい.これは一見複雑に見えるが,単純な発想から出発している.求めたい関数$F(x)$を
\begin{equation*}
F(x) = y_0 L_0(x)+y_1 L_1(x)+y_2 L_2(x)
\end{equation*}
とすると
\begin{equation*}
\begin{array}{ccc}
 L_0(x_0) = 1 &L_0(x_1) = 0 &L_0(x_2) = 0 \\
 L_1(x_0) = 0 &L_1(x_1) = 1 &L_1(x_2) = 0 \\
 L_2(x_0) = 0 &L_2(x_1) = 0 &L_2(x_2) = 1 
\end{array}
\end{equation*}
となるように関数$L_i(x)$を決めればよい.これを以下のようにとればLagrangeの内挿公式となる.
  
\ifHIKI %%%%
||               ||
\else %%%%
\begin{equation*}
\setlength{\unitlength}{1cm}
\begin{picture}(10,10)
\put(0,0){\framebox(10,10){}}
\end{picture}
\end{equation*}
\fi %%%%

\section{Newton(ニュートン) の差分商公式}
もう一つ有名なNewton(ニュートン) の内挿公式は,
\begin{equation*}
\begin{array}{rc}
F (x )&=F (x _{0})+
(x -x _{0})f _{1}\lfloor x_0,x_1\rfloor+
(x -x _{0})(x -x _{1})
f _{2}\lfloor x_0,x_1,x_2\rfloor + \\
& \cdots + \displaystyle \prod_{i=0}^{n-1} (x-x_i) \, 
f_n \lfloor x_0,x_1,\cdots,x_n \rfloor
\end{array}
\end{equation*}
となる.ここで$f_i \lfloor\, \rfloor$ は次のような関数を意味していて,
\begin{equation*}
\begin{array}{rl}
f _{1}\lfloor x_0,x_1\rfloor &= \displaystyle \frac{y_1-y_0}{x_1-x_0} \\
f _{2}\lfloor x_0,x_1,x_2\rfloor &= \displaystyle \frac{f _{1}\lfloor x_1,x_2\rfloor-
f _{1}\lfloor x_0,x_1\rfloor}{x_2-x_0} \\
\vdots & \\
f _{n}\lfloor x_0,x_1,\cdots,x_n\rfloor &= \displaystyle \frac{f_{n-1}\lfloor x_1,x_2\cdots,x_{n}\rfloor-
f _{n-1}\lfloor x_0,x_1,\cdots,x_{n-1}\rfloor}{x_n-x_0} \\
\end{array}
\end{equation*}
差分商と呼ばれる.得られた多項式は,Lagrange の内挿公式で得られたものと当然一致する.Newtonの内挿公式の利点は,新たなデータ点が増えたときに,新たな項を加えるだけで,内挿式が得られる点である.

\subsection{Newton補間と多項式補間の一致の検証}
関数$F(x)$を$x$の多項式として展開.その時の,係数の取るべき値と,差分商で得られる値が一致.
\begin{MapleInput}
> restart: F:=x->f0+(x-x0)*f1p+(x-x0)*(x-x1)*f2p;
\end{MapleInput}
\begin{MapleOutput}
F\, := \,x\mapsto f0+ ( x-x0 ) f1p+ ( x-x0 )  ( x-x1 ) f2p
\end{MapleOutput}
\begin{MapleInput}
> F(x1); 
  sf1p:=solve(F(x1)=f1,f1p);
\end{MapleInput}
\begin{MapleOutputGather}
f0+ ( x1-x0 ) f1p \notag \\
\displaystyle sf1p\, := \,{\frac {f0-f1}{-x1+x0}} \notag
\end{MapleOutputGather}
f20の取るべき値の導出
\begin{MapleInput}
> sf2p:=solve(F(x2)=f2,f2p); 
  fac_f2p:=factor(subs(f1p=sf1p,sf2p));
\end{MapleInput}
\begin{MapleOutputGather}
sf2p\, := \displaystyle -{\frac {f0+f1p\,x2-f1p\,x0-f2}{ ( -x2+x0 ) 
 ( -x2+x1 ) }} \notag \\
\displaystyle {\it fac\_f2p}\, := \displaystyle {\frac {f0\,x1-x2\,f0+x2\,f1-x0\,f1-f2\,x1+f2\,x0}{ ( -x1+x0 )  ( -x2+x0 )  ( -x2+x1 ) }} \notag
\end{MapleOutputGather}
ニュートンの差分商公式を変形
\begin{MapleInput}
> ff11:=(f0-f1)/(x0-x1); 
  ff12:=(f1-f2)/(x1-x2); 
  ff2:=(ff11-ff12)/(x0-x2);
  fac_newton:=factor(ff2);
\end{MapleInput}
\begin{MapleOutputGather}
ff11:= {\displaystyle \frac {f0-f1}{-x1+x0}} \notag \\
ff12 := {\displaystyle \frac {f1-f2}{-x2+x1}} \notag \\
ff2 :=  \frac{ {\displaystyle \frac {f0-f1}{-x1+x0}}-{\displaystyle \frac {f1-f2}{-x2+x1}} }{-x2+x0 }\notag \\
{\it fac\_newton} := {\displaystyle \frac {f0\,x1-x2\,f0+x2\,f1-x0\,f1-f2\,x1+f2\,x0}{ ( -x1+x0 )  ( -x2+x0 )  ( -x2+x1 ) }} \notag 
\end{MapleOutputGather}

二式が等しいかどうかをevalbで判定
\begin{MapleInput}
> evalb(fac_f2p=fac_newton);
\end{MapleInput}
\begin{MapleOutput}
true
\end{MapleOutput}

\section{数値積分 (Numerical integration)}
\documentclass[10pt,a4j]{jreport}
\usepackage[dvips]{graphicx,color}
\usepackage{verbatim}
\usepackage{amsmath,amsthm,amssymb}
\topmargin -15mm\oddsidemargin -4mm\evensidemargin\oddsidemargin
\textwidth 170mm\textheight 257mm\columnsep 7mm
\setlength{\fboxrule}{0.2ex}
\setlength{\fboxsep}{0.6ex}

\pagestyle{empty}

\newcommand{\MaplePlot}[2]{{\begin{center}
    \includegraphics[width=#1,clip]{#2}
                     \end{center}
%
} }

\newenvironment{MapleInput}{%
    \color{red}\verbatim
}{%
    \endverbatim
}

\newenvironment{MapleError}{%
    \color{blue}\verbatim
}{%
    \endverbatim
}

\newenvironment{MapleOutput}{%
    \color{blue}\begin{equation*}
}{%
    \end{equation*}
}

\newenvironment{MapleOutputGather}{%
    \color{blue}\gather
}{%
    \endgather
}
\newcommand{\ChartElement}[1]{{
	\color{magenta}\begin{flushleft}$\left[\left[\left[\textbf{\large #1}\right]\right]\right]$
	\end{flushleft}\vspace{-10mm}
} }

\newcommand{\ChartElementTwo}[1]{{
	\color{magenta}\begin{flushleft}$\left[\left[\left[\textbf{\large #1}\right]\right]\right]$
	\end{flushleft}
} }
\begin{document}
\pagebreak
\section{積分(int)}
\ChartElement{解説}

\subsection{\textbf{�ۑ�}}
\subsubsection{\textbf{\textit{1. ���̕s��ϕ������߂�D}}}
\begin{maplegroup}
\begin{Maple Normal}{
i)\mapleinline{inert}{2d}{int(4*x+3, x)}{$\displaystyle \int \!4\,x+3{dx}$}
�Cii)\mapleinline{inert}{2d}{int(1/(1+e^x), x)}{$\displaystyle \int \! \left( 1+{e}^{x} \right) ^{-1}{dx}$}
�Ciii)\mapleinline{inert}{2d}{int(1/(e^(-x)+e^x), x)}{$\displaystyle \int \! \left( {e}^{-x}+{e}^{x} \right) ^{-1}{dx}$}
�Civ)\mapleinline{inert}{2d}{int(sqrt(1-x^2), x)}{$\displaystyle \int \! \sqrt{1-{x}^{2}}{dx}$}
}\end{Maple Normal}

\end{maplegroup}
\subsubsection{\textbf{\textit{2. ���̒�ϕ������߂�D}}}
\begin{maplegroup}
\begin{Maple Normal}{
i)\mapleinline{inert}{2d}{int(sin*x, x = 0 .. Pi)}{$\displaystyle \int _{0}^{\pi }\! \left( \sin \right) \,x{dx}$}
�Cii)\mapleinline{inert}{2d}{int(arctan*x, x = 0 .. 1)}{$\displaystyle \int _{0}^{1}\! \left( \arctan \right) \,x{dx}$}
�Ciii)\mapleinline{inert}{2d}{int(1/sqrt(4-x^2), x = -2 .. 2)}{$\displaystyle \int _{-2}^{2}\! \left(  \sqrt{4-{x}^{2}} \right) ^{-1}{dx}$}
�Civ)\mapleinline{inert}{2d}{int(1/(x^2+x+1), x = 0 .. 1)}{$\displaystyle \int _{0}^{1}\! \left( {x}^{2}+x+1 \right) ^{-1}{dx}$}
}\end{Maple Normal}

\end{maplegroup}
\subsubsection{\textbf{\textit{3.(���W�ۑ�j����2�d�ϕ������߂�D}}}
\begin{maplegroup}
\begin{Maple Normal}{
\mapleinline{inert}{2d}{Typesetting:-mrow(Typesetting:-msubsup(Typesetting:-mo("&Integral;", mathvariant = "normal", fence = "unset", separator = "unset", stretchy = "true", symmetric = "unset", largeop = "true", movablelimits = "unset", accent = "unset", lspace = "0.0em", rspace = "0.0em"), Typesetting:-mi(""), Typesetting:-mi(""), superscriptshift = "0", subscriptshift = "0"), Typesetting:-msubsup(Typesetting:-mo("&Integral;", mathvariant = "normal", fence = "unset", separator = "unset", stretchy = "true", symmetric = "unset", largeop = "true", movablelimits = "unset", accent = "unset", lspace = "0.0em", rspace = "0.0em"), Typesetting:-mrow(Typesetting:-mi("D", italic = "false", mathvariant = "normal")), Typesetting:-mi(""), superscriptshift = "0", subscriptshift = "0"), Typesetting:-msqrt(Typesetting:-mrow(Typesetting:-msup(Typesetting:-mi("x", italic = "true", mathvariant = "italic"), Typesetting:-mrow(Typesetting:-mn("2", mathvariant = "normal")), superscriptshift = "0"), Typesetting:-mo("+", mathvariant = "normal", fence = "false", separator = "false", stretchy = "false", symmetric = "false", largeop = "false", movablelimits = "false", accent = "false", lspace = "0.2222222em", rspace = "0.2222222em"), Typesetting:-msup(Typesetting:-mi("y", italic = "true", mathvariant = "italic"), Typesetting:-mrow(Typesetting:-mn("2", mathvariant = "normal")), superscriptshift = "0"))), Typesetting:-mi(""), Typesetting:-mspace(height = "0.0ex", width = "0.5em", depth = "0.0ex", linebreak = "auto"), Typesetting:-mo("&DifferentialD;", mathvariant = "normal", fence = "unset", separator = "unset", stretchy = "unset", symmetric = "unset", largeop = "unset", movablelimits = "unset", accent = "unset", lspace = "0.0em", rspace = "0.0em"), Typesetting:-mi("x", italic = "true", mathvariant = "italic"), Typesetting:-mo(" ", mathvariant = "normal", fence = "false", separator = "false", stretchy = "false", symmetric = "false", largeop = "false", movablelimits = "false", accent = "false", lspace = "0.0em", rspace = "0.0em"), Typesetting:-mo("&DifferentialD;", mathvariant = "normal", fence = "unset", separator = "unset", stretchy = "unset", symmetric = "unset", largeop = "unset", movablelimits = "unset", accent = "unset", lspace = "0.0em", rspace = "0.0em"), Typesetting:-mi("y", italic = "true", mathvariant = "italic"), Typesetting:-mo(" ", mathvariant = "normal", fence = "false", separator = "false", stretchy = "false", symmetric = "false", largeop = "false", movablelimits = "false", accent = "false", lspace = "0.0em", rspace = "0.0em"), Typesetting:-mi(""), Typesetting:-mi("D", italic = "false", mathvariant = "normal"), Typesetting:-mo(":", mathvariant = "normal", fence = "false", separator = "false", stretchy = "false", symmetric = "false", largeop = "false", movablelimits = "false", accent = "false", lspace = "0.2777778em", rspace = "0.2777778em"), Typesetting:-mn("0", mathvariant = "normal"), Typesetting:-mo("&leq;", mathvariant = "normal", fence = "false", separator = "false", stretchy = "false", symmetric = "false", largeop = "false", movablelimits = "false", accent = "false", lspace = "0.2777778em", rspace = "0.2777778em"), Typesetting:-mi("y", italic = "true", mathvariant = "italic"), Typesetting:-mo("&leq;", mathvariant = "normal", fence = "false", separator = "false", stretchy = "false", symmetric = "false", largeop = "false", movablelimits = "false", accent = "false", lspace = "0.2777778em", rspace = "0.2777778em"), Typesetting:-mi("x", italic = "true", mathvariant = "italic"), Typesetting:-mo("&leq;", mathvariant = "normal", fence = "false", separator = "false", stretchy = "false", symmetric = "false", largeop = "false", movablelimits = "false", accent = "false", lspace = "0.2777778em", rspace = "0.2777778em"), Typesetting:-mn("1", mathvariant = "normal"))}{$\displaystyle \int _{}^{}\int _{\mathop{\rm D}}^{}\sqrt{x ^{2}+y ^{2}}d x \mathop{\rm  }d y \mathop{\rm  }\mathop{\rm D}:0\le y \le x \le 1$}
}\end{Maple Normal}

\end{maplegroup}
\begin{maplegroup}
\newpage
\end{maplegroup}
\subsection{\textbf{�𓚗�}}
\subsubsection{\textbf{\textit{1.}}}
\begin{maplegroup}
\begin{mapleinput}
\mapleinline{active}{1d}{int(4*x+3,x);
int( 1/(1+exp(x)),x);
int(1/(exp(-x)+exp(x)),x);
int(sqrt(1-x\symbol{94}2),x);
}{}
\end{mapleinput}
\mapleresult
\begin{maplelatex}
\mapleinline{inert}{2d}{2*x^2+3*x}{\[\displaystyle 2\,{x}^{2}+3\,x\]}
\end{maplelatex}
\mapleresult
\begin{maplelatex}
\mapleinline{inert}{2d}{-ln(1+exp(x))+ln(exp(x))}{\[\displaystyle -\ln  \left( 1+{{\rm e}^{x}} \right) +\ln  \left( {{\rm e}^{x}} \right) \]}
\end{maplelatex}
\mapleresult
\begin{maplelatex}
\mapleinline{inert}{2d}{arctan(exp(x))}{\[\displaystyle \arctan \left( {{\rm e}^{x}} \right) \]}
\end{maplelatex}
\mapleresult
\begin{maplelatex}
\mapleinline{inert}{2d}{(1/2)*x*sqrt(1-x^2)+(1/2)*arcsin(x)}{\[\displaystyle 1/2\,x \sqrt{1-{x}^{2}}+1/2\,\arcsin \left( x \right) \]}
\end{maplelatex}
\end{maplegroup}
\subsubsection{\textbf{\textit{2.}}}
\begin{maplegroup}
\begin{mapleinput}
\mapleinline{active}{1d}{int(sin(x),x=0..Pi);
int(arctan(x),x=0..1);
int(1/(sqrt(4-x\symbol{94}(2))),x=-2..2);
int(1/(x\symbol{94}2+x+1),x=0..1);
}{}
\end{mapleinput}
\mapleresult
\begin{maplelatex}
\mapleinline{inert}{2d}{2}{\[\displaystyle 2\]}
\end{maplelatex}
\mapleresult
\begin{maplelatex}
\mapleinline{inert}{2d}{(1/4)*Pi-(1/2)*ln(2)}{\[\displaystyle 1/4\,\pi -1/2\,\ln  \left( 2 \right) \]}
\end{maplelatex}
\mapleresult
\begin{maplelatex}
\mapleinline{inert}{2d}{Pi}{\[\displaystyle \pi \]}
\end{maplelatex}
\mapleresult
\begin{maplelatex}
\mapleinline{inert}{2d}{(1/9)*Pi*sqrt(3)}{\[\displaystyle 1/9\,\pi \, \sqrt{3}\]}
\end{maplelatex}
\end{maplegroup}
\subsubsection{\textbf{\textit{3.}}}
\begin{maplegroup}
\begin{mapleinput}
\mapleinline{active}{1d}{with(plots):
}{}
\end{mapleinput}
\end{maplegroup}
\begin{maplegroup}
\begin{mapleinput}
\mapleinline{active}{1d}{inequal(\{x-y>=0,x>=0,x<=1,y>=0\},x=-0.5..1.5,y=-0.5..1.5,optionsexcluded=(color=white));
}{}
\end{mapleinput}
\mapleresult
\mapleplot{Intplot2d1.eps}
\end{maplegroup}
\begin{maplegroup}
\begin{mapleinput}
\mapleinline{active}{1d}{f:=unapply(sqrt(x\symbol{94}2+y\symbol{94}2),(x,y)):
plot3d(f(x,y),x=0..1,y=0..1,axes=box);}{}
\end{mapleinput}
\mapleresult
\mapleplot{Intplot3d2.eps}
\end{maplegroup}
\begin{maplegroup}
\begin{mapleinput}
\mapleinline{active}{1d}{int(int(f(x,y),y=0..x),x=0..1);
}{}
\end{mapleinput}
\mapleresult
\begin{maplelatex}
\mapleinline{inert}{2d}{(1/6)*sqrt(2)+(1/6)*ln(1+sqrt(2))}{\[\displaystyle 1/6\, \sqrt{2}+1/6\,\ln  \left( 1+ \sqrt{2} \right) \]}
\end{maplelatex}
\end{maplegroup}
\begin{maplegroup}
\begin{mapleinput}
\mapleinline{active}{1d}{}{}
\end{mapleinput}
\end{maplegroup}
\end{document}
 
\subsection{studentパッケージいろいろ}
ちょっとぐらい難しい積分も,Mapleは単純にintコマンドだけで実行してくれる.しかし,時には,途中の計算法である部分積分,置換積分,部分分数展開が必要になる.このような計算はstudentパッケージに用意さている.
\begin{MapleInput}
> with(student):
\end{MapleInput}
\paragraph{部分積分(integration by parts)} 
\begin{MapleInput}
> intparts(Int(x*exp(x),x),x);
\end{MapleInput}
\begin{MapleOutput}
x\exp(x)-\int \exp(x){dx}
\end{MapleOutput}
\paragraph{置換(change of variables)による積分} 
\begin{MapleInput}
> Int((cos(x)+1)^3*sin(x), x);
> changevar(cos(x)+1=u, Int((cos(x)+1)^3*sin(x), x=a..b), u);
> changevar(cos(x)+1=u, int((cos(x)+1)^3*sin(x), x), u);
\end{MapleInput}
\begin{MapleOutputGather}
\int \left( \cos \left( x \right) +1 \right) ^{3}\sin \left( x \right) {dx} \notag \\
\int _{\cos \left( a \right) +1}^{\cos \left( b \right) +1}-{u}^{3}{du} \notag \\
 -\frac{1}{4}\,{u}^{4} \notag \\
\end{MapleOutputGather}
\paragraph{部分分数(partial fraction)展開による積分}
部分分数(partial fraction)展開による積分では,convertコマンドを用いる.
\begin{MapleInput}
> pf1:=convert(1/(1+x^3),parfrac,x);
  int(pf1,x);
\end{MapleInput}
\begin{MapleOutputGather}
 {\it pf1}\, := \frac{1}{3}\,{\frac {-x+2}{{x}^{2}-x+1}}+ \frac{1}{3\,\left( x+1 \right) } \notag \\
 -\frac{1}{6}\,\ln  \left( {x}^{2}-x+1 \right) +\frac{1}{3}\, \sqrt{3}\arctan \left( 1/3\, \left( 2\,x-1 \right)  \sqrt{3} \right) +\frac{1}{3}\,\ln  \left( x+1 \right) \notag
\end{MapleOutputGather}

 
\ChartElementTwo{課題}
\begin{enumerate}
\item 不定積分:次の不定積分を求めよ.

i) $\int 4\,x+3{dx}$
,ii)$\displaystyle \int  \frac{1}{ 1+\mbox{e}^{x} }{dx}$
,iii)$\displaystyle \int  \frac{1}{ \mbox{e}^{-x}+\mbox{e}^{x} }{dx}$
,iv)$\displaystyle \int  \sqrt{1-{x}^{2}}{dx}$


\item 定積分:次の定積分を求めよ.

i)$\displaystyle \int _{0}^{\pi } \sin x{dx}$
,ii)$\displaystyle \int _{0}^{1} \arctan x{dx}$
,iii)$\displaystyle \int _{-2}^{2} \frac{1}{ \sqrt{4-{x}^{2}} }{dx}$
,iv)$\displaystyle \int _{0}^{1} \frac{1}{ {x}^{2}+x+1 }{dx}$

\item (発展課題,重積分)次の2重積分を求めよ.
\begin{equation*}
\int \int_{D} \sqrt{x^2+y^2}dxdy\,\, D:0\leq y \leq x \leq 1
\end{equation*}
\end{enumerate} 
\ChartElementTwo{解答例}
\begin{enumerate}

\item 
\begin{MapleInput}
> int(4*x+3,x);
> int( 1/(1+exp(x)),x);
> int(1/(exp(-x)+exp(x)),x);
> int(sqrt(1-x^2),x);
\end{MapleInput}
\begin{MapleOutputGather}
\displaystyle 2\,{x}^{2}+3\,x  \notag \\
\displaystyle -\ln  \left( 1+ \mbox{e}^x \right) +\ln  \left( \mbox{e}^x \right)  \notag \\
\displaystyle \arctan \left( \mbox{e}^x \right)  \notag \\
\displaystyle \frac{1}{2}\,x \sqrt{1-{x}^{2}}+\frac{1}{2}\,\arcsin \left( x \right)  \notag
\end{MapleOutputGather}

\item
\begin{MapleInput}
> int(sin(x),x=0..Pi);
> int(arctan(x),x=0..1);
> int(1/(sqrt(4-x^(2))),x=-2..2);
> int(1/(x^2+x+1),x=0..1);
\end{MapleInput}
\begin{MapleOutputGather}
\displaystyle 2   \notag \\
\displaystyle \frac{1}{4}\,\pi -\frac{1}{2}\,\ln  \left( 2 \right)   \notag \\
\displaystyle \pi   \notag \\
\displaystyle \frac{1}{9}\,\pi \, \sqrt{3}  \notag
\end{MapleOutputGather}

\item
\begin{MapleInput}
> with(plots):
> inequal({x-y>=0,x>=0,x<=1,y>=0},x=-0.5..1.5,y=-0.5..1.5,optionsexcluded=(color=white));
\end{MapleInput}

\MaplePlot{40mm}{./figures/Intplot2d1.eps}

\begin{MapleInput}
> f:=unapply(sqrt(x^2+y^2),(x,y)):
> plot3d(f(x,y),x=0..1,y=0..1,axes=box);
\end{MapleInput}

\MaplePlot{40mm}{./figures/Intplot3d2.eps}
\begin{MapleInput}
> int(int(f(x,y),y=0..x),x=0..1);
\end{MapleInput}
\begin{MapleOutput}
\displaystyle \frac{1}{6} \sqrt{2}+\frac{1}{6}\,\ln  \left( 1+ \sqrt{2} \right) 
\end{MapleOutput}

\end{enumerate} 

\end{document}

\section{数値積分のコード}
次の積分を例に,Mapleのコードを示す.
\begin{equation*}
\int_0^1 \frac{4}{1+x^2} \, dx
\end{equation*}
先ずは問題が与えられたらできるだけMapleで解いてしまう.答えをあらかじめ知っておくと間違いを見つけるのが容易.プロットしてみる.
\begin{MapleInput}
> restart; 
  f1:=x->4/(1+x^2); 
  plot(f1(x),x=0..5);
\end{MapleInput}
\begin{MapleOutput}
{\it f1}\, := \,x\mapsto \frac{4}{1+{x}^{2}}
\end{MapleOutput}
\MaplePlot{50mm}{./figures/C7_InterpolationIntegralplot2d3.eps}
Mapleで解いてみる.
\begin{MapleInput}
>int(f1(x),x=0..1);
\end{MapleInput}
\begin{MapleOutput}
\pi
\end{MapleOutput}
えっと思うかも知れないが,
\begin{MapleInput}
>int(1/(1+x^2),x);
\end{MapleInput}
\begin{MapleOutput}
arctan(x)
\end{MapleOutput}
となるので,納得できるでしょう.

具体的にMapleでコードを示す.先ずは初期設定.
\begin{MapleInput}
>N:=8: x0:=0: xn:=1: Digits:=20:
\end{MapleInput}

\paragraph{Midpoint rule(中点法)} 
\begin{MapleInput}
> h:=(xn-x0)/N: S:=0: 
  for i from 0 to N-1 do 
    xi:=x0+(i+1/2)*h; 
    dS:=h*f1(xi);
    S:=S+dS; 
  end do: 
  evalf(S);
\end{MapleInput}
\begin{MapleOutput}
3.1428947295916887799
\end{MapleOutput}

\paragraph{Trapezoidal rule(台形公式)} 
\begin{MapleInput}
> h:=(xn-x0)/N: S:=f1(x0)/2: 
  for i from 1 to N-1 do 
    xi:=x0+i*h; 
    dS:=f1(xi);
    S:=S+dS; 
  end do: 
  S:=S+f1(xn)/2: 
  evalf(h*S);
\end{MapleInput}
\begin{MapleOutput}
3.1389884944910890093
\end{MapleOutput}

\paragraph{Simpson's rule(シンプソンの公式)} 
\begin{MapleInput}
> M:=N/2: h:=(xn-x0)/(2*M): Seven:=0: Sodd:=0: 
  for i from 1 to 2*M-1 by 2 do
    xi:=x0+i*h; 
    Sodd:=Sodd+f1(xi); 
  end do: 
  for i from 2 to 2*M-1 by 2 do
    xi:=x0+i*h; 
    Seven:=Seven+f1(xi); 
  end do:
  evalf(h*(f1(x0)+4*Sodd+2*Seven+f1(xn))/3);
\end{MapleInput}
\begin{MapleOutput}
3.1415925024587069144
\end{MapleOutput}


\section{課題}
\begin{enumerate}
\item 4x4の行列を適当に作り,Mapleで固有値を求めよ.求め方はマニュアルを参照せよ.
\item Jacobi法によって固有値を求めよ.
\item LAPACKに含まれているdsyev関数を用いて実対称行列の固有値を求めよ.(演習で詳しく取り上げている.研究や今後のために必要と思うときは,テキストを取りにおいで)
\end{enumerate}

\end{document}
