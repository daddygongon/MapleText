\documentclass[10pt,a4j]{jreport}
\usepackage[dvips]{graphicx,color}
\usepackage{verbatim}
\usepackage{amsmath,amsthm,amssymb}
\topmargin -15mm\oddsidemargin -4mm\evensidemargin\oddsidemargin
\textwidth 170mm\textheight 257mm\columnsep 7mm
\setlength{\fboxrule}{0.2ex}
\setlength{\fboxsep}{0.6ex}

\pagestyle{empty}

\newcommand{\MaplePlot}[2]{{\begin{center}
    \includegraphics[width=#1,clip]{#2}
                     \end{center}
%
} }

\newenvironment{MapleInput}{%
    \color{red}\verbatim
}{%
    \endverbatim
}

\newenvironment{MapleError}{%
    \color{blue}\verbatim
}{%
    \endverbatim
}

\newenvironment{MapleOutput}{%
    \color{blue}\begin{equation*}
}{%
    \end{equation*}
}

\newenvironment{MapleOutputGather}{%
    \color{blue}\gather
}{%
    \endgather
}

\newif\ifHIKI
%\HIKItrue % TRUEの設定
\HIKIfalse % FALSEの設定

\begin{document}
\chapter{線形代数--固有値(LAEigen)}
\section{固有値}
では,ここでクイズです.固有ベクトルは上のグラフの何処に対応するか? ヒントは,
\begin{quotation}
行列Aの固有値,固有ベクトルを$\lambda, x_0$とすると,
\begin{equation*}
A \,x_0 = \lambda \, x_0
\end{equation*}
が成立する
\end{quotation}
です.固有値と固有ベクトルはMapleでは以下のコマンドで求まります.
\begin{MapleInput}
> lambda,P:=Eigenvectors(A);
\end{MapleInput}
\begin{MapleOutput}
\lambda,\,P\, := \, \left[ \begin {array}{c} -3\\6\end {array} \right] ,\, \left[ \begin {array}{cc} -1&5/4\\1&1\end {array} \right]
\end{MapleOutput}
ここではMapleコマンドのEigenvectorsで戻り値を$\lambda$(lambdaと書きます),$P$に代入しています.この後ろ側にある行列$P$の1列目で構成されるベクトルが固有値-3に対応する固有ベクトル,2列目のベクトルが固有値6に対応する固有ベクトルです.

\subsection{解答}
固有値$\lambda$,固有ベクトル$x_0$の関係式
\begin{equation*}
A \,x_0 = \lambda \, x_0
\end{equation*}
を言葉で言い直すと,
\begin{quote}
固有ベクトル$x_0$は変換行列$A$によって,自分の固有値倍のベクトル$\lambda x_0$に写されるベクトル
\end{quote}
となります.つまり変換の図で言うと,
\begin{quote}
変換しても方向が変わらない赤点(の方向)
\end{quote}
となります.これは図で書くと
\begin{MapleInput}
> vv1:=Column(P,1): vv2:=Column(P,2): 
  a1:=vv1[2]/vv1[1]: a2:=vv2[2]/vv2[1]:
  pp1:=plot({a2*x,a1*x},x=-d..d):
\end{MapleInput}
Columnによって行列の第i列目をとりだし,その比によって直線の傾きを求めています.そうして引いた2本の直線をpp1としてため込んで,先ほど描いた変換の図に加えて表示(display)させます.
\begin{MapleInput}
> display(point2,line2,pp1,view=[-d..d,-d..d]);
\end{MapleInput}
\MaplePlot{80mm}{./figures/LAFundamentalsplot2d3.eps}
pp1を入れて描いた直線が引かれた方向ではたしかに変換によっても方向が変わらなさそうに見えるでしょう.

おまけですが,行列の対角化は次のようにしてできます.
\begin{MapleInput}
> MatrixInverse(P).A.P;
\end{MapleInput}
\begin{MapleOutput}
\left[ \begin {array}{cc} -3&0\\ 0&6\end {array} \right]
\end{MapleOutput}
\section{固有値の幾何学的意味}
次に,固有値の幾何学的な意味を2次元行列で確認しておこう.ある点$x_0$に対称正方行列$A$を作用すると,$x_1$に移動する.これを原点を中心とする円上の点に次々に作用させ,移動前後の点を結ぶ.

\begin{MapleInput}
> restart;
  with(LinearAlgebra):with(plots):with(plottools):
  A:=Matrix(1..2,1..2,[[3,2/3],[2/3,2]]):
\end{MapleInput}


\begin{MapleInput}
> N:=30:p1:=[]:l1:=[]:
for k from 0 to N-1 do
  x0:=Vector([sin(2*Pi*k/N),cos(2*Pi*k/N)]);
  x1:=MatrixVectorMultiply(A,x0);
  p1:=[op(p1),pointplot({x0,x1})];
  l1:=[op(l1),line( evalf(convert(x0,list)),evalf(convert(x1,list)) )];
end do:

> n:=4;
  display(p1,l1,view=[-n..n,-n..n]);
\end{MapleInput}
\MaplePlot{70mm}{./figures/EigenVectors3plot1.eps}

真ん中の円状の領域が,外側の楕円状の領域に写像されている様子が示されている.この図の中に固有ベクトル,固有値が隠れている.どこかわかる?


\section{Googleのページランク}
\begin{quote}
多くの良質なページからリンクされているページはやはり良質なページである
\end{quote}
Googleのpage rankは上のような非常に単純な仮定から成り立っている.ページランクを実際に求めよう.つぎのようなリンクが張られたページを考える.
\MaplePlot{70mm}{./figures/linkstruct.eps}

計算手順は以下の通り\footnote{詳しくは\texttt{http://www.kusastro.kyoto-u.ac.jp/\~baba/wais/pagerank.html}を参照せよ.}.
\begin{enumerate}
\item リンクを再現する隣接行列を作る.ページに番号をつけて,その間が結ばれているi-j要素を1,そうでない要素を0とする.
\item 隣接行列を転置する
\item 列ベクトルの総和が1となるように規格化する.
\item こうして得られた推移確率行列の最大固有値に属する固有ベクトルを求め,適当に規格化する.
\end{enumerate}

\subsection{課題}
\begin{enumerate}
\item 上記手順を参考にして,Mapleでページランクを求めよ.
\item このような問題ではすべての固有値・固有ベクトルを求める必要はなく,最大の固有値を示す固有ベクトルを求めるだけでよい.初期ベクトルを適当に決めて,何度も推移確率行列を掛ける反復法でページランクを求めよ.
\end{enumerate}
\begin{itemize}
\item[隣接行列]
\begin{equation*}
{\it A1}\, := \, \left[ \begin {array}{c|c|c|c|c|c|c|c} 
 &1&2&3&4&5&6&7\\
1&0&1&1&1&1&0&1\\
2&1&0&0&0&0&0&0\\
3& & & & & & & \\
4& & & & & & & \\
5& & & & & & & \\
6& & & & & & & \\
7& & & & & & & 
\end {array} \right] 
\end{equation*}
\item[転置行列]
\begin{equation*}
{Transpose}({\it A1})\, := \, \left[ \begin {array}{c|c|c|c|c|c|c} 
\, \, &\, \, &\, \, &\, \, &\, \, &\, \, &\, \, \\
 & & & & & & \\
 & & & & & & \\
 & & & & & & \\
 & & & & & & \\
 & & & & & & \\
 & & & & & & 
\end {array} \right] 
\end{equation*}
\item[規格化]
\begin{equation*}
\left[ \begin {array}{c|c|c|c|c|c|c} 
\, \, &\, \, &\, \, &\, \, &\, \, &\, \, &\, \, \\
 & & & & & & \\
 & & & & & & \\
 & & & & & & \\
 & & & & & & \\
 & & & & & & \\
 & & & & & & 
\end {array} \right] 
\end{equation*}
\item[遷移]
\begin{equation*}
\left( \begin {array}{ccccccc} 
0 &1 &1/2 &0 &1/4 &1/2 &0 \\
1/5 &0 &1/2 &1/3 &0 &0 &0 \\
1/5 &0 &0 &1/3 &1/4 &0 &0 \\
1/5 &0 &0 &0 &1/4 &0 &0 \\
1/5 &0 &0 &1/3 &0 &1/2 &1 \\
0 &0 &0 &0 &1/4 &0 &0 \\
1/5 &0 &0 &0 &0 &0 &0 
\end {array} \right) 
\left( \begin {array}{c} 
1/7\\ 
1/7\\ 
1/7\\ 
1/7\\ 
1/7\\ 
1/7\\ 
1/7 
\end {array} \right) \, = \, 
\left( \begin {array}{ccccccc} 
\, \, &\, \, &\, \, &\, \, &\, \, &\, \, &\, \, \\
 & & & & & & \\
 & & & & & & \\
 & & & & & & \\
 & & & & & & \\
 & & & & & & \\
 & & & & & & 
\end {array} \right)
\, = \, \left( \begin {array}{c} 
0.32\\ 
0.15\\ 
0.11\\ 
0.06\\ 
0.29\\ 
0.04\\ 
0.03 
\end {array} \right) 
\end{equation*}
\end{itemize}
\pagebreak

\section{累乗(べき乗)法により最大固有値が求まる原理}
累乗(べき乗)法は,最大固有値とその固有ベクトルを効率的に見つける算法である.すこし,固有値について復習しておく.正方行列$A$に対して,
\begin{equation}
A x = \lambda x
\end{equation}
の解$\lambda$を固有値,$x$を固有ベクトルという.$\lambda$は,
\begin{equation}
\det( A - \lambda E) =0
\end{equation}
として求まる永年方程式の解である.

では,なぜ適当な初期ベクトル$x_0$から始めて,反復
\begin{equation}
x_{k+1} = A x_k
\end{equation}
を繰り返すと,$A$の絶対値最大の固有値に属する固有ベクトルに近づいていくのかを見ておこう.

すべての固有値がお互いに異なる場合を考える.今,行列の固有値を絶対値の大きなもの順に並べて,$|\lambda_1|>|\lambda_2|>\cdots>|\lambda_n|$とし,対応する長さを1に規格化した固有ベクトルを$x_1, x_2, \ldots, x_n$とする.初期ベクトルは固有ベクトルの線形結合で表わせて,
\begin{equation}
X_0 = c_1x_1+c_2x_2+\cdots+c_nx_n
\end{equation}
となるとする.これに行列$A$を$N$回掛けると,
\begin{equation}
A^N X_0 = c_1 \lambda_1^N x_1+
c_2  \lambda_2^N x_2+\cdots+
c_n  \lambda_n^N x_n
\end{equation}
となる.これを変形すると,
\begin{equation}
A^NX_0 = X_{N}
= c_1 \lambda_1^N \left\{ x_1+
\frac{c_2}{c_1}\left(\frac{\lambda_2}{\lambda_1}\right)^N  x_2+\cdots+
\frac{c_n}{c_1}\left(\frac{\lambda_n}{\lambda_1}\right)^N  x_n \right\}
\end{equation}
となる.$|\lambda_1|>|\lambda_i|(i\ge2)$だから括弧の中は$x_1$だけが生き残る.

こうして最大固有値に属する固有ベクトルが,反復計算を繰り返すだけで求められる.


\section{Jacobi回転による固有値の求め方}
固有値を求める手法として,永年方程式を解くというやり方は回りくどすぎる.少し古めかしいが非対角要素を0にする回転行列を反復的に作用させるJacobi(ヤコビ)法を紹介する.現在認められている最適の方策は,ハウスホルダー(Householder)変換で行列を単純な三重対角化行列に変形してから,反復法で解を追い込んでいくやり方である.Jacobi法は,Householder法ほど万能ではないが,10次程度までの行列には今でも役に立つ.

\subsection{Mapleでみる回転行列}
行列の軸回転の復習をする.対称行列$B$に回転行列$U$を作用すると
\begin{equation}
B.U =  
\left(
\begin{array}{cc}
{a_{1\,1}} & {a_{1\,2}}\\
{a_{2\,1}(={a_{1\,2}})} & {a_{2\,2}}
\end{array}
 \right)
\left( 
\begin{array}{cc}
\cos(\theta) &  -\sin(\theta)\\
\sin(\theta) & \cos(\theta)
\end{array}
 \right) 
\end{equation}

\ifHIKI
\begin{tabular}{|c|}
\hline
         \\ 
\hline
\end{tabular}
\else
\begin{equation*}
\setlength{\unitlength}{1cm}
\begin{picture}(10,3.5)
\put(0,0){\framebox(10,3.5){}}
\end{picture}
\end{equation*}
\fi
となる.回転行列を4x4の行列に
\begin{equation}
U^t B U
\end{equation}
と作用させたときの各要素の様子を以下に示した.
\begin{MapleInput}
> restart:
> n:=4:
> with(LinearAlgebra):
> B:=Matrix(n,n,shape=symmetric,symbol=a);
\end{MapleInput}

\begin{MapleOutput}
B :=  \left[{
	\begin{array}{cccc}
	{a_{1, \,1}} & {a_{1, \,2}} & {a_{1, \,3}} & {a_{1, \,4}} \\
	{a_{1, \,2}} & {a_{2, \,2}} & {a_{2, \,3}} & {a_{2, \,4}} \\
	{a_{1, \,3}} & {a_{2, \,3}} & {a_{3, \,3}} & {a_{3, \,4}} \\
	{a_{1, \,4}} & {a_{2, \,4}} & {a_{3, \,4}} & {a_{4, \,4}}
	\end{array}}
 \right] 
\end{MapleOutput}

\begin{MapleInput}
> U:=Matrix(n,n,[[c,-s,0,0],[s,c,0,0],[0,0,1,0],[0,0,0,1]]);
#U:=Matrix(n,n,[[c,-s],[s,c]]);
\end{MapleInput}
\begin{MapleOutput}
U :=  \left[ 
{\begin{array}{ccrr}
c &  - s & 0 & 0 \\
s & c & 0 & 0 \\
0 & 0 & 1 & 0 \\
0 & 0 & 0 & 1
\end{array}}
 \right] 
\end{MapleOutput}

\begin{MapleInput}
>TT:=Transpose(U).B.U;
\end{MapleInput}

\begin{MapleOutputGather}
\mathit{TT} :=  \\ \notag
{\begin{array}{c}
 \left[ \right.  
  (c\,{a_{1, \,1}} + s\,{a_{1, \,2}})\,c + (c\,{a_{1, \,2}} + s
\,{a_{2, \,2}})\,s\,, \, - (c\,{a_{1, \,1}} + s\,{a_{1, \,2}})\,s
 + (c\,{a_{1, \,2}} + s\,{a_{2, \,2}})\,c\,,  \\
c\,{a_{1, \,3}} + s\,{a_{2, \,3}}\,, \,c\,{a_{1, \,4}} + s\,{a_{2
, \,4}}   \left. \right]  \\
 \left[ \right.  
  ( - s\,{a_{1, \,1}} + c\,{a_{1, \,2}})\,c + ( - s\,{a_{1, \,2
}} + c\,{a_{2, \,2}})\,s\,, \, - ( - s\,{a_{1, \,1}} + c\,{a_{1, 
\,2}})\,s + ( - s\,{a_{1, \,2}} + c\,{a_{2, \,2}})\,c\,,  \\
 - s\,{a_{1, \,3}} + c\,{a_{2, \,3}}\,, \, - s\,{a_{1, \,4}} + c
\,{a_{2, \,4}}   \left. \right] \\
 \left[   c\,{a_{1, \,3}} + s\,{a_{2, \,3}}\,, \, - s\,{a_{1, 
\,3}} + c\,{a_{2, \,3}}\,, \,{a_{3, \,3}}\,, \,{a_{3, \,4}}  
 \right]  \\
 \left[   c\,{a_{1, \,4}} + s\,{a_{2, \,4}}\,, \, - s\,{a_{1, 
\,4}} + c\,{a_{2, \,4}}\,, \,{a_{3, \,4}}\,, \,{a_{4, \,4}}  
 \right] 
 \end{array}}
\end{MapleOutputGather}

\begin{MapleInput}
>expand(TT[1,1]);
expand(TT[2,2]);
expand(TT[1,2]);
expand(TT[2,1]);
\end{MapleInput}
\begin{MapleOutput}
c^{2}\,{a_{1, \,1}} + 2\,c\,s\,{a_{1, \,2}} + s^{2}\,{a_{2, \,2}}
\end{MapleOutput}
\begin{MapleOutput}
s^{2}\,{a_{1, \,1}} - 2\,c\,s\,{a_{1, \,2}} + c^{2}\,{a_{2, \,2}}
\end{MapleOutput}
\begin{MapleOutput}
 - s\,c\,{a_{1, \,1}} - s^{2}\,{a_{1, \,2}} + c^{2}\,{a_{1, \,2}}
 + c\,s\,{a_{2, \,2}}
\end{MapleOutput}
\begin{MapleOutput}
 - s\,c\,{a_{1, \,1}} - s^{2}\,{a_{1, \,2}} + c^{2}\,{a_{1, \,2}}
 + c\,s\,{a_{2, \,2}}
\end{MapleOutput}
この非対角要素を0にする$\theta$は以下のように求まる.
\ifHIKI
\begin{tabular}{|c|}
\hline
         \\ 
\hline
\end{tabular}
\else
\begin{equation*}
\setlength{\unitlength}{1cm}
\begin{picture}(10,3.5)
\put(0,0){\framebox(10,3.5){}}
\end{picture}
\end{equation*}
\fi
このとき注目している$i,j=1,2$以外の要素も変化する.
\begin{MapleInput}
>expand(TT[3,1]);
expand(TT[3,2]);
\end{MapleInput}
\begin{MapleOutput}
c\,{a_{1, \,3}} + s\,{a_{2, \,3}}
\end{MapleOutput}
\begin{MapleOutput}
 - s\,{a_{1, \,3}} + c\,{a_{2, \,3}}
\end{MapleOutput}
これによって一旦0になった要素も値を持つが,なんども繰り返すことによって,徐々に0へ近づいていく.
\section{Jacobi法による固有値を求めるCコード}
以下にはヤコビ法を用いた固有値と固有ベクトルを求めるコードを示した.結果は,固有値とそれに対応する規格化された固有ベクトルが縦(column)ベクトルで表示される.

リスト: ヤコビ法.
\begin{MapleInput}
#include <stdio.h>
#include <math.h>

#define M 10
void PrintMatrix(double a[M][M], int n);

int main(void){
  double a[M][M],v[M][M];
  double eps=0.0001,div,r,t,s,c,apj,aqj,aip,aiq,vip,viq;
  int i,j,n,iter,count,iterMax=1000000,p,q;
  
  scanf("%d",&n);
  for(i=1;i<=n;i++){
    for(j=1;j<=n;j++) scanf("%lf",&a[i][j]);
  }
  PrintMatrix(a,n);
  for(i=1;i<=n;i++){
    for(j=1;j<=n;j++) v[i][j]=0.;    
    v[i][i]=1.;
  }
  
  for(iter=1;iter<=iterMax;iter++){
    count=0;
    for(p=1;p<=n-1;p++){
      for(q=p+1;q<=n;q++){
	if(fabs(a[p][q])<eps) continue;
	count++;
	div=a[p][p]-a[q][q];
	if (div != 0.0){
	  r=2.0*a[p][q]/div;
	  t=0.5*atan(r);
	} else {
	  t=0.78539818;
	}
	s=sin(t);
	c=cos(t);
	for(j=1;j<=n;j++){
	  apj=a[p][j];
	  aqj=a[q][j];
	  a[p][j]=apj*c+aqj*s;
	  a[q][j]=-apj*s+aqj*c;
	}
	for(i=1;i<=n;i++){
	  aip=a[i][p];
	  aiq=a[i][q];
	  a[i][p]=aip*c+aiq*s;
	  a[i][q]=-aip*s+aiq*c;
	  vip=v[i][p];
	  viq=v[i][q];
	  v[i][p]=vip*c+viq*s;
	  v[i][q]=-vip*s+viq*c;
	}
	printf("p,q=%3d,%3d\n",p,q);
	PrintMatrix(a,n);
      }
    }
    if (count==0) break;
  }
  printf("Eigen values:\n");
  for(i=1;i<=n;i++) printf("%6.2f",a[i][i]);
  printf("\nEigen vectors:\n");
  PrintMatrix(v,n);
  
  return 0;
}

void PrintMatrix(double a[M][M], int n){
  int i,j;
  for(i=1;i<=n;i++){
    for(j=1;j<=n;j++) printf("%6.2f",a[i][j]);
    printf("\n");
  }
  printf("\n");
}
\end{MapleInput}


リスト: ヤコビ法の計算結果.

\begin{MapleError}
[BobsNewPBG4:~/NumRecipe/chap8] bob% cat input.txt
4
5 4 1 1
4 5 1 1
1 1 4 2
1 1 2 4

BobsNewPBG4:~/NumRecipe/chap8] bob% Jacobi2<input.txt
  5.00  4.00  1.00  1.00
  4.00  5.00  1.00  1.00
  1.00  1.00  4.00  2.00
  1.00  1.00  2.00  4.00

p,q=  1,  2
  9.00 -0.00  1.41  1.41
 -0.00  1.00 -0.00 -0.00
  1.41 -0.00  4.00  2.00
  1.41 -0.00  2.00  4.00

p,q=  1,  3
  9.37 -0.00 -0.00  1.88
 -0.00  1.00  0.00 -0.00
 -0.00  0.00  3.63  1.57
  1.88 -0.00  1.57  4.00

p,q=  1,  4
  9.96 -0.00  0.47 -0.00
 -0.00  1.00  0.00  0.00
  0.47  0.00  3.63  1.50
  0.00  0.00  1.50  3.41

...<中略>...

Eigen values:
 10.00  1.00  5.00  2.00
Eigen vectors:
  0.63 -0.71 -0.32  0.00
  0.63  0.71 -0.32  0.00
  0.32  0.00  0.63 -0.71
  0.32  0.00  0.63  0.71
\end{MapleError}


\section{数値計算ライブラリーについて}
一般の数値計算ライブラリーについては,時間の関係で講義ではその能力を紹介するにとどめる.昔の演習で詳しく取り上げていたので,研究や今後のために必要と思うときは,テキストを取りにおいで.

行列の計算は,数値計算の中でも特に利用する機会が多く,また,律速ルーチンとなる可能性が高い.そこで,古くから行列計算の高速ルーチンが開発されてきた.なかでもBLASとLAPACKはフリーながら非常に高速である. 

前回に示した,逆行列を求める単純なLU分解法をC言語でコーディングしたものと,LAPACKのルーチンを比べた場合,1000次元の行列で計測すると
\begin{MapleInput}
>  1000 [dim]     2.5200 [sec] #BOB
>  1000 [dim]     0.4700 [sec] #LAPACK
\end{MapleInput}
となった.用いたPCはMacBook(2GHz Interl Core Duo)であるが,この計算での0.47秒は1.4GFLOPに相当する.07年のMacBook(2GHz Interl Core 2 Duo)ではさらに早くなって
\begin{MapleInput}
bob% gcc -O3 bob.c -o bob
bob% ./bob
1000
 1000 [dim]     1.7543 [sec] #BOB
bob% gcc -O3 lapack.c -llapack -lblas -o lapack
bob% ./lapack
1000
 1000 [dim]     0.1893 [sec] #LAPACK
\end{MapleInput}
で,3.5GFLOPSが出ている.

ライブラリーは世界中の計算機屋さんがよってたかって検証しているので,バグがほとんど無く,また,高速である.初学者はライブラリーを使うべきである.ただし,下のサンプルプログラムの行列生成の違いのように,ブラックボックス化すると思わぬ間違い(ここではFortranとCでの行列の並び順の違いが原因)をしでかすことがあるので,プログラムに組み込む前に必ず小さい次元(サンプルコード)で検証しておくこと.

添付のコードはちょっと長いが時間があればフォローせよ.コンパイルは,OSXでは
\begin{MapleInput}
> gcc -O3 -UPRINT lapack.c -llapack -lblas
\end{MapleInput}
とすればできる.linuxではLAPACK, BLASがインストールされていれば,
\begin{MapleInput}
> #include <vecLib/vecLib.h>
\end{MapleInput}
をコメントアウトして,
\begin{MapleInput}
> gcc -O3 -DPRINT lapack.c -L/usr/local/lib64 -llapack -lblas -lg2c
\end{MapleInput}
などとすればコンパイルできるはず.

\paragraph{リスト: 西谷製lazy逆行列計算プログラム} 
\begin{MapleInput}
#include <stdio.h>
#include <stdlib.h>
#include <math.h>
#include <time.h>

//#undef PRINT
//#define PRINT

void printMatrix(double *a, double *b, long n);
int MatrixInverse(double *a, double *b, long n);

int main(void){
  clock_t start, end;
  int i,j;
  long n;
  double *a,*b;

  scanf("%ld",&n);

  a=(double *)malloc(n*n*sizeof(double));
  b=(double *)malloc(n*sizeof(double));

  for (i=0;i<n;i++){
    for (j=0;j<n;j++){
      a[i*n+j]= 2*(double) random() / RAND_MAX - 1.0;
    }
  }
  for (i=0;i<n;i++){
    b[i]= 2*(double) random() / RAND_MAX - 1.0;
  }
  printMatrix(a,b,n);

  start = clock();
  MatrixInverse(a,b,n);
  end = clock();
  printf("%5d [dim] %10.4f [sec] #BOB\n", 
	 n,(double)(end-start)/CLOCKS_PER_SEC);
  printMatrix(a,b,n);

  free(a);
  free(b);
  return 0;
}

int MatrixInverse(double *a, double *b, long n){
  double *x;
  double pvt=0.00005,am;
  int i,j,k;

  x=(double *)malloc(n*sizeof(double));

  for(i=0;i<n-1;i++){
    if(fabs(a[i*n+i])<pvt){
      printf("Pivot %3d=%10.5f is too small.\n",i,a[i*n+i]);
      return 1;
    }
    for(j=i+1;j<n;j++){
      am=a[j*n+i]/a[i*n+i];
      for(k=0;k<n;k++) a[j*n+k]-=am*a[i*n+k];
      b[j]-=am*b[i];
    }
  }
  //Backward substitution
  for(j=n-1;j>=0;j--){
    x[j]=b[j];
    for(k=j+1;k<n;k++){
      x[j]-=a[j*n+k]*x[k];
    }
    b[j]=x[j]/=a[j*n+j];
  }
  free(x);
  return 0;
}

void printMatrix(double *a, double *b, long n){
  int i,j;
#ifdef PRINT
  printf("\n");
  for (i=0;i<n;i++){
    for (j=0;j<n;j++){
      printf("%10.5f",a[i*n+j]);
    }
    printf(":%10.5f",b[i]);
    printf("\n");
  }
  printf("\n");
#endif
  return;
}
\end{MapleInput}


\paragraph{リスト : LAPACK謹製smart逆行列計算プログラム} 
\begin{MapleInput}
#include <stdio.h>
#include <stdlib.h>
#include <math.h>
#include <time.h>

//#define PRINT
//#undef PRINT

void printMatrix(double *a, double *b, long n);

int main(void){
  clock_t start, end;
  int i,j;
  double *a,*b;
  long n,nrhs=1, lda,ldb, info, *ipiv;

  scanf("%ld",&n);

  a=(double *)malloc(n*n*sizeof(double));
  b=(double *)malloc(n*sizeof(double));
  lda=ldb=n;
  ipiv=(long *)malloc(n*sizeof(long));
  
  for (i=0;i<n;i++){
    for (j=0;j<n;j++){
      a[j*n+i]= 2*(double) random() / RAND_MAX - 1.0;
    }
  }

  for (i=0;i<n;i++){
    b[i]= 2*(double) random() / RAND_MAX - 1.0;
  }
  printMatrix(a,b,n);

  start = clock();
  dgesv_(&n, &nrhs, a, &lda, ipiv, b, &ldb, &info);
  end = clock();
  printf("%5d [dim] %10.4f [sec] #LAPACK\n", 
	 n, (double)(end-start)/CLOCKS_PER_SEC);
  printMatrix(a,b,n);

  free(a);
  free(b);
  free(ipiv);

  return 0;
}

void printMatrix(double *a, double *b, long n){
  int i,j;
#ifdef PRINT
  printf("\n");
  for (i=0;i<n;i++){
    for (j=0;j<n;j++){
      printf("%10.5f",a[i*n+j]);
    }
    printf(":%10.5f",b[i]);
    printf("\n");
  }
  printf("\n");
#endif
  return;
}
\end{MapleInput}




\section{課題}
\begin{enumerate}
\item 4x4の行列を適当に作り,Mapleで固有値を求めよ.求め方はマニュアルを参照せよ.
\item Jacobi法によって固有値を求めよ.
\item LAPACKに含まれているdsyev関数を用いて実対称行列の固有値を求めよ.(演習で詳しく取り上げている.研究や今後のために必要と思うときは,テキストを取りにおいで)
\end{enumerate}
\end{document}
