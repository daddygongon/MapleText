\begin{enumerate}
\item 次の項目について答えよ.(2004, 05, 06年度期末試験)
\begin{enumerate}
\item 数値計算の精度を制約するデータ形式とその特徴は何か.
\item 丸め誤差とは何か.
\item 打ち切り誤差とは何か.
\item 安定性とは何か.
\end{enumerate}
\item 10 進数 4 桁の有効桁数をもった計算機になったつもりで,a=0.81543, b=35.724, c=35.596として,以下の計算をおこなえ.
  \begin{enumerate}
  \item b-c
  \item a/(b-c)
  \end{enumerate}
\item 10 進数 3 桁の有効桁数をもった計算機になったつもりで,以下の条件で預金を求める計算をおこなえ.
  \begin{enumerate}
  \item 元本を10000円とする
  \item 利息0.3\%とする
  \item 複利計算で10年でいくらになるか.
  \end{enumerate}
\item 係数を a = 1,  b = 10000000,  c = 1としたときに, 通常の解の公式を使った解と, 解と係数の関係(下記の記述を参照)を使った解とを出力するプログラムをMapleで作成し, 2種類計算方法の違いで解の有効数字がどう違うかを比べよ.

2 次方程式 $ax^2+bx+c=0$の
係数$a, b, c$が特殊な値をもつ場合,通常の解の公式 

\begin{equation*}
x = \frac {-b \pm \sqrt{{b}^{2}-4ac}}{2a}
\end{equation*}
にしたがって計算するとケタ落ちによる間違った答えを出す.その特殊な値とは

\begin{equation*}
\sqrt{{b}^{2}-4ac} \approx |b|
\end{equation*}
となる場合である.

ケタ落ちを防ぐには, $b > 0$の場合は,  

\begin{equation*}
x_1 = \frac {-b - \sqrt{{b}^{2}-4ac}}{2a}
\end{equation*}
として,ケタ落ちを起こさずに求め, この解を使って, 解と係数の関係より

\begin{equation*}
x_2 = \frac {c}{a\,  x_1}
\end{equation*}
で求める.$b < 0$ の場合は,解の公式の足し算の方を使って同様に求める.

\end{enumerate}
