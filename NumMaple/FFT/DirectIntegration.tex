対象とする関数をまず作る.
\begin{MapleInput}
> restart;
> #F:=x->piecewise(x=0,1/2,x>0,x); 
  #F:=x->piecewise(x<1/2,x,x>=1/2,1-x);
  #F:=x->piecewise(x<1/2,-1,x>1/2,1); 
  F:=x->piecewise(x<1/2,-1,x>=1/2,1);
  #F:=x->piecewise(x>0 and x<1/2,-1,x>1/2,1); 
  #F:=x->x-1/2; 
  plot(F(x),x=0..1);
\end{MapleInput}
\begin{MapleOutput}
F := x \mapsto piecewise(x < \frac{1}{2}, -1, \frac{1}{2} \leq x, 1) 
\end{MapleOutput}
\MaplePlot{50mm}{./figures/C10_FFTplot2d11.eps}
piecewise関数は階段関数で,振る舞いはコメント(\#)を適当に外して確認せよ.
初期設定.
\begin{MapleInput}
> KK:=3; N:=2^KK;L:=1-0;
> 2*Pi*1/L*x;
\end{MapleInput}
\begin{MapleOutput}
2 \pi x
\end{MapleOutput}
\begin{MapleInput}
> int(F(x)*cos(2*Pi*1/L*x),x=0..L);
\end{MapleInput}
\begin{MapleOutput}
0
\end{MapleOutput}

\begin{MapleInput}
> for n from 0 to N do
    a[n]:=2/L*int(F(x)*cos(2*Pi*n/L*x),x=0..L); 
  end do;
\end{MapleInput}
\begin{MapleOutputGather}
a_0:=0 \notag \\
a_1:=0 \notag \\
a_2:=0 \notag \\
a_3:=0 \notag \\
a_4:=0 \notag \\
a_5:=0 \notag \\
a_6:=0 \notag \\
a_7:=0 \notag \\
a_8:=0 \notag 
\end{MapleOutputGather}

\begin{MapleInput}
> for n from 0 to N do 
    b[n]:=2/L*int(F(x)*sin(2*Pi*n/L*x),x=0..L); 
  end do;
\end{MapleInput}
\begin{MapleOutputGather}
b_0:=0 \notag \\
b_1:=\frac{4}{\pi} \notag \\
b_2:=0 \notag \\
b_3:=\frac{4}{3\pi} \notag \\
b_4:=0 \notag \\
b_5:=\frac{4}{5\pi} \notag \\
b_6:=0 \notag \\
b_7:=\frac{4}{7\pi} \notag \\
b_8:=0 \notag 
\end{MapleOutputGather}
ここで,オイラーの関係
\begin{MapleOutputGather}
a[n]=c[n]+c[-n],\, b[n]=I (c[n]-c[-n]) \notag \\
c[-n]= \frac{1}{2} (a[n] + b[n]),\, c[n]=\frac{1}{2} (a[n] - I b[n])
\end{MapleOutputGather}
を使って,三角関数系から$\exp$へ変換する.

\begin{MapleInput}
> for n from 0 to N do c[n]:=1/L*int(F(x)*exp(-I*2*Pi*n/L*x),x=0..L); end do;
> for n from 1 to N do c[-n]:=1/L*int(F(x)*exp(I*2*Pi*n/L*x),x=0..L); end do;
\end{MapleInput}
\begin{MapleOutputGather}
c_0:=0 \notag \\
c_1:=\frac{2I}{\pi} \notag \\
c_2:=0 \notag \\
c_3:=\frac{2I}{3\pi} \notag \\
c_4:=0 \notag \\
c_5:=\frac{2I}{5\pi} \notag \\
c_6:=0 \notag \\
c_7:=\frac{2I}{7\pi} \notag \\
c_8:=0 \notag \\
c_{-1}:=-\frac{2I}{\pi} \notag \\
c_{-2}:=0 \notag \\
c_{-3}:=-\frac{2I}{3\pi} \notag \\
c_{-4}:=0 \notag \\
c_{-5}:=-\frac{2I}{5\pi} \notag \\
c_{-6}:=0 \notag \\
c_{-7}:=-\frac{2I}{7\pi} \notag \\
c_{-8}:=0 \notag 
\end{MapleOutputGather}

\begin{MapleInput}
> F1:=unapply(sum(evalf(c[i]*exp(I*2*Pi*i/L*x)),i=-(N-1)..(N-1)),x):
> plot({Re(F1(x)),F(x)},x=0..1);
\end{MapleInput}
\MaplePlot{50mm}{./figures/C10_FFTplot2d12.eps}

\begin{MapleInput}
> evalf(2/3/Pi);
\end{MapleInput}
\begin{MapleOutput}
0.2122065907
\end{MapleOutput}
                                
