%% Created by Maple 2015.2, Mac OS X
%% Source Worksheet: FFTEx.mw
%% Generated: Fri Aug 19 15:39:11 JST 2016
\documentclass{article}
\usepackage{maplestd2e}
\def\emptyline{\vspace{12pt}}
\begin{document}
\pagestyle{empty}
\DefineParaStyle{Maple Bullet Item}
\DefineParaStyle{Maple Heading 1}
\DefineParaStyle{Maple Warning}
\DefineParaStyle{Maple Heading 4}
\DefineParaStyle{Maple Heading 2}
\DefineParaStyle{Maple Heading 3}
\DefineParaStyle{Maple Dash Item}
\DefineParaStyle{Maple Error}
\DefineParaStyle{Maple Title}
\DefineParaStyle{Maple Text Output}
\DefineParaStyle{Maple Normal}
\DefineCharStyle{Maple 2D Output}
\DefineCharStyle{Maple 2D Input}
\DefineCharStyle{Maple Maple Input}
\DefineCharStyle{Maple 2D Math}
\DefineCharStyle{Maple Hyperlink}
\section{\textbf{FFT課題1}}
\begin{maplelatex}\begin{Maple Normal}{
下の例に従って,sin(i/13)とsin(i/2)の合成波を作成し,FFTをかけた後,周波数での強度を表示せよ.合成波(2*sin(i*2)+sin(i/2))との違いをのべよ.}\end{Maple Normal}
\end{maplelatex}
\begin{maplegroup}
\begin{mapleinput}
\mapleinline{active}{1d}{restart:
funcs:=[sin(i/13),sin(i/2)];
#funcs:=[2*sin(i*2),sin(i/2)];
plot(funcs,i=0..300);
\begin{Maple Normal}{
}\end{Maple Normal}
}{}
\end{mapleinput}
\mapleresult
\begin{maplelatex}
\mapleinline{inert}{2d}{funcs := [sin((1/13)*i), sin((1/2)*i)]}{\[\displaystyle {\it funcs}\, := \,[\sin \left( i/13 \right) ,\sin \left( i/2 \right) ]\]}
\end{maplelatex}
\mapleresult
\mapleplot{FFTExplot2d1.eps}
\end{maplegroup}
\begin{maplegroup}
\begin{mapleinput}
\mapleinline{active}{1d}{data1:=[]:
for i from 1 to 256 do
  data1:=[op(data1),
   evalf(funcs[1]+funcs[2])]; 
#   evalf(funcs[1])]; 
end do:
with(plots): 
listplot(data1);
}{}
\end{mapleinput}
\mapleresult
\mapleplot{FFTExplot2d2.eps}
\end{maplegroup}
\begin{maplegroup}
\begin{mapleinput}
\mapleinline{active}{1d}{X:=array(data1): 
Y:=array(1..256,sparse):
FFT(8,X,Y);
}{}
\end{mapleinput}
\mapleresult
\begin{maplelatex}
\mapleinline{inert}{2d}{256}{\[\displaystyle 256\]}
\end{maplelatex}
\end{maplegroup}
\begin{maplegroup}
\begin{mapleinput}
\mapleinline{active}{1d}{Data2:=[seq([i,sqrt(X[i]\symbol{94}2+Y[i]\symbol{94}2)],i=1..128)]:
plot(Data2);
}{}
\end{mapleinput}
\mapleresult
\mapleplot{FFTExplot2d3.eps}
\end{maplegroup}
\begin{maplegroup}
\begin{mapleinput}
\mapleinline{active}{1d}{}{}
\end{mapleinput}
\end{maplegroup}
\end{document}
