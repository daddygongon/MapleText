\begin{itemize}
\item[課題]

下の例に従って,sin(i/13)とsin(i/2)の合成波を作成し,FFTをかけた後,周波数での強度を表示せよ.合成波(2*sin(i*2)+sin(i/2))との違いをのべよ.
\item[解答例]

\begin{MapleInput}
restart:
funcs:=[sin(i/13),sin(i/2)];
#funcs:=[2*sin(i*2),sin(i/2)];
plot(funcs,i=0..300);
\end{MapleInput}
\begin{MapleOutput}
{\it funcs}\, := \,[\sin \left( i/13 \right) ,\,\sin \left( i/2 \right) ]
\end{MapleOutput}
\MaplePlot{50mm}{./figures/FFTExplot2d1.eps}

\begin{MapleInput}
> data1:=[]:
for i from 1 to 256 do
  data1:=[op(data1),
   evalf(funcs[1]+funcs[2])]; 
#   evalf(funcs[1])]; 
end do:
with(plots): 
listplot(data1);
\end{MapleInput}
\MaplePlot{50mm}{./figures/FFTExplot2d2.eps}

\begin{MapleInput}
> X:=array(data1): 
Y:=array(1..256,sparse):
FFT(8,X,Y);
\end{MapleInput}
\begin{MapleOutput}
256
\end{MapleOutput}

\begin{MapleInput}
> Data2:=[seq([i,sqrt(X[i]^2+Y[i]^2)],i=1..128)]:
plot(Data2);
\end{MapleInput}
\begin{MapleOutput}
\end{MapleOutput}
\MaplePlot{100mm}{./figures/FFTExplot2d3.eps}

\end{itemize}
