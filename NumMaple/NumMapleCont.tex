\chapter{代数方程式(fsolve)}
\section{概要}
代数方程式の解f(x)=0を数値的に求めることを考える.標準的な
\begin{center}
二分法(bisection method)とニュートン法(Newton's method)
\end{center}
の考え方と例を説明し,
\begin{center}
収束性(convergency)と安定性(stability)
\end{center}
について議論する.さらに収束判定条件について言及する.


二分法のアイデアは単純.中間値の定理より連続な関数では,関数の符号が変わる二つの変数の間には根が必ず存在する.したがって,この方法は収束性は決して高くはないが,
確実.一方,Newton法は関数の微分を用いて収束性を速めた方法である.しかし,不幸にして収束しない場合や微分に時間がかかる場合があり,初期値や使用対象には注意
を要する.
 
\section{Mapleでの解}
Mapleでは代数方程式の解は,fsolveで求まる.
\begin{equation*}
x^2-4x+1 = 0
\end{equation*}
の解を考える.未知の問題では時として異常な振る舞いをする関数を相手にすることがあるので,先ずは関数の概形を見ることを常に心がけるべき.
\begin{MapleInput}
> restart;
> func:=x->x^2-4*x+1;
\end{MapleInput}
\begin{MapleOutput}
{\it func}\, := \,x\mapsto {x}^{2}-4\,x+1
\end{MapleOutput}
\begin{MapleInput}
> plot(func(z),z=-1..7);
\end{MapleInput}
\MaplePlot{60mm}{./figures/C2_fsolveplot2d1.eps}

もし,解析解が容易に求まるなら,その結果を使うほうがよい.Maple scriptの解析解を求めるsolveでは,

\begin{MapleInput}
> solve(func(x)=0,x);
\end{MapleInput}

\begin{table}[h]
\begin{center}
\begin{tabular}{|c|c|}
\hline
         &          \\ 
         &          \\ 
\hline
\end{tabular}
\end{center}
\label{default}
\end{table}%

と即座に求めてくれる.数値解は以下の通り求められる.
\begin{MapleInput}
> fsolve(func(x)=0,x);
\end{MapleInput}
\begin{table}[h]
\begin{center}
\begin{tabular}{|c|c|}
\hline
         &          \\ 
         &          \\ 
\hline
\end{tabular}
\end{center}
\label{default}
\end{table}%


 
\section{二分法とNewton法の原理}
\subsection{二分法(bisection)}
二分法は領域の端$x_1, x_2$で関数値$f(x_1),f(x_2)$を求め,中間の値を次々に計算して,解を囲い込んでいく方法である.
\begin{MapleInput}
> plot(func(z),z=0..0.8,gridlines=true);
\end{MapleInput}
\MaplePlot{70mm}{./figures/C2_fsolveplot2d2.eps}

\begin{table}[htbp]\begin{center}
\begin{tabular}{|c|c|c|c|}
\hline
$x_1$ & $x_2$ &$f(x_1)$ & $f(x_2)$ \\ \hline
0.0 & 0.8 &      &      \\ \hline
    &      &      &     \\ \hline
    &      &      &     \\ \hline
    &      &      &     \\ \hline
\end{tabular}
\end{center}\label{default}\end{table}%


\subsection{Newton法(あるいはNewton-Raphson法)}
Newton法は最初の点$x_1$から接線をひき,それが$x$軸(y=0)と交わった点を新たな点$x_2$とする.さらにそこでの接線を求めて...

という操作を繰り返しながら解を求める方法である.関数の微分をdf(x)とすると,これらの間には
\ifHIKI %%%%
||               ||
\else %%%%
\begin{equation*}
\setlength{\unitlength}{1cm}
\begin{picture}(10,1.5)
\put(0,0){\framebox(10,1.5){}}
\end{picture}
\end{equation*}
\fi %%%%
という関係が成り立つ.
\begin{MapleInput}
> df:=unapply(diff(func(x),x),x);
\end{MapleInput}
\ifHIKI %%%%
||               ||
\else %%%%
\begin{equation*}
\setlength{\unitlength}{1cm}
\begin{picture}(10,1.5)
\put(0,0){\framebox(10,1.5){}}
\end{picture}
\end{equation*}
\fi %%%%

\begin{MapleInput}
> with(plots):with(plottools): 
> x1:=1.0:x0:=0.0: 
> p:=plot(func(z),z=0..1.1):
> p1:=plot(df(x1)*(z-x1)+func(x1),z=0..1.1,color=blue):
> p2:=[disk([x1,func(x1)],0.02), disk([x0,0],0.02)]:
> display(p,p1,p2,gridlines=true);
\end{MapleInput}
\MaplePlot{70mm}{./figures/C2_fsolveplot2d3.eps}

\begin{table}[h]\begin{center}\begin{tabular}{|l|c|c|}
\hline
$x_1$ &$f(x_1)$ & $df(x_1)$ \\ \hline
1.0 &      &     \\  \hline
       &         &        \\  \hline
       &         &        \\  \hline
\end{tabular}\end{center}\label{default}\end{table}%

 
\section{二分法とNewton法のコード}
\subsection{二分法(bisection)}
\begin{MapleInput}
> x1:=0: x2:=0.8: 
  f1:=func(x1): f2:=func(x2): 
  for i from 1 to 5 do 
    x:=(x1+x2)/2;
    f:=func(x); 
    if f*f1>=0.0 then 
      x1:=x; f1:=f; 
    else 
      x2:=x; f2:=f; 
    end if;
    printf("%20.15f, %20.15f\n",x,f); 
  end do:
\end{MapleInput}
\begin{MapleError}
0.400000000000000, -0.440000000000000 
0.200000000000000,  0.240000000000000
0.300000000000000, -0.110000000000000 
0.250000000000000,  0.062500000000000
0.275000000000000, -0.024375000000000
\end{MapleError}

\subsection{Newton法(あるいはNewton-Raphson法)}

\begin{MapleInput}
> dfunc:=unapply(diff(func(z),z),z);
\end{MapleInput}
\begin{MapleOutput}
{\it dfunc} := z\mapsto 2 z - 4
\end{MapleOutput}
\begin{MapleInput}
> x:=1: f:=func(x): 
  printf("%15.10f, %+24.25f\n",x,f); 
  for i from 1 to 5 do
    x:=x-f/dfunc(x); 
    f:=func(x); 
    printf("%15.10f, %+24.25f\n",x,f); 
  end do:
\end{MapleInput}
\begin{MapleError}
1.0000000000, -2.0000000000000000000000000 
0.0000000000, +1.0000000000000000000000000 
0.2500000000, +0.0625000000000000000000000
0.2678571429, +0.0003188775510000000000000 
0.2679491900, +0.0000000084726737970000000 
0.2679491924, +0.0000000000000000059821834
\end{MapleError}

以下のようにDigitsを変更すれば,Mapleでは浮動小数点演算の有効数字を変えることができる.
\begin{MapleInput}
> Digits:=40;
\end{MapleInput}
\begin{MapleOutput}
40
\end{MapleOutput}

 
\section{収束性と安定性}
実際のコードの出力からも分かる通り,解の収束の速さは2つの手法で極端に違う.2分法では一回の操作で解の区間が半分になる.このように繰り返しごとに誤差幅が前回の誤差幅の定数($<1$)倍になる方法は1次収束(linear convergence)するという.Newton法では関数・初期値が素直な場合($f^{\prime}(x) <> 0$)に,収束が誤差の2乗に比例する2次収束を示す.以下はその導出を示した.


\begin{MapleInput}
> restart; ff:=subs(xi-x[f]=ei,series(f(xi),xi=x[f],4));
\end{MapleInput}

\begin{MapleOutput}
{\it ff}\, := \,f \left( x_{{f}} \right) +D \left( f \right)  \left( x_{{f}} \right) {\it ei}+\frac{1}{2}\,  D^{ \left( 2 \right) }   \left( f \right)  \left( x_{{f}} \right) {{\it ei}}^{2} +\frac{1}{6}\, 
D^{ \left( 3 \right) }   \left( f \right)  \left( x_{{f}} \right) {{\it ei}}^{3}+O \left( {{\it ei}}^{4} \right)
\end{MapleOutput}
\begin{MapleInput}
> dff:=subs({0=x[f],x=ei},series(diff(f(x),x),x,3));
\end{MapleInput}
\begin{MapleOutput}
{\it dff}\, := \,D \left( f \right)  \left( x_{{f}} \right) + 
D^{ \left( 2 \right) } \left( f \right)  \left( x_{{f}} \right) {\it ei}+
\frac{1}{2}\, D^{ \left( 3 \right) } \left( f \right)  \left( x_{{f}} \right) {{\it ei}}^{2} +O \left( {{\it ei}}^{3} \right)
\end{MapleOutput}
\begin{MapleInput}
> ei1:=ei-ff/dff;
\end{MapleInput}
\begin{MapleOutput}
{\it ei1}\, := \,{\it ei}-{\frac {f \left( x_{{f}} \right) +D \left( f \right)  \left( x_{{f}} \right) {\it ei}+\frac{1}{2}\,  D^{ \left( 2 \right) }  \left( f \right)  \left( x_{{f}} \right) {{\it ei}}^{2}+\frac{1}{6}\,  D^{ \left( 3 \right) }   \left( f \right)  \left( x_{{f}} \right) {{\it ei}}^{3}+O \left( {{\it ei}}^{4} \right) }{D \left( f \right)  \left( x_{{f}} \right) +  D^{ \left( 2 \right) }  \left( f \right)  \left( x_{{f}} \right) {\it ei} +\frac{1}{2}\, D^{ \left( 3 \right) } \left( f \right)  \left( x_{{f}} \right) {{\it ei}}^{2}+O \left( {{\it ei}}^{3} \right) }}
\end{MapleOutput}
\begin{MapleInput}
> ei2:=simplify(convert(ei1,polynom));
\end{MapleInput}
\begin{MapleOutput}
{\it ei2}\, := \,\frac{1}{3}\,\frac {3\, D^{ \left( 2 \right) }  \left( f \right)  \left( x_{{f}} \right) {{\it ei}}^{2}+2\, D^{ \left( 3 \right) } \left( f \right)  \left( x_{{f}} \right) {{\it ei}}^{3}
-6\,f \left( x_{{f}} \right) }{2\,D \left( f \right)  \left( x_{{f}} \right) +2\, D^{ \left( 2 \right) }   \left( f \right)  \left( x_{{f}} \right) {\it ei}+ D^{ \left( 3 \right) } \left( f \right)  \left( x_{{f}} \right) {{\it ei}}^{2}
}
\end{MapleOutput}
\begin{MapleInput}
> ei3:=series(ei2,ei,3);
\end{MapleInput}
\begin{MapleOutputGather}
{\it ei3}\, := \,-{\frac {f \left( x_{{f}} \right) }{D \left( f \right)  \left( x_{{f}} \right) }}+{\frac {f \left( x_{{f}} \right)  \left( D^{ \left( 2 \right) } \right)  \left( f \right)  \left( x_{{f}} \right) {\it ei}}{ \left( D \left( f \right)  \left( x_{{f}} \right)  \right) ^{2}}}+ \notag \\
\frac{1}{6}\, \frac{ 3\, \left( D^{ \left( 2 \right) } \right)  \left( f \right)  \left( x_{{f}} \right) +3\,{\frac {f \left( x_{{f}} \right)  \left( D^{ \left( 3 \right) } \right)  \left( f \right)  \left( x_{{f}} \right) }{D \left( f \right)  \left( x_{{f}} \right) }}-6\,{\frac {f \left( x_{{f}} \right)  \left(  \left( D^{ \left( 2 \right) } \right)  \left( f \right)  \left( x_{{f}} \right)  \right) ^{2}}{ \left( D \left( f \right)  \left( x_{{f}} \right)  \right) ^{2}}}}
{ \left( D \left( f \right)  \left( x_{{f}} \right)  \right)}{{\it ei}}^{2} +O \left( {{\it ei}}^{3} \right)
\end{MapleOutputGather}
\begin{MapleInput}
> subs(f(x[f])=0,ei3);
\end{MapleInput}
\begin{MapleOutput}
\frac{1}{2}\,{\frac {  D^{ \left( 2 \right) }   \left( f \right)  \left( x_{{f}} \right) {{\it ei}}^{2}}{D \left( f \right)  \left( x_{{f}} \right) }}+O \left( {{\it ei}}^{3} \right)
\end{MapleOutput}
注意すべきは,この収束性には一回の計算時間の差は入っていないことである.Newton法で解析的に微分が求まらない場合,数値的に求めるという手法がとられるが,これにかかる計算時間はばかにできない.二分法を改良した割線法(secant method)がより速い場合がある(NumRecipe9章参照).

二分法では,収束は遅いが,正負の関数値の間に連続関数では必ず解が存在するという意味で解が保証されている.しかし,Newton法では,収束は速いが,必ずしも素直に解に収束するとは限らない.解を確実に囲い込む,あるいは解に近い値を初期値に選ぶ手法が種々考案されている.解が安定であるかどうかは,問題,解法,初期値に大きく依存する.収束性と安定性のコントロールが数値計算のツボとなる.

 
\section{収束判定条件}
どこまで値が解に近づけば計算を打ち切るかを決める条件を収束判定条件と呼ぶ.以下のような条件がある.

\begin{table}[h]
\begin{center}
\begin{tabular}{|l|l|}

\ifHIKI

\hline
$\varepsilon$(イプシロン,epsilon)法 &  \\ 
\hline
$\delta$(デルタ,delta)法 &  \\ 
\hline
占部法 & 数値計算の際の丸め誤差までも含めて判定する条件で,$\left|f(x_{i+1})\right| > \left|f(x_i)\right|$とする. \\ 
\hline

\else

\hline
$\varepsilon$(イプシロン,epsilon)法 &  \\ 
&  \\
&  \\
\hline
$\delta$(デルタ,delta)法 &  \\ 
&  \\
& \\
\hline
占部法 & 数値計算の際の丸め誤差までも含めて判定する条件で,\\
& $\left|f(x_{i+1})\right| > \left|f(x_i)\right|$ \\
&とする. \\ 
\hline

\fi

\end{tabular}
\end{center}
\end{table}%

\paragraph{$\epsilon, \delta$を説明するための図} 
\begin{MapleInput}
> with(plots):with(plottools):
> f2:=x->0.4*(x^2-4*x+1):x1:=0.25:x0:=0.4:
  p1:=plot([f2(z),f2(x1)],z=0.2..0.5):
  p2:=[disk([x1,f2(x1)],0.005),disk([x0,f2(x0)],0.005)]:
  l1:=line([x0,f2(x0)],[x0,f2(x1)]):
  t1 := textplot([0.45,0.0,`epsilon`],align=above):
  t2 := textplot([0.325,0.05,`delta`],align=below):
> display(p,p1,p2,l1,t1,t2);
\end{MapleInput}
\MaplePlot{50mm}{./figures/C2_fsolveplot2d4.eps}

 
\section{2変数関数の場合}
2変数の関数では,解を求める一般的な手法は無い.この様子は実際に2変数の関数で構成される面の様子をみれば納得されよう.
\begin{MapleInput}
> restart;
> f:=(x,y)->4*x+2*y-6*x*y; g:=(x,y)->10*x-2*y+1;
\end{MapleInput}
\begin{MapleOutputGather}
f\, := \,( {x,y} )\mapsto 4\,x+2\,y-6\,xy \notag \\
g\, := \,( {x,y} )\mapsto 10\,x-2\,y+1 \notag
\end{MapleOutputGather}
\begin{MapleInput}
> p1:=plot3d({f(x,y)},x=-2..2,y=-2..2,color=red):
  p2:=plot3d({g(x,y)},x=-2..2,y=-2..2,color=blue):
  p3:=plot3d({0},x=-2..2,y=-2..2,color=gray):
  with(plots):
  display([p1,p2,p3],axes=boxed,orientation=[-150,70]);
\end{MapleInput}
\MaplePlot{60mm}{./figures/C2_fsolveplot3d5.eps}

解のある程度近くからは,Newton法で効率良く求められる.
\begin{MapleInput}
> fsolve({f(x,y)=0,g(x,y)=0},{x,y});
\end{MapleInput}
\begin{MapleOutput}
\left\{ x=- 0.07540291160,y= 0.1229854420 \right\}
\end{MapleOutput}
 
\section{例題:二分法とNewton法の収束性}
代数方程式に関する次の課題に答えよ.(2004年度期末試験)
\begin{enumerate}
\item $\exp(-x) = x^2$を二分法およびニュートン法で解け.
\item $n$回目の値$x_n$と小数点以下10桁まで求めた値$x_f=0.7034674225$との差$\Delta x_n$の絶対値(abs)のlogを$n$の関数としてプロットし,その収束性を比較せよ.また,その傾きの違いを両解法の原理から説明せよ.
\end{enumerate}

\subsubsection{解答例}
計算精度を40桁にしておく.funcで関数を定義.
\begin{MapleInput}
> restart; Digits:=40: func:=unapply(exp(-x)-x^2,x);
\end{MapleInput}
\begin{MapleOutput}
func\, := \,x\mapsto {e}^{-x}-{x}^{2}
\end{MapleOutput}
先ずは,関数をplotして概形を確認.
\begin{MapleInput}
> plot(func(x),x=-5..5,y=-10..10);
\end{MapleInput}
\MaplePlot{50mm}{./figures/fsolveExampleplot2d1.eps}
Mapleの組み込みコマンドで正解を確認しておく.
\begin{MapleInput}
> x0:=fsolve(func(x)=0,x);
\end{MapleInput}
\begin{MapleOutput}
0.7034674224983916520498186018599021303429
\end{MapleOutput}
テキストからプログラムをコピーして走らせてみる.環境によっては,printf分の中の"\verb|\|"が文字化けしているので修正.
\begin{MapleInput}
> x1:=0: x2:=0.8: res1:=[]:
  f1:=func(x1): f2:=func(x2):
  for i from 1 to 20 do
    x:=(x1+x2)/2;
    f:=func(x);
    if f*f1>=0.0 then
      x1:=x; f1:=f;
    else
      x2:=x; f2:=f;
    end if;
    printf("%20.15f, %20.15f\n",x,f); 
    res1:=[op(res1),[i,abs(x-x0)]]:
  end do:
\end{MapleInput}
\begin{MapleError}
0.400000000000000, 0.510320046035639 
0.600000000000000, 0.188811636094026
0.700000000000000, 0.006585303791410 
0.750000000000000, -0.090133447258985
0.725000000000000, -0.041300431044638 

中略

0.703468322753906, -0.000001712107681 
0.703467559814453, -0.000000261147873
\end{MapleError}
プロットのためにリストをres1で作成している.

同様にNewton法での結果をres2に入れる.
\begin{MapleInput}
> dfunc:=unapply(diff(func(z),z),z);
\end{MapleInput}
\begin{MapleOutput}
{\it dfunc}\, := \,z\mapsto -{{e}^{-z}}-2\,z
\end{MapleOutput}
\begin{MapleInput}
> x:=1.0: f:=func(x): 
  printf("%15.10f, %+24.25f\n",x,f); 
  res2:=[[1,abs(x-x0)]]:
  for i from 2 to 5 do
    x:=x-f/dfunc(x);
    f:=func(x);
    printf("%15.10f, %+24.25f\n",x,f); 
    res2:=[op(res2),[i,abs(x-x0)]]:
  end do:
\end{MapleInput}
\begin{MapleError}
1.0000000000, -0.6321205588285576784044762 
0.7330436052, -0.0569084480040254074684576 
0.7038077863, -0.0006473915387465014761973
0.7034674683, -0.0000000871660305624231097
0.7034674225, -0.0000000000000015809178420
\end{MapleError}

res1, res2を片対数プロットして同時に表示.
\begin{MapleInput}
> l1:=logplot(res1);
> l2:=logplot(res2);
> display(l1,l2);
\end{MapleInput}
\MaplePlot{50mm}{./figures/fsolveExampleplot2d2.eps}
2分法で求めた解は,Newton法で求めた解よりもゆっくりと精密解へ収束している.これは,二分法が原理的に計算回数について一次収束なのに対して,Newton法は2次収束であるためである.解の差($\delta$)だけでなく,関数値$f(x),\epsilon$をとっても同様の振る舞いを示す.
 
\section{課題}
\begin{enumerate}
\item Newton法の$f(x), df(x)$の関係を示す式を導け.
\item 次の関数 $f(x) = \exp(-x)-2\exp(-2x)$ の解を二分法,Newton法で求めよ.
\item 代数方程式に関する次の課題に答えよ.(2004年度期末試験)
\begin{enumerate}
\item $\exp(-x) = x^2$を二分法およびニュートン法で解け.
\item $n$回目の値$x_n$と小数点以下10桁まで求めた値$x_f=0.7034674225$との差$\Delta x_n$の絶対値(abs)のlogを$n$の関数としてプロットし,その収束性を比較せよ.また,その傾きの違いを両解法の原理から説明せよ.
\end{enumerate}
\item 次の方程式 $f(x) = x^4-x-0.12$ の正数解を二分法で求めよ.(2008年度期末試験)
\item 収束条件がうまく機能しない例を示せ.
\item 割線法は,微分がうまく求まらないような場合に効率がよい,二分法を改良した方法である.二分法では新たな点を元の2点の中点に取っていた.そのかわりに下図に示すごとく,新たな点を元の2点を直線で内挿した点に取る.二分法のコードを少し換えて,割線法のコードを書け.また,収束の様子を二分法,Newton法と比べよ.
\begin{MapleInput}
> func:=x->x^2-4*x+1: x1:=0: x2:=2: f1:=func(x1): f2:=func(x2):
> plot({(z-x1)*(f1-f2)/(x1-x2)+f1,func(z)},z=0..2);
\end{MapleInput}
\MaplePlot{50mm}{./figures/C2_fsolveplot2d6.eps}
\item 次の方程式 $f(x) = \cos(x)-x^2$ の正数解を二分法で求めよ.割線法でも求め,収束性を比べよ.(2009年度期末試験)
\item 次の方程式 $f(x)=x^3-3x+3$の解をニュートン法で求めよ.初期値をそれぞれ$x = -3, x = 2$とした時を比べ,その差について論ぜよ.(2010年度期末試験)
\end{enumerate}
 

\chapter{誤差(Error)}
\section{打ち切り誤差と丸め誤差(Truncation and round off errors)}
数値計算のねらいは,できるだけ精確・高速に解を得ることである.誤差 (精度) と収束性 (安定性,速度)
が数値計算のキモとなる.前回に説明した収束判定条件による誤差は打ち切り誤差 (truncation error)
と呼ばれる.ここでは,誤差のもう一つの代表例である,計算機に特有の丸め誤差 (roundoff error) について見ておこう.

\subsection{整数型と実数型の内部表現}
計算機は一般に無限精度の計算をおこなっているわけではない.CPU で足し算をおこなう以上,一般的な計算においては CPUが扱う一番効率のいい数の大きさが存在する.これが,32bit の CPU では 1 ワード,4byte(4x8bits) である.1ワードで表現できる最大整数は,符号に 1bit 必要なので,2\verb|^|(31)-1 となる.実数は以下のような仮数部と指数を取る浮動小数点数で表わされる.

\begin{table}[htbp]
\caption{浮動小数点数の内部表現 (IEEE754).}
\begin{center}
\begin{tabular}{c|c}
&$s \times f \times B ^{e-E}$ \\
\hline
s&sign bit(符号ビット:正負の区別を表す) \\
e&biased exponent(指数部) \\
f&fraction portion of the number(仮数部) \\\hline
B&base(基底) で通常は 2 \\
E&bias(下駄)と呼ばれる \\ \hline
real(単精度)&s=1, e=8, f=23 E=127 \\
double precision(倍精度)&s=1, e=11, f=52 E=1023
\end{tabular}
\end{center}
\label{default}
\end{table}%
Eは指数が負となる小数点以下の数を表現するためのもの.演算結果は実際の値から浮動小数点数に変換するための操作「丸め (round-off)」が常に行われる.それに伴って現れる誤差を丸め誤差と呼ぶ.

 
\section{有効桁数(Significant digits)}
1 ワードの整数の最大値とその2進数表示.
\begin{MapleInput}
> restart;
> 2^(4*8-1)-1;#res: 2147483647

\end{MapleInput}
この整数を2進数で表示するように変換するには,convert(n,binary)を用いて,
\begin{MapleInput}
> convert(2^(4*8-1)-1,binary);  #res: 1111111111111111111111111111111
\end{MapleInput}
となり,31個の1が並んでいることが分かる.1 ワードの整数の最大桁は,$n$ の桁数を戻すコマンドlength(n)を使って,
\begin{MapleInput}
> length(2^(4*8-1)-1); #res: 10
\end{MapleInput}
となり,たかだか10桁程度であることが分かる.一方,64bit の場合の整数の最大桁.
\begin{MapleInput}
> length(2^(8*8-1)-1); #res: 19
\end{MapleInput}
である.
                                    
Maple では多倍長計算するので,通常のプログラミング言語で起こるintの最大数あたりでの奇妙な振る舞いは示さない.

\begin{MapleInput}
> 2147483647+100;  #res: 2147483747
\end{MapleInput}

単精度の浮動小数点数は,仮数部2進数23bit,2倍長実数で52bitである.この有効桁数は以下の通り.
\begin{MapleInput}
> length(2^(23)); #res: 7
> length(2^(52)); #res: 16
\end{MapleInput}
 
\section{浮動小数点演算による過ち(FloatingPointArithmetic)}
「丸め」にともなって誤差が生じる. CやFortran等の通常のプログラミング言語では「丸める」仕様なのでプログラマーが気をつけなければならない.
\begin{MapleInput}
プログラムリスト : 実数のケタ落ち
#include <stdio.h>

int main(void){
  float a,b,c;
  double x,y,z;

  a=1.23456789;
  printf(" a= %17.10f\n",a);

  b=100.0;
  c=a+b;
  printf("%20.10f %20.10f %20.10f\n",a,b,c);

  x=(float)1.23456789;
  y=(double)100;
  z=x+y;
  printf("%20.12e %20.12e %20.12e\n",x,y,z);
 
  x=(double)1.23456789;
  y=(double)100;
  z=x+y;
  printf("%20.12e %20.12e %20.12e\n",x,y,z);

  return 0;
}
\end{MapleInput}

分かっているつもりでも,よくやる間違い.
\begin{MapleInput}
プログラムリスト : 丸め誤差
#include <stdio.h>

int main(void){
  float x=77777,y=7,y1,z,z1;
  y1=1/y;
  z=x/y;
  z1=x*y1;
  printf("%10.2f %10.2f\n",z,z1);
  if (z!=z1){
    printf("z is not equal to z1.\n");
  }
  printf("Surprising?? \n\n\n\n\n%10.5f %10.5f\n",z,z1);
  return 0; 
}
\end{MapleInput}
これを避けるには,EPSILONという小さな数字を定義しておいて,値の差の絶対値を求めるfabsを使って
\begin{table}[h]\begin{center}\begin{tabular}{|c|}
\hline
\hspace{100mm} \\ 
\\
\\
\hline
\end{tabular}\end{center}\end{table}%

とすべき.このときは数学関数であるfabsを使っているので,
\begin{MapleInput}
> gcc -lm test.c
\end{MapleInput}
とmath libraryを明示的に呼ぶのを忘れないように. 
\section{機械精度(Machine epsilon)}
上の例では,浮動小数点数で計算した場合に小さい数の差を区別することができなくなるということを示している.これは,CPUに固有の精度で,機械精度(Machine epsilon)と呼ばれる.つまり,小さい数を足したときにその計算機がその差を認識できなくなる限界ということで,以下のようにして求めることができる.
\begin{MapleInput}
> Digits:=7; 
> e:=evalf(1.0);
> w:=evalf(1.0+e); 
> while (w>1.0) do 
    printf("%-15.10e %-15.10e %-15.10e\n",e,w,evalf(w-1.0)); 
    e:=evalf(e/2.0); 
    w:=evalf(1.0+e); 
  end do:
\end{MapleInput}
\begin{MapleOutputGather}
7 \notag \\
1.0 \notag \\
2.0 \notag
\end{MapleOutputGather}
\begin{MapleError}
1.0000000000e+00 2.0000000000e+00 1.0000000000e+00
5.0000000000e-01 1.5000000000e+00 5.0000000000e-01
2.5000000000e-01 1.2500000000e+00 2.5000000000e-01
1.2500000000e-01 1.1250000000e+00 1.2500000000e-01
6.2500000000e-02 1.0625000000e+00 6.2500000000e-02
3.1250000000e-02 1.0312500000e+00 3.1250000000e-02
1.5625000000e-02 1.0156250000e+00 1.5625000000e-02
7.8125000000e-03 1.0078120000e+00 7.8120000000e-03
3.9062500000e-03 1.0039060000e+00 3.9060000000e-03
1.9531250000e-03 1.0019530000e+00 1.9530000000e-03
9.7656250000e-04 1.0009770000e+00 9.7700000000e-04
4.8828120000e-04 1.0004880000e+00 4.8800000000e-04
2.4414060000e-04 1.0002440000e+00 2.4400000000e-04
1.2207030000e-04 1.0001220000e+00 1.2200000000e-04
6.1035150000e-05 1.0000610000e+00 6.1000000000e-05
3.0517580000e-05 1.0000310000e+00 3.1000000000e-05
1.5258790000e-05 1.0000150000e+00 1.5000000000e-05
7.6293950000e-06 1.0000080000e+00 8.0000000000e-06
3.8146980000e-06 1.0000040000e+00 4.0000000000e-06
1.9073490000e-06 1.0000020000e+00 2.0000000000e-06
9.5367450000e-07 1.0000010000e+00 1.0000000000e-06
\end{MapleError} 
\section{桁落ち,情報落ち,積み残し(Cancellation)}
\paragraph{桁落ち(Cancellation)} 
\ifHIKI
\begin{MapleInput}
   0.723657
-  0.723649
------------

\end{MapleInput}
\else
\begin{equation*}
\begin{array}{rr}
  & 0.723657 \\
- & 0.723649\\ \hline
\\
\end{array}
\end{equation*}

\fi

\paragraph{情報落ち(Loss of Information)} 
\ifHIKI
\begin{MapleInput}
   72365.7
-      1.23659
------------

\end{MapleInput}
\else

\begin{equation*}
\begin{array}{rr@{.}l}
  & 72365&7 \\
+& 1&23659\\ \hline
\\
\end{array}
\end{equation*}
\fi

\paragraph{積み残し} 
\ifHIKI
\begin{MapleInput}
   72365.7
-      0.001
------------

\end{MapleInput}
\else
\begin{equation*}
\begin{array}{rr@{.}l}
  & 72365&7 \\
+& 0&001\\ \hline
\\
\end{array}
\end{equation*}
\fi 
\section{課題}
\begin{enumerate}
\item 次の項目について答えよ.(2004, 05, 06年度期末試験)
\begin{enumerate}
\item 数値計算の精度を制約するデータ形式とその特徴は何か.
\item 丸め誤差とは何か.
\item 打ち切り誤差とは何か.
\item 安定性とは何か.
\end{enumerate}
\item 10 進数 4 桁の有効桁数をもった計算機になったつもりで,以下の計算をおこなえ. 
 (a) 2718-0.5818 (b) 2718+0.5818 (c)  2718/0.5818 (d) 2718*0.5818
\item 自分の計算機で機械精度がどの位かを確かめよ.Maple スクリプトを参照して, C あるいは Fortran で作成し, 適当に調べよ.
\item (2147483647 + 100) を C あるいは Fortran で試せ.
\item 係数を a = 1,  b = 10000000,  c = 1としたときに, 通常の解の公式を使った解と, 解と係数の関係(下記の記述を参照)を使った解とを出力するプログラムを作成し, 解を比べよ.

2 次方程式 $ax^2+bx+c=0$の
係数$a, b, c$が特殊な値をもつ場合,通常の解の公式 

\begin{equation*}
x = \frac {-b \pm \sqrt{{b}^{2}-4ac}}{2a}
\end{equation*}
にしたがって計算するとケタ落ちによる間違った答えを出す.その特殊な値とは

\begin{equation*}
\sqrt{{b}^{2}-4ac} \approx |b|
\end{equation*}
となる場合である.

ケタ落ちを防ぐには, $b > 0$の場合は,  

\begin{equation*}
x_1 = \frac {-b - \sqrt{{b}^{2}-4ac}}{2a}
\end{equation*}
として,ケタ落ちを起こさずに求め, この解を使って, 解と係数の関係より

\begin{equation*}
x_2 = \frac {c}{a\,  x_1}
\end{equation*}
で求める.$b < 0$ の場合は,解の公式の足し算の方を使って同様に求める.

\end{enumerate}
 



\chapter{線形代数--写像(LAFundamentals)}
\section{行列と連立方程式}
大学の理系で必修なのは微積分と線形代数です.線形代数というと逆行列と固有値の計算がすぐに思い浮かぶでしょう.計算がややこしくてそれだけでいやになります.でも,行列の計算法は一連の手順で記述できるので,Mapleでは微積分とおなじように一個のコマンドで片が付きます.それが3x3以上でも同じです.問題はその意味です.ここでは,線形代数の計算がMapleを使えばどれほど簡単にできるかを示すと共に,線形代数の基本となる概念についてスクリプトと描画を使って,直観的に理解することを目的とします.

先ずは連立方程式から入っていきます.中学の時に
\begin{equation*}
4x = 2
\end{equation*}
というのを解きますよね.一般的には
\begin{equation*}
\begin {array}{rl}
ax &= b \\
x &= b/a
\end {array}
\end{equation*}
と書けるというのは皆さんご存知のはず.これと同じようにして連立方程式を書こうというのが逆行列の基本.つまり
\begin{equation*}
\begin {array}{rrl}
2x\, + &5y &=7 \\
4x\, + &y &=5
\end {array}
\end{equation*}
という連立方程式は,係数から作られる2x2行列を係数行列$A$,左辺の値で作るベクトルを$b$として,
\begin{equation*}
\begin {array}{rll}
Ax &= b & \\
x &= b/A &= A^{-1}b
\end {array}
\end{equation*}
としたいわけです.

実際にMapleでやってみましょう.行列は英語でMatrixです.
\begin{MapleInput}
> restart: A:=Matrix([[2,5],[4,1]]); 
\end{MapleInput}
\begin{MapleOutput}
A\, := \, \left[ \begin {array}{cc} 2&5\\ 4&1\end {array} \right]
\end{MapleOutput}
こうして行列を作ります.

\begin{MapleInput}
> b:=Vector([7,5]); #(2)ベクトルは英語でVectorです.これで縦ベクトルができます.
\end{MapleInput}
\begin{MapleOutput}
b\, := \, \left[ \begin {array}{c} 7\\ 5\end {array} \right]
\end{MapleOutput}
線形代数はlinear algebraと言います.withでLinearAlgebraというライブラリーパッケージを読み込んでおきます.
\begin{MapleInput}
> with(LinearAlgebra): 
\end{MapleInput}
逆行列はmatrix inverseと言います.
\begin{MapleInput}
> x0:=MatrixInverse(A).b;
\end{MapleInput}
行列AのMatrixInverseを求めて,ベクトルbに掛けています.
\begin{MapleOutput}
{\it x0}\, := \, \left[ \begin {array}{c} 1\\ 1\end {array} \right] 
\end{MapleOutput}
と簡単に求めることができます.
 
\section{掃き出し}
線形代数の計算にはあらかじめ関数パッケージ(LinearAlgebra)を呼び出しておく.
\begin{MapleInput}
> with(LinearAlgebra):
\end{MapleInput}

\subsection{行列の基本操作}
行列の掃き出しに必要となる行列の基本操作はRowOperation, ColumnOperationを参照.

\subsection{掃き出し法,LU分解(LUDecomposition)}
掃き出し法の計算は,LUDecompositionでおこなう.まず拡大係数行列を作る.
\begin{MapleInput}
> A1:=<1,2;3,4>; b:=<2,3>; <A1|b>;
\end{MapleInput}
\begin{MapleOutputGather}
{\it A1}\, := \, \left[ \begin {array}{cc} 1&2\\ 3&4\end {array} \right] \notag \\
b\, := \, \left[ \begin {array}{c} 2\\ 3\end {array} \right] \notag \\
\left[ \begin {array}{ccc} 1&2&2\\ 3&4&3\end {array} \right]  \notag
\end{MapleOutputGather}
これにLU分解をかける.それぞれP(permutation,置換), L(lower triangle,下三角), U(upper
triangle,上三角)行列に代入している.
\begin{MapleInput}
> P,L,U:=LUDecomposition(<A1|b>);
\end{MapleInput}
\begin{MapleOutput}
P,\,L,\,U\, := \, \left[ \begin {array}{cc} 1&0\\ 0&1\end {array} \right] ,\, \left[ \begin {array}{cc} 1&0\\ 3&1\end {array} \right] ,\, \left[ \begin {array}{ccc} 1&2&2\\ 0&-2&-3\end {array} \right] 
\end{MapleOutput}
さらに被約階段行列(row reduced echelonmatrix;後退代入までおこなって,解まで求めた状態)を求めるには,output='R'を指定する.
\begin{MapleInput}
> LUDecomposition(<A1|b>, output='R');
\end{MapleInput}
\begin{MapleOutput}
\left[ \begin {array}{ccc} 1&0&-1\\ 0&1&3/2\end {array} \right]
\end{MapleOutput}

\subsection{階数(Rank)}
行列の性質の中でも特に重要な階数(Rank)は次のコマンドで求められる.
\begin{MapleInput}
> Rank(A1);
\end{MapleInput}
\begin{MapleOutput}
2
\end{MapleOutput} 
\section{写像}
次に,これを2次元上のグラフで見てみましょう.先ず描画に必要なライブラリーパッケージ(plotsおよびplottools)をwithで読み込んでおきます.
\begin{MapleInput}
> with(plots):with(plottools): 
\end{MapleInput}
ベクトルは,位置座標を意味するようにlistへ変換(convert)しておきます.
\begin{MapleInput}
> p0:=convert(x0,list); p1:=convert(b,list); 
\end{MapleInput}
位置p0に円(disk)を半径0.2,赤色で描きます.同じように位置p1に半径0.2,青色でdiskを描きます.もう一つ,p0からp1に向かう矢印(arrow)
を適当な大きさで描きます.後ろの数字をいじると線の幅や矢印の大きさが変わります.
\begin{MapleInput}
> point1:=[disk(p0,0.2,color=red), disk(p1,0.2,color=blue)]:
> line1:=arrow(p0,p1,.05,.3,.1 ):
\end{MapleInput}

これらをまとめて表示(display)します.このとき,表示範囲を-8..8,-8..8とします.
\begin{MapleInput}
> display(point1,line1,view=[-2..8,-2..8],gridlines=true);
\end{MapleInput}
\MaplePlot{70mm}{./figures/LAFundamentalsplot2d1.eps}

逆行列は
\begin{MapleInput}
> MatrixInverse(A);
\end{MapleInput}
\begin{MapleOutput}
\left[ \begin {array}{cc} -1/18&5/18\\ 2/9&-1/9\end {array} \right]
%\left[  \begin {array}{cc} \displaystyle -\frac{1}{18}&  \displaystyle \frac{5}{18}\\
% \displaystyle  \frac{2}{9}&  \displaystyle -\frac{1}{9}\end {array} \right]
\end{MapleOutput}
で求まります.先ほどの矢印を逆に青から赤へたどる変換になっています.これが,連立方程式を解く様子をグラフで示しています.つまり,行列Aで示される変換によって求まる青点で示したベクトルb(7,5)を指す元の赤点を捜すというものです.答えは(1,1)となります.

では,元の赤点をもう少しいろいろ取って,行列Aでどのような点へ写されるかを見てみましょう.
\begin{MapleInput}
> N:=30:point2:=[]:line2:=[]: 
  for k from 0 to N-1 do
    x0:=Vector([sin(2*Pi*k/N),cos(2*Pi*k/N)]); 
    x1:=A.x0; 
    p0:=convert(x0,list);
    p1:=convert(x1,list); 
    point2:=[op(point2),disk(p0,0.05,color=red)];
    point2:=[op(point2),disk(p1,0.05,color=blue)]; 
    line2:=[op(line2),line(p0,p1)];
  end do:
\end{MapleInput}
N:=30で分割した円周上の点をx0で求めて,point2にその円とそれのA.x0を,line2にはその2点を結ぶline(線)を足しています.
使っているコマンドは,先ほどの描画とほぼ同じです.ただし,Mapleスクリプトに特有のidiom(熟語)を使っています.この基本形をとり出すと,
\begin{MapleInput}
> list1:=[]; 
  for k from 0 to 2 do 
    list1:=[op(list1),k]; 
  end do; 
  list1;
\end{MapleInput}
\begin{MapleOutputGather}
[] \notag \\
[0] \notag \\
[0, 1] \notag \\
[0, 1, 2] \notag \\
[0, 1, 2] \notag
\end{MapleOutputGather}
となります.for-loopでkを0から4まで回し,list1に次々と値を追加していくというテクです.

できあがりの次の図を見てください.
\begin{MapleInput}
> d:=6: display(point2,line2,view=[-d..d,-d..d]);
\end{MapleInput}
\MaplePlot{100mm}{./figures/LAFundamentalsplot2d2.eps}

何やっているか分かります? 中心の赤点で示される円が,青点で示される楕円へ写されていることが分かるでしょうか.

線形代数の講義で,写像を示すときによく使われるポンチ絵を現実の空間で示すとこのようになります.ポンチ絵では,赤で示した$V$空間が青で示した$W$空間へ行列$A$によって写像され,それぞれの要素$v$が$w$へ移されると意図しています.
\MaplePlot{60mm}{./figures/Projection3-4.eps}
 
\section{固有ベクトルの幾何学的意味}
では,ここでクイズです.固有ベクトルは上のグラフの何処に対応するか? ヒントは,
\begin{quotation}
行列Aの固有値,固有ベクトルを$\lambda, x_0$とすると,
\begin{equation*}
A \,x_0 = \lambda \, x_0
\end{equation*}
が成立する
\end{quotation}
です.固有値と固有ベクトルはMapleでは以下のコマンドで求まります.
\begin{MapleInput}
> lambda,P:=Eigenvectors(A);
\end{MapleInput}
\begin{MapleOutput}
\lambda,\,P\, := \, \left[ \begin {array}{c} -3\\6\end {array} \right] ,\, \left[ \begin {array}{cc} -1&5/4\\1&1\end {array} \right]
\end{MapleOutput}
ここではMapleコマンドのEigenvectorsで戻り値を$\lambda$(lambdaと書きます),$P$に代入しています.この後ろ側にある行列$P$の1列目で構成されるベクトルが固有値-3に対応する固有ベクトル,2列目のベクトルが固有値6に対応する固有ベクトルです.

\subsection{解答}
固有値$\lambda$,固有ベクトル$x_0$の関係式
\begin{equation*}
A \,x_0 = \lambda \, x_0
\end{equation*}
を言葉で言い直すと,
\begin{quote}
固有ベクトル$x_0$は変換行列$A$によって,自分の固有値倍のベクトル$\lambda x_0$に写されるベクトル
\end{quote}
となります.つまり変換の図で言うと,
\begin{quote}
変換しても方向が変わらない赤点(の方向)
\end{quote}
となります.これは図で書くと
\begin{MapleInput}
> vv1:=Column(P,1): vv2:=Column(P,2): 
  a1:=vv1[2]/vv1[1]: a2:=vv2[2]/vv2[1]:
  pp1:=plot({a2*x,a1*x},x=-d..d):
\end{MapleInput}
Columnによって行列の第i列目をとりだし,その比によって直線の傾きを求めています.そうして引いた2本の直線をpp1としてため込んで,先ほど描いた変換の図に加えて表示(display)させます.
\begin{MapleInput}
> display(point2,line2,pp1,view=[-d..d,-d..d]);
\end{MapleInput}
\MaplePlot{80mm}{./figures/LAFundamentalsplot2d3.eps}
pp1を入れて描いた直線が引かれた方向ではたしかに変換によっても方向が変わらなさそうに見えるでしょう.

おまけですが,行列の対角化は次のようにしてできます.
\begin{MapleInput}
> MatrixInverse(P).A.P;
\end{MapleInput}
\begin{MapleOutput}
\left[ \begin {array}{cc} -3&0\\ 0&6\end {array} \right]
\end{MapleOutput} 
\section{行列式の幾何学的意味}
行列Aの行列式($\left|A\right|$あるいはdet$A$と表記)はDeterminantで求まります.
\begin{MapleInput}
> Determinant(A);
\end{MapleInput}
\begin{MapleOutput}
-18
\end{MapleOutput}
では次のクイズ.先ほど求めた,行列Aの行列式は,どこに対応するでしょう?
以下の(1,0),(0,1)の点を変換した点に原点からベクトルを結んでその意味を説明してください.さらに,そのマイナスの意味は?.
\begin{MapleInput}
> point3:=[]:line3:=[]: XX:=Matrix([[1,0],[0,1]]): 
  for i from 1 to 2 do
    x0:=Column(XX,i); x1:=A.x0; 
    p0:=convert(x0,list): 
    p1:=convert(x1,list):
    point3:=[op(point3),disk(p0,0.2,color=red),disk(p1,0.2,color=blue)]:
    line3:=[op(line3),arrow([0,0],p0,.05,.3,.1 ),arrow([0,0],p1,.05,.3,.1 )]:
  end do:
  display(point3,line3,view=[-1..8,-1..8],gridlines=true);
\end{MapleInput}
\MaplePlot{80mm}{./figures/LAFundamentalsplot2d4.eps}
 
\section{行列式が0の写像}
では,行列式が0になるというのはどういう状態でしょう? 次のような行列を考えてみましょう.
\begin{MapleInput}
> A:=Matrix([[2,1],[4,2]]);
\end{MapleInput}
\begin{MapleOutput}
\end{MapleOutput}
この行列式は
\begin{MapleInput}
> Determinant(A);
\end{MapleInput}
\begin{MapleOutput}
0
\end{MapleOutput}
です.この変換行列で,上と同じように写像の様子を表示させてみましょう.
\begin{MapleInput}
> N:=30:point2:=[]:line2:=[]: 
  for k from 0 to N-1 do
    x0:=Vector([sin(2*Pi*k/N),cos(2*Pi*k/N)]); x1:=A.x0; p0:=convert(x0,list);
    p1:=convert(x1,list);
    point2:=[op(point2),disk(p0,0.05,color=red),disk(p1,0.05,color=blue)];
    line2:=[op(line2),line(p0,p1)]; 
  end:
> d:=6: display(point2,line2,view=[-d..d,-d..d]);
\end{MapleInput}
\MaplePlot{80mm}{./figures/LAFundamentalsplot2d5.eps}

わかります?

今回の移動先の青点は直線となっています.つまり,determinantが0ということは,変換すると面積がつぶれるという事を意味しています.平面がひとつ次元を落として線になるということです.

次に,この行列の表わす写像によって原点(0,0)に写される元の座標を求めてみます.連立方程式に戻してみると
\begin{MapleInput}
> A.Vector([x,y])=Vector([0,0]);
\end{MapleInput}
\begin{MapleOutput}
\left[ \begin {array}{c} 2\,x+y\\ 4\,x+2\,y\end {array} \right] = \left[ \begin {array}{c} 0\\ 0\end {array} \right]
\end{MapleOutput}
となります.とよく見ると,1行目も2行目もおなじ式になっています.2次元正方行列で,行列式が0の時には必ずこういう形になり,直線の式となります.これを表示すると
\begin{MapleInput}
> plot([-2*x,-2*x+1,-2*x-1],x=-4..4,y=-4..4);
> plot([2*x],x=-4..4,y=-4..4,color=blue);
\end{MapleInput}
\MaplePlot{100mm}{./figures/LAFundamentalsKernelImage.eps}
左図の赤線となります.この直線上の全ての点が[0,0]へ写されることを確認してください.また,緑の線上の点は全て[1,2]へ写されることが確認できます.
\begin{MapleInput}
> A.Vector([-1,2]);
\end{MapleInput}
\begin{MapleOutput}
\left[ \begin {array}{c} 0\\ 0\end {array} \right]
\end{MapleOutput}

こうしてすべて調べていけば,左の平面上のすべて点は右の青の直線上へ写されることが分かります.今まで見てきた円と楕円とはまったく違った写像が,行列式が0の行列では起こっていることが分かると思います.右の青線を行列Aによる像(Image, Im$A$と表記),左の赤線,つまり写像によって[0,0]へ写される集合を核(Kernel, Ker$A$と表記)と呼びます.

これをポンチ絵で描くと,次の通りです.
\ifHIKI

||像(Image) || 核(Kernel) 
||{{attach_view(Projection1.png,LAFundamentals)}}||{{attach_view(Projection2.png,LAFundamentals)}}

\else

\begin{center}
\begin{tabularx}{100mm}{|X|X|}
\hline
像(Image) & 核(Kernel) \\
\MaplePlot{40mm}{./figures/Projection1.eps}&
\MaplePlot{40mm}{./figures/Projection2.eps}\\
\hline
\end{tabularx}
\end{center}

\fi 
\section{全単射}
行列$A$による写像を$f$として,赤点に限らず元の点の集合を$V$, 移った先の点の集合を$W$とすると,
\begin{equation*}
f: V \rightarrow W
\end{equation*}
と表記されます.$v,w$を$V,W$の要素としたとき,異なる$v$が異なる$w$に写されることを単射,全ての$w$に対応する$v$がある写像を全射と言います.全単射,つまり全射でかつ単射,だと要素は一対一に対応します.先ほどのAは全射でもなく,単射でもない例です.

行列式が0の場合の写像は単射ではありません.このとき,逆写像が作れそうにありません.これを連立方程式に戻して考えましょう.もともと,
\begin{equation*}
v = A^{-1} w
\end{equation*}
の解$v$は点$w$が写像$A$によってどこから写されてきたかという意味を持ちます.逆写像が作れない場合は,連立方程式の解はパラメータをひとつ持った複数の解(直線)となります.これが係数行列の行列式が0の場合に,連立方程式の解が不定となる,あるいは像がつぶれるという関係です.

行列の次元が高い場合には,いろいろなつぶれかたをします.行列の階数と次元は
\begin{MapleInput}
> Rank(A); 
  Dimension(A);
\end{MapleInput}
\begin{MapleOutputGather}
1 \notag \\
2, 2 \notag
\end{MapleOutputGather}
で求まります.

Aをm行n列の行列とするとき,
\ifHIKI
""Rank(''A'') = Dimension (Im ''A'')

""Dimension (Ker ''A'') = ''n'' - Rank(''A'') 

\else

\begin{center}
Rank({\it A}) = Dimension (Im {\it A}) \\
Dimension ({\rm Ker} {\it A}) = {\it n} - Rank({\it A}) 
\end{center}

\fi
が成立し,これを次元定理といいます.
全射と単射の関係は,下の表のような一変数の方程式での解の性質の拡張と捉えることができます.

\begin{table}[htbp]
\caption{代数方程式$a x =b$の解の存在性.}
\begin{center}
\begin{tabular}{|c|l|l|}
\hline
一意&$a<>0$ &$x=b/a$ \\
不定&$a=0, b=0$ &解は無数\\ 
不能&$a=0, b<>0$ &解は存在しない\\
\hline
\end{tabular}
\end{center}
\label{default}
\end{table}%

\ifHIKI

||m x n行列A||全射でない(Im $A < m$), 値域上にあるときのみ解が存在||全射(Im $A =m$), 解は必ず存在
||単射でない(Ker $A <> 0$), 解は複数 ||{{attach_view(Projection3-1.png,LAFundamentals)}}||{{attach_view(Projection3-2.png,LAFundamentals)}}
||単射(Ker $A = 0$), 解はひとつ||{{attach_view(Projection3-3.png,LAFundamentals)}}||{{attach_view(Projection3-4.png,LAFundamentals)}}

\else

\begin{table}[htbp]
\caption{連立方程式$A x =b$の解の存在性.}
\begin{center}
\begin{tabularx}{150mm}{|X|X|X|}
\hline
$m \times n$行列$A$ &全射でない(Im $A < m$), 値域上にあるときのみ解が存在 &全射(Im$ A =m$),解は必ず存在 \\
\hline
単射でない(Ker $A <> 0$), 解は複数 & 
\MaplePlot{40mm}{./figures/Projection3-1.eps}&
\MaplePlot{40mm}{./figures/Projection3-2.eps}\\
\hline
単射(Ker $A = 0$), 解はひとつ &
\MaplePlot{40mm}{./figures/Projection3-3.eps} &
\MaplePlot{40mm}{./figures/Projection3-4.eps}\\
\hline
\end{tabularx}
\end{center}
\label{default}
\end{table}%

\fi

 
\section{課題}
\begin{enumerate}
\item 4x4の行列を適当に作り,Mapleで固有値を求めよ.求め方はマニュアルを参照せよ.
\item Jacobi法によって固有値を求めよ.
\item LAPACKに含まれているdsyev関数を用いて実対称行列の固有値を求めよ.(演習で詳しく取り上げている.研究や今後のために必要と思うときは,テキストを取りにおいで)
\end{enumerate} 


\chapter{線形代数--逆行列(LAMatrixInverse)}
\section{行列計算の概要}
数値計算の中心課題の一つである,行列に関する演算について見ていく.多次元,大規模な行列に対する効率のよい計算法が多数開発されており,多くの既存のライブラリが用意
されている.本章ではそれらの中心をなす,逆行列(matrix inverse)と固有値(Eigen
values)に関して具体的な計算方法を示す.現実的な問題には既存のライブラリを使うのが上策であるが,それでも基礎となる原理の理解や,ちょっとした計算,ライブラ
リの結果の検証に使えるルーチンを示す.

逆行列は連立一次方程式を解くことと等価である.ルーチン的なやり方にガウスの消去法がある.これは上三角行列になれば代入を適宜おこなうことで解が容易に求まることを利
用する.さらに,初期値から始めて次々に解に近づけていく反復法がある.この代表例であるJacobi(ヤコビ)法と,収束性を高めたGauss-Seidel(ガウス-
ザイデル)法を紹介する.

上記の手法をより高速にした修正コレスキー分解と共役傾斜(共役勾配)法があるが,少し複雑になるので割愛する.必要ならばNumRecipeを読め. 
\section{ガウス消去法による連立一次方程式の解}
逆行列は連立一次方程式を解くことと等価である.すなわち,$A$を行列,$x$を未知数ベクトル,$b$を数値ベクトルとすると,
\begin{equation*}
\begin{array}{rl} Ax &= b \\
A^{-1}Ax &= A^{-1}b \\
x &= A^{-1}b 
\end{array}
\end{equation*}
である.未知数の少ない連立一次方程式では,適当に組み合わせて未知数を消していけばいいが,未知数が多くなってしまうと破綻する.未知数の多い多元連立一次方程式で,ルーチン的に解を求めていく方法がガウス消去法で,前進消去と後退代入という2つの操作からなる.

後退代入(Backward substitution)による解の求め方を先ず見よう.たとえば,
\begin{equation*}
\begin{array}{rl}
x+y-2z & = -4 \\
-3y+3z & = 9\\
-z & = -2
\end{array}
\end{equation*}
では,下から順番に$z\rightarrow y\rightarrow x$と適宜代入することによって,簡単に解を求めることが出来る.係数で作る行列でこのような形をした上三角行列にする操作を前進消去あるいはガウスの消去法(Gaussian elimination)という.下三角行列L(lower triangular matrix)と上三角行列U(upper triangular matrix)の積に分解する操作
\begin{equation*}
A = L.U
\end{equation*}
をLU分解(LU decomposition)という.例えば先に示した上三角行列を係数とする連立方程式は,
\begin{equation*}
\begin{array}{rl}
x+y-2z&=-4 \\
x-2y+z&=5 \\
2x-2y-z&=2
\end{array}
\end{equation*}
を変形することで得られる.この変形を示せ.
\ifHIKI
\begin{tabular}{|c|}
\hline
         \\ 
\hline
\end{tabular}
\else
\begin{equation*}
\setlength{\unitlength}{1cm}
\begin{picture}(10,3.5)
\put(0,0){\framebox(10,3.5){}}
\end{picture}
\end{equation*}
\fi


  
 
\section{MapleによるLU分解}
線形代数の計算にはあらかじめ関数パッケージ(LinearAlgebra)を呼び出しておく.
\begin{MapleInput}
> with(LinearAlgebra):
\end{MapleInput}

\subsection{行列の基本操作}
行列の掃き出しに必要となる行列の基本操作はRowOperation, ColumnOperationを参照.

\subsection{掃き出し法,LU分解(LUDecomposition)}
掃き出し法の計算は,LUDecompositionでおこなう.まず拡大係数行列を作る.
\begin{MapleInput}
> A1:=<1,2;3,4>; b:=<2,3>; <A1|b>;
\end{MapleInput}
\begin{MapleOutputGather}
{\it A1}\, := \, \left[ \begin {array}{cc} 1&2\\ 3&4\end {array} \right] \notag \\
b\, := \, \left[ \begin {array}{c} 2\\ 3\end {array} \right] \notag \\
\left[ \begin {array}{ccc} 1&2&2\\ 3&4&3\end {array} \right]  \notag
\end{MapleOutputGather}
これにLU分解をかける.それぞれP(permutation,置換), L(lower triangle,下三角), U(upper
triangle,上三角)行列に代入している.
\begin{MapleInput}
> P,L,U:=LUDecomposition(<A1|b>);
\end{MapleInput}
\begin{MapleOutput}
P,\,L,\,U\, := \, \left[ \begin {array}{cc} 1&0\\ 0&1\end {array} \right] ,\, \left[ \begin {array}{cc} 1&0\\ 3&1\end {array} \right] ,\, \left[ \begin {array}{ccc} 1&2&2\\ 0&-2&-3\end {array} \right] 
\end{MapleOutput}
さらに被約階段行列(row reduced echelonmatrix;後退代入までおこなって,解まで求めた状態)を求めるには,output='R'を指定する.
\begin{MapleInput}
> LUDecomposition(<A1|b>, output='R');
\end{MapleInput}
\begin{MapleOutput}
\left[ \begin {array}{ccc} 1&0&-1\\ 0&1&3/2\end {array} \right]
\end{MapleOutput}

\subsection{階数(Rank)}
行列の性質の中でも特に重要な階数(Rank)は次のコマンドで求められる.
\begin{MapleInput}
> Rank(A1);
\end{MapleInput}
\begin{MapleOutput}
2
\end{MapleOutput} 
\section{LU分解のコード}
LU分解すれば線形方程式の解が容易に求まることは理解できると思う.具体的に$A$をLU分解する行列(消去行列と称す)T1,T2の係数は次のようにして求められる.
\begin{MapleInput}
> A0:=Matrix([[1,1,-2],[1,-2,1],[2,-2,-1]]): 
  b0:=Vector([-4,5,2]):
  A:=Matrix(A0): B:=Vector(b0): n:=3: 
  L:=Matrix(array(1..n,1..n,identity)): 
  for i from 1 to n do #i行目
    T[i]:=Matrix(array(1..n,1..n,identity)): 
                            #i番目の消去行列を作る
    for j from i+1 to n do 
      am:=A[j,i]/A[i,i];    #i行の要素を使って,i+1行目の先頭を消す係数を求める
      T[i][j,i]:=-am;       #i番目の消去行列に要素を入れる
      L[j,i]:=am;           #LTMの要素
      for k from 1 to n do
        A[j,k]:=A[j,k]-am*A[i,k]; #もとの行列をUTMにしていく
      end do; 
      B[j]:=B[j]-B[i]*am;   #数値ベクトルも操作
    end do; 
  end do:
\end{MapleInput}
\begin{MapleOutput}
\end{MapleOutput}

上のコードによって得られた消去行列.
\begin{MapleInput}
> T[1]; T[2];
\end{MapleInput}
\begin{MapleOutputGather}
\left[ \begin {array}{ccc} 1&0&0\\ -1&1&0\\ -2&0&1\end {array} \right] \notag \\
\left[ \begin {array}{ccc} 1&0&0\\ 0&1&0\\ 0&-4/3&1\end {array} \right] \notag 
\end{MapleOutputGather}
これを実際に元の行列$A0$に作用させると,UTMが求められる.
\begin{MapleInput}
> U:=T[2].T[1].A0;
\end{MapleInput}
\begin{MapleOutput}
U\, := \, \left[ \begin {array}{ccc} 1&1&-2\\ 0&-3&3\\ 0&0&-1\end {array} \right]
\end{MapleOutput}
求められたLTM, UTMを掛けると
\begin{MapleInput}
> L.U;
\end{MapleInput}
\begin{MapleOutput}
\left[ \begin {array}{ccc} 1&1&-2\\ 1&-2&1\\ 2&-2&-1\end {array} \right]
\end{MapleOutput}
元の行列を得られる.L,Aに求めたい行列が入っていることを確認.
\begin{MapleInput}
> L;A;
\end{MapleInput}
\begin{MapleOutputGather}
\left[ \begin {array}{ccc} 1&0&0\\ 1&1&0\\ 2&4/3&1\end {array} \right] \notag \\
\left[ \begin {array}{ccc} 1&1&-2\\ 0&-3&3\\ 0&0&-1\end {array} \right]  \notag 
\end{MapleOutputGather}
数値ベクトルも期待通り変換されている.
\begin{MapleInput}
> B;
\end{MapleInput}
\begin{MapleOutput}
\left[ \begin {array}{c} -4\\ 9\\ -2\end {array} \right]
\end{MapleOutput}

 
\section{ピボット操作}
ガウス消去法で困るのは,割ろうとした対角要素が0の場合である.しかし,この場合にも,方程式の順序を,行列の行と右辺の値をペアにして入れ替えれば解決する.この割る
ほうの要素をピボット要素あるいはピボット(pivot,バスケの軸足を動かさずにくるくる回すやつ)と呼ぶ.この操作は,変数の並びを変えたわけではなく,単に方程式の
順番を変更する操作に相当する.

さらに対角要素の数値が厳密に0でなくとも,極端に0に近づいた場合にも,その数で割った数値が大きくなり他の数との差を取ると以前に示した情報落ちの可能性が出てくる.
この現象を防ぐためには,絶対値が最大のピボットを選んで行の入れ替えを毎回おこなうといい結果が得られることが知られている.

MapleのLUDecompositionコマンドをこのような行列に適用すると,置換行列(permutation
matrix)Pが単位行列ではなく,ピボット操作に対応した行列となる.P.A=L.Uとなることに注意. 
\section{反復法による連立方程式の解}
以下のような連立方程式を
\begin{equation*}
\left[ \begin {array}{c} 5\,x+y+z+u\\ x+3\,y+z+u\\ x-2\,y-9\,z+u\\ x+3\,y-2\,z+5\,u\end {array} \right] = \left[ \begin {array}{c} -6\\ 2\\ -7\\ 3\end {array} \right]
\end{equation*}
形式的に解くと
\begin{equation*}
x=\frac{-6-(y+z+u)}{5}
\end{equation*}
となる.他の未知数も,
\ifHIKI
||y=||         ||
||z=||         ||
||u=||         ||
\else
\begin{equation*}
\setlength{\unitlength}{1cm}
\begin{picture}(5,4.5)
\put(0,3.0){y=\framebox(4,1.3){}}
\put(0,1.5){z=\framebox(4,1.3){}}
\put(0,0.0){u=\framebox(4,1.3){}}
\end{picture}
\end{equation*}
\fi
となる.適当に初期値($x_0,y_0,z_0,u_0$)をとり,下側の方程式に代入すると,得られた出力($x_1,y_1,z_1,u_1$)はより正解に近い値となる.これを繰り返すことによって正解が得られる.これをヤコビ(Jacobi)法と呼び,係数行列の対角要素が非対角要素にくらべて大きいときに適用できる.多くの現実の問題ではこの状況が成り立っている.

Gauss-Seidel法はJacobi法の高速版である.$n$番目の解の組が得られた後に一度に次の解の組に入れ替えるのではなく,得られた解を順次改良した解として使っていく.これにより,収束が早まる.以下にはヤコビ法のコードを示した.x1[i]の配列を変数に換えるだけで,Gauss-Seidel法となる.
\begin{MapleInput}
> AA:=Matrix([[5,1,1,1],[1,3,1,1],[1,-2,-9,1],[1,3,-2,5]]):
  b:=Vector([-6,2,-7,3]): n:=4; 
  x0:=[0,0,0,0]: x1:=[0,0,0,0]: 
  for iter from 1 to 20 do
    for i from 1 to n do
      x1[i]:=b[i]; 
      for j from 1 to n do
        x1[i]:=x1[i]-AA[i,j]*x0[j];
      end do:
      x1[i]:=x1[i]+AA[i,i]*x0[i];
      x1[i]:=x1[i]/AA[i,i];
    end do:
    x0:=evalf(x1);
    print(iter,x0);
  end do:
\end{MapleInput}
\begin{MapleError}
                                      4
          1, [-1.200000000, 0.6666666667, 0.7777777778, 0.6000000000]
          2, [-1.608888889, 0.6074074073, 0.5629629630, 0.7511111112]
          3, [-1.584296296, 0.7649382717, 0.5474897119, 0.7825185186]
          4, [-1.618989300, 0.7514293553, 0.5187050756, 0.6768921810]
          5, [-1.589405322, 0.8077973477, 0.5061160189, 0.6804222770]
          6, [-1.598867129, 0.8009556753, 0.4972691400, 0.6356490634]
          7, [-1.586774776, 0.8219829753, 0.4927633981, 0.6381076766]
          8, [-1.590570810, 0.8186345670, 0.4897074389, 0.6212705292]
          9, [-1.585922507, 0.8265309473, 0.4881589539, 0.6228163974]
         10, [-1.587501260, 0.8249823853, 0.4870924439, 0.6165295146]
         11, [-1.585720869, 0.8279597673, 0.4865626093, 0.6173477984]
         12, [-1.586374035, 0.8272701537, 0.4861897104, 0.6149933572]
         13, [-1.585690644, 0.8283969890, 0.4860087794, 0.6153885990]
         14, [-1.585958873, 0.8280977553, 0.4858782197, 0.6145034472]
         15, [-1.585695884, 0.8285257353, 0.4858165626, 0.6146844092]
         16, [-1.585805341, 0.8283983040, 0.4857707838, 0.6143503606]
         17, [-1.585703890, 0.8285613990, 0.4857498236, 0.6144303994]
         18, [-1.585748324, 0.8285078890, 0.4857337457, 0.6143038680]
         19, [-1.585709101, 0.8285702367, 0.4857266407, 0.6143384296]
         20, [-1.585727061, 0.8285480103, 0.4857209840, 0.6142903344]
\end{MapleError} 
\section{課題}
\begin{enumerate}
\item 4x4の行列を適当に作り,Mapleで固有値を求めよ.求め方はマニュアルを参照せよ.
\item Jacobi法によって固有値を求めよ.
\item LAPACKに含まれているdsyev関数を用いて実対称行列の固有値を求めよ.(演習で詳しく取り上げている.研究や今後のために必要と思うときは,テキストを取りにおいで)
\end{enumerate} 
\section{解答例}
4. Jacobi法のプログラムを参照してGauss-Seidel法のプログラムを作れ.Jacobi法と収束性を比べよ.
\begin{MapleInput}
#Gauss-Seidel
AA:=Matrix([[5,1,1,1],[1,3,1,1],[1,-2,-9,1],[1,3,-2,5]]):
b:=Vector([-6,2,-7,3]):
n:=4;
x0:=[0,0,0,0]:
x1:=[0,0,0,0]:
for iter from 1 to 20 do
for i from 1 to n do
  x1[i]:=b[i];
  for j from 1 to n do
    x1[i]:=x1[i]-AA[i,j]*x0[j];
  end do:
  x1[i]:=x1[i]+AA[i,i]*x0[i];
  x1[i]:=x1[i]/AA[i,i];
  x0:=evalf(x1);  #change here from ...
end do:
print(iter,x0);
end do:
\end{MapleInput}
\begin{MapleError}
                               4
   1, [-1.200000000, 1.066666667, 0.4074074073, 0.3629629628]
  2, [-1.567407407, 0.9323456790, 0.4367626887, 0.5287791494]
  3, [-1.579577503, 0.8713452217, 0.4673901337, 0.5800644210]
  4, [-1.583759955, 0.8454351333, 0.4783815777, 0.6008435420]
  5, [-1.584932051, 0.8352356437, 0.4828266893, 0.6089756998]
  6, [-1.585407607, 0.8312017393, 0.4845738460, 0.6121900162]
  7, [-1.585593120, 0.8296097527, 0.4852641546, 0.6134584342]
  8, [-1.585666468, 0.8289812930, 0.4855365978, 0.6139591570]
  9, [-1.585695410, 0.8287332183, 0.4856441456, 0.6141568092]
  10, [-1.585706835, 0.8286352933, 0.4856865986, 0.6142348304]
  11, [-1.585711344, 0.8285966383, 0.4857033566, 0.6142656284]
  12, [-1.585713125, 0.8285813800, 0.4857099714, 0.6142777856]
  13, [-1.585713827, 0.8285753567, 0.4857125829, 0.6142825846]
  14, [-1.585714105, 0.8285729793, 0.4857136134, 0.6142844788]
  15, [-1.585714214, 0.8285720407, 0.4857140204, 0.6142852266]
  16, [-1.585714258, 0.8285716703, 0.4857141809, 0.6142855218]
  17, [-1.585714275, 0.8285715240, 0.4857142443, 0.6142856384]
  18, [-1.585714281, 0.8285714660, 0.4857142694, 0.6142856844]
  19, [-1.585714284, 0.8285714433, 0.4857142792, 0.6142857024]
  20, [-1.585714285, 0.8285714343, 0.4857142831, 0.6142857096]
\end{MapleError}

 









\chapter{線形代数--固有値(LAEigen)}
\section{固有値}
では,ここでクイズです.固有ベクトルは上のグラフの何処に対応するか? ヒントは,
\begin{quotation}
行列Aの固有値,固有ベクトルを$\lambda, x_0$とすると,
\begin{equation*}
A \,x_0 = \lambda \, x_0
\end{equation*}
が成立する
\end{quotation}
です.固有値と固有ベクトルはMapleでは以下のコマンドで求まります.
\begin{MapleInput}
> lambda,P:=Eigenvectors(A);
\end{MapleInput}
\begin{MapleOutput}
\lambda,\,P\, := \, \left[ \begin {array}{c} -3\\6\end {array} \right] ,\, \left[ \begin {array}{cc} -1&5/4\\1&1\end {array} \right]
\end{MapleOutput}
ここではMapleコマンドのEigenvectorsで戻り値を$\lambda$(lambdaと書きます),$P$に代入しています.この後ろ側にある行列$P$の1列目で構成されるベクトルが固有値-3に対応する固有ベクトル,2列目のベクトルが固有値6に対応する固有ベクトルです.

\subsection{解答}
固有値$\lambda$,固有ベクトル$x_0$の関係式
\begin{equation*}
A \,x_0 = \lambda \, x_0
\end{equation*}
を言葉で言い直すと,
\begin{quote}
固有ベクトル$x_0$は変換行列$A$によって,自分の固有値倍のベクトル$\lambda x_0$に写されるベクトル
\end{quote}
となります.つまり変換の図で言うと,
\begin{quote}
変換しても方向が変わらない赤点(の方向)
\end{quote}
となります.これは図で書くと
\begin{MapleInput}
> vv1:=Column(P,1): vv2:=Column(P,2): 
  a1:=vv1[2]/vv1[1]: a2:=vv2[2]/vv2[1]:
  pp1:=plot({a2*x,a1*x},x=-d..d):
\end{MapleInput}
Columnによって行列の第i列目をとりだし,その比によって直線の傾きを求めています.そうして引いた2本の直線をpp1としてため込んで,先ほど描いた変換の図に加えて表示(display)させます.
\begin{MapleInput}
> display(point2,line2,pp1,view=[-d..d,-d..d]);
\end{MapleInput}
\MaplePlot{80mm}{./figures/LAFundamentalsplot2d3.eps}
pp1を入れて描いた直線が引かれた方向ではたしかに変換によっても方向が変わらなさそうに見えるでしょう.

おまけですが,行列の対角化は次のようにしてできます.
\begin{MapleInput}
> MatrixInverse(P).A.P;
\end{MapleInput}
\begin{MapleOutput}
\left[ \begin {array}{cc} -3&0\\ 0&6\end {array} \right]
\end{MapleOutput} 
\section{固有値の幾何学的意味}
次に,固有値の幾何学的な意味を2次元行列で確認しておこう.ある点$x_0$に対称正方行列$A$を作用すると,$x_1$に移動する.これを原点を中心とする円上の点に次々に作用させ,移動前後の点を結ぶ.

\begin{MapleInput}
> restart;
  with(LinearAlgebra):with(plots):with(plottools):
  A:=Matrix(1..2,1..2,[[3,2/3],[2/3,2]]):
\end{MapleInput}


\begin{MapleInput}
> N:=30:p1:=[]:l1:=[]:
for k from 0 to N-1 do
  x0:=Vector([sin(2*Pi*k/N),cos(2*Pi*k/N)]);
  x1:=MatrixVectorMultiply(A,x0);
  p1:=[op(p1),pointplot({x0,x1})];
  l1:=[op(l1),line( evalf(convert(x0,list)),evalf(convert(x1,list)) )];
end do:

> n:=4;
  display(p1,l1,view=[-n..n,-n..n]);
\end{MapleInput}
\MaplePlot{70mm}{./figures/EigenVectors3plot1.eps}

真ん中の円状の領域が,外側の楕円状の領域に写像されている様子が示されている.この図の中に固有ベクトル,固有値が隠れている.どこかわかる?

 
\section{Googleのページランク}
\begin{quote}
多くの良質なページからリンクされているページはやはり良質なページである
\end{quote}
Googleのpage rankは上のような非常に単純な仮定から成り立っている.ページランクを実際に求めよう.つぎのようなリンクが張られたページを考える.
\MaplePlot{70mm}{./figures/linkstruct.eps}

計算手順は以下の通り\footnote{詳しくは\texttt{http://www.kusastro.kyoto-u.ac.jp/\~baba/wais/pagerank.html}を参照せよ.}.
\begin{enumerate}
\item リンクを再現する隣接行列を作る.ページに番号をつけて,その間が結ばれているi-j要素を1,そうでない要素を0とする.
\item 隣接行列を転置する
\item 列ベクトルの総和が1となるように規格化する.
\item こうして得られた推移確率行列の最大固有値に属する固有ベクトルを求め,適当に規格化する.
\end{enumerate}

\subsection{課題}
\begin{enumerate}
\item 上記手順を参考にして,Mapleでページランクを求めよ.
\item このような問題ではすべての固有値・固有ベクトルを求める必要はなく,最大の固有値を示す固有ベクトルを求めるだけでよい.初期ベクトルを適当に決めて,何度も推移確率行列を掛ける反復法でページランクを求めよ.
\end{enumerate}
\begin{itemize}
\item[隣接行列]
\begin{equation*}
{\it A1}\, := \, \left[ \begin {array}{c|c|c|c|c|c|c|c} 
 &1&2&3&4&5&6&7\\
1&0&1&1&1&1&0&1\\
2&1&0&0&0&0&0&0\\
3& & & & & & & \\
4& & & & & & & \\
5& & & & & & & \\
6& & & & & & & \\
7& & & & & & & 
\end {array} \right] 
\end{equation*}
\item[転置行列]
\begin{equation*}
{Transpose}({\it A1})\, := \, \left[ \begin {array}{c|c|c|c|c|c|c} 
\, \, &\, \, &\, \, &\, \, &\, \, &\, \, &\, \, \\
 & & & & & & \\
 & & & & & & \\
 & & & & & & \\
 & & & & & & \\
 & & & & & & \\
 & & & & & & 
\end {array} \right] 
\end{equation*}
\item[規格化]
\begin{equation*}
\left[ \begin {array}{c|c|c|c|c|c|c} 
\, \, &\, \, &\, \, &\, \, &\, \, &\, \, &\, \, \\
 & & & & & & \\
 & & & & & & \\
 & & & & & & \\
 & & & & & & \\
 & & & & & & \\
 & & & & & & 
\end {array} \right] 
\end{equation*}
\item[遷移]
\begin{equation*}
\left( \begin {array}{ccccccc} 
0 &1 &1/2 &0 &1/4 &1/2 &0 \\
1/5 &0 &1/2 &1/3 &0 &0 &0 \\
1/5 &0 &0 &1/3 &1/4 &0 &0 \\
1/5 &0 &0 &0 &1/4 &0 &0 \\
1/5 &0 &0 &1/3 &0 &1/2 &1 \\
0 &0 &0 &0 &1/4 &0 &0 \\
1/5 &0 &0 &0 &0 &0 &0 
\end {array} \right) 
\left( \begin {array}{c} 
1/7\\ 
1/7\\ 
1/7\\ 
1/7\\ 
1/7\\ 
1/7\\ 
1/7 
\end {array} \right) \, = \, 
\left( \begin {array}{ccccccc} 
\, \, &\, \, &\, \, &\, \, &\, \, &\, \, &\, \, \\
 & & & & & & \\
 & & & & & & \\
 & & & & & & \\
 & & & & & & \\
 & & & & & & \\
 & & & & & & 
\end {array} \right)
\, = \, \left( \begin {array}{c} 
0.32\\ 
0.15\\ 
0.11\\ 
0.06\\ 
0.29\\ 
0.04\\ 
0.03 
\end {array} \right) 
\end{equation*}
\end{itemize}
\pagebreak
 
\section{累乗(べき乗)法により最大固有値が求まる原理}
累乗(べき乗)法は,最大固有値とその固有ベクトルを効率的に見つける算法である.すこし,固有値について復習しておく.正方行列$A$に対して,
\begin{equation}
A x = \lambda x
\end{equation}
の解$\lambda$を固有値,$x$を固有ベクトルという.$\lambda$は,
\begin{equation}
\det( A - \lambda E) =0
\end{equation}
として求まる永年方程式の解である.

では,なぜ適当な初期ベクトル$x_0$から始めて,反復
\begin{equation}
x_{k+1} = A x_k
\end{equation}
を繰り返すと,$A$の絶対値最大の固有値に属する固有ベクトルに近づいていくのかを見ておこう.

すべての固有値がお互いに異なる場合を考える.今,行列の固有値を絶対値の大きなもの順に並べて,$|\lambda_1|>|\lambda_2|>\cdots>|\lambda_n|$とし,対応する長さを1に規格化した固有ベクトルを$x_1, x_2, \ldots, x_n$とする.初期ベクトルは固有ベクトルの線形結合で表わせて,
\begin{equation}
X_0 = c_1x_1+c_2x_2+\cdots+c_nx_n
\end{equation}
となるとする.これに行列$A$を$N$回掛けると,
\begin{equation}
A^N X_0 = c_1 \lambda_1^N x_1+
c_2  \lambda_2^N x_2+\cdots+
c_n  \lambda_n^N x_n
\end{equation}
となる.これを変形すると,
\begin{equation}
A^NX_0 = X_{N}
= c_1 \lambda_1^N \left\{ x_1+
\frac{c_2}{c_1}\left(\frac{\lambda_2}{\lambda_1}\right)^N  x_2+\cdots+
\frac{c_n}{c_1}\left(\frac{\lambda_n}{\lambda_1}\right)^N  x_n \right\}
\end{equation}
となる.$|\lambda_1|>|\lambda_i|(i\ge2)$だから括弧の中は$x_1$だけが生き残る.

こうして最大固有値に属する固有ベクトルが,反復計算を繰り返すだけで求められる.

 
\section{Jacobi回転による固有値の求め方}
固有値を求める手法として,永年方程式を解くというやり方は回りくどすぎる.少し古めかしいが非対角要素を0にする回転行列を反復的に作用させるJacobi(ヤコビ)法を紹介する.現在認められている最適の方策は,ハウスホルダー(Householder)変換で行列を単純な三重対角化行列に変形してから,反復法で解を追い込んでいくやり方である.Jacobi法は,Householder法ほど万能ではないが,10次程度までの行列には今でも役に立つ.

\subsection{Mapleでみる回転行列}
行列の軸回転の復習をする.対称行列$B$に回転行列$U$を作用すると
\begin{equation}
B.U =  
\left(
\begin{array}{cc}
{a_{1\,1}} & {a_{1\,2}}\\
{a_{2\,1}(={a_{1\,2}})} & {a_{2\,2}}
\end{array}
 \right)
\left( 
\begin{array}{cc}
\cos(\theta) &  -\sin(\theta)\\
\sin(\theta) & \cos(\theta)
\end{array}
 \right) 
\end{equation}

\ifHIKI
\begin{tabular}{|c|}
\hline
         \\ 
\hline
\end{tabular}
\else
\begin{equation*}
\setlength{\unitlength}{1cm}
\begin{picture}(10,3.5)
\put(0,0){\framebox(10,3.5){}}
\end{picture}
\end{equation*}
\fi
となる.回転行列を4x4の行列に
\begin{equation}
U^t B U
\end{equation}
と作用させたときの各要素の様子を以下に示した.
\begin{MapleInput}
> restart:
> n:=4:
> with(LinearAlgebra):
> B:=Matrix(n,n,shape=symmetric,symbol=a);
\end{MapleInput}

\begin{MapleOutput}
B :=  \left[{
	\begin{array}{cccc}
	{a_{1, \,1}} & {a_{1, \,2}} & {a_{1, \,3}} & {a_{1, \,4}} \\
	{a_{1, \,2}} & {a_{2, \,2}} & {a_{2, \,3}} & {a_{2, \,4}} \\
	{a_{1, \,3}} & {a_{2, \,3}} & {a_{3, \,3}} & {a_{3, \,4}} \\
	{a_{1, \,4}} & {a_{2, \,4}} & {a_{3, \,4}} & {a_{4, \,4}}
	\end{array}}
 \right] 
\end{MapleOutput}

\begin{MapleInput}
> U:=Matrix(n,n,[[c,-s,0,0],[s,c,0,0],[0,0,1,0],[0,0,0,1]]);
#U:=Matrix(n,n,[[c,-s],[s,c]]);
\end{MapleInput}
\begin{MapleOutput}
U :=  \left[ 
{\begin{array}{ccrr}
c &  - s & 0 & 0 \\
s & c & 0 & 0 \\
0 & 0 & 1 & 0 \\
0 & 0 & 0 & 1
\end{array}}
 \right] 
\end{MapleOutput}

\begin{MapleInput}
>TT:=Transpose(U).B.U;
\end{MapleInput}

\begin{MapleOutputGather}
\mathit{TT} :=  \\ \notag
{\begin{array}{c}
 \left[ \right.  
  (c\,{a_{1, \,1}} + s\,{a_{1, \,2}})\,c + (c\,{a_{1, \,2}} + s
\,{a_{2, \,2}})\,s\,, \, - (c\,{a_{1, \,1}} + s\,{a_{1, \,2}})\,s
 + (c\,{a_{1, \,2}} + s\,{a_{2, \,2}})\,c\,,  \\
c\,{a_{1, \,3}} + s\,{a_{2, \,3}}\,, \,c\,{a_{1, \,4}} + s\,{a_{2
, \,4}}   \left. \right]  \\
 \left[ \right.  
  ( - s\,{a_{1, \,1}} + c\,{a_{1, \,2}})\,c + ( - s\,{a_{1, \,2
}} + c\,{a_{2, \,2}})\,s\,, \, - ( - s\,{a_{1, \,1}} + c\,{a_{1, 
\,2}})\,s + ( - s\,{a_{1, \,2}} + c\,{a_{2, \,2}})\,c\,,  \\
 - s\,{a_{1, \,3}} + c\,{a_{2, \,3}}\,, \, - s\,{a_{1, \,4}} + c
\,{a_{2, \,4}}   \left. \right] \\
 \left[   c\,{a_{1, \,3}} + s\,{a_{2, \,3}}\,, \, - s\,{a_{1, 
\,3}} + c\,{a_{2, \,3}}\,, \,{a_{3, \,3}}\,, \,{a_{3, \,4}}  
 \right]  \\
 \left[   c\,{a_{1, \,4}} + s\,{a_{2, \,4}}\,, \, - s\,{a_{1, 
\,4}} + c\,{a_{2, \,4}}\,, \,{a_{3, \,4}}\,, \,{a_{4, \,4}}  
 \right] 
 \end{array}}
\end{MapleOutputGather}

\begin{MapleInput}
>expand(TT[1,1]);
expand(TT[2,2]);
expand(TT[1,2]);
expand(TT[2,1]);
\end{MapleInput}
\begin{MapleOutput}
c^{2}\,{a_{1, \,1}} + 2\,c\,s\,{a_{1, \,2}} + s^{2}\,{a_{2, \,2}}
\end{MapleOutput}
\begin{MapleOutput}
s^{2}\,{a_{1, \,1}} - 2\,c\,s\,{a_{1, \,2}} + c^{2}\,{a_{2, \,2}}
\end{MapleOutput}
\begin{MapleOutput}
 - s\,c\,{a_{1, \,1}} - s^{2}\,{a_{1, \,2}} + c^{2}\,{a_{1, \,2}}
 + c\,s\,{a_{2, \,2}}
\end{MapleOutput}
\begin{MapleOutput}
 - s\,c\,{a_{1, \,1}} - s^{2}\,{a_{1, \,2}} + c^{2}\,{a_{1, \,2}}
 + c\,s\,{a_{2, \,2}}
\end{MapleOutput}
この非対角要素を0にする$\theta$は以下のように求まる.
\ifHIKI
\begin{tabular}{|c|}
\hline
         \\ 
\hline
\end{tabular}
\else
\begin{equation*}
\setlength{\unitlength}{1cm}
\begin{picture}(10,3.5)
\put(0,0){\framebox(10,3.5){}}
\end{picture}
\end{equation*}
\fi
このとき注目している$i,j=1,2$以外の要素も変化する.
\begin{MapleInput}
>expand(TT[3,1]);
expand(TT[3,2]);
\end{MapleInput}
\begin{MapleOutput}
c\,{a_{1, \,3}} + s\,{a_{2, \,3}}
\end{MapleOutput}
\begin{MapleOutput}
 - s\,{a_{1, \,3}} + c\,{a_{2, \,3}}
\end{MapleOutput}
これによって一旦0になった要素も値を持つが,なんども繰り返すことによって,徐々に0へ近づいていく. 
\section{Jacobi法による固有値を求めるCコード}
以下にはヤコビ法を用いた固有値と固有ベクトルを求めるコードを示した.結果は,固有値とそれに対応する規格化された固有ベクトルが縦(column)ベクトルで表示される.

リスト: ヤコビ法.
\begin{MapleInput}
#include <stdio.h>
#include <math.h>

#define M 10
void PrintMatrix(double a[M][M], int n);

int main(void){
  double a[M][M],v[M][M];
  double eps=0.0001,div,r,t,s,c,apj,aqj,aip,aiq,vip,viq;
  int i,j,n,iter,count,iterMax=1000000,p,q;
  
  scanf("%d",&n);
  for(i=1;i<=n;i++){
    for(j=1;j<=n;j++) scanf("%lf",&a[i][j]);
  }
  PrintMatrix(a,n);
  for(i=1;i<=n;i++){
    for(j=1;j<=n;j++) v[i][j]=0.;    
    v[i][i]=1.;
  }
  
  for(iter=1;iter<=iterMax;iter++){
    count=0;
    for(p=1;p<=n-1;p++){
      for(q=p+1;q<=n;q++){
	if(fabs(a[p][q])<eps) continue;
	count++;
	div=a[p][p]-a[q][q];
	if (div != 0.0){
	  r=2.0*a[p][q]/div;
	  t=0.5*atan(r);
	} else {
	  t=0.78539818;
	}
	s=sin(t);
	c=cos(t);
	for(j=1;j<=n;j++){
	  apj=a[p][j];
	  aqj=a[q][j];
	  a[p][j]=apj*c+aqj*s;
	  a[q][j]=-apj*s+aqj*c;
	}
	for(i=1;i<=n;i++){
	  aip=a[i][p];
	  aiq=a[i][q];
	  a[i][p]=aip*c+aiq*s;
	  a[i][q]=-aip*s+aiq*c;
	  vip=v[i][p];
	  viq=v[i][q];
	  v[i][p]=vip*c+viq*s;
	  v[i][q]=-vip*s+viq*c;
	}
	printf("p,q=%3d,%3d\n",p,q);
	PrintMatrix(a,n);
      }
    }
    if (count==0) break;
  }
  printf("Eigen values:\n");
  for(i=1;i<=n;i++) printf("%6.2f",a[i][i]);
  printf("\nEigen vectors:\n");
  PrintMatrix(v,n);
  
  return 0;
}

void PrintMatrix(double a[M][M], int n){
  int i,j;
  for(i=1;i<=n;i++){
    for(j=1;j<=n;j++) printf("%6.2f",a[i][j]);
    printf("\n");
  }
  printf("\n");
}
\end{MapleInput}


リスト: ヤコビ法の計算結果.

\begin{MapleError}
[BobsNewPBG4:~/NumRecipe/chap8] bob% cat input.txt
4
5 4 1 1
4 5 1 1
1 1 4 2
1 1 2 4

BobsNewPBG4:~/NumRecipe/chap8] bob% Jacobi2<input.txt
  5.00  4.00  1.00  1.00
  4.00  5.00  1.00  1.00
  1.00  1.00  4.00  2.00
  1.00  1.00  2.00  4.00

p,q=  1,  2
  9.00 -0.00  1.41  1.41
 -0.00  1.00 -0.00 -0.00
  1.41 -0.00  4.00  2.00
  1.41 -0.00  2.00  4.00

p,q=  1,  3
  9.37 -0.00 -0.00  1.88
 -0.00  1.00  0.00 -0.00
 -0.00  0.00  3.63  1.57
  1.88 -0.00  1.57  4.00

p,q=  1,  4
  9.96 -0.00  0.47 -0.00
 -0.00  1.00  0.00  0.00
  0.47  0.00  3.63  1.50
  0.00  0.00  1.50  3.41

...<中略>...

Eigen values:
 10.00  1.00  5.00  2.00
Eigen vectors:
  0.63 -0.71 -0.32  0.00
  0.63  0.71 -0.32  0.00
  0.32  0.00  0.63 -0.71
  0.32  0.00  0.63  0.71
\end{MapleError}

 
\section{数値計算ライブラリーについて}
一般の数値計算ライブラリーについては,時間の関係で講義ではその能力を紹介するにとどめる.昔の演習で詳しく取り上げていたので,研究や今後のために必要と思うときは,テキストを取りにおいで.

行列の計算は,数値計算の中でも特に利用する機会が多く,また,律速ルーチンとなる可能性が高い.そこで,古くから行列計算の高速ルーチンが開発されてきた.なかでもBLASとLAPACKはフリーながら非常に高速である. 

前回に示した,逆行列を求める単純なLU分解法をC言語でコーディングしたものと,LAPACKのルーチンを比べた場合,1000次元の行列で計測すると
\begin{MapleInput}
>  1000 [dim]     2.5200 [sec] #BOB
>  1000 [dim]     0.4700 [sec] #LAPACK
\end{MapleInput}
となった.用いたPCはMacBook(2GHz Interl Core Duo)であるが,この計算での0.47秒は1.4GFLOPに相当する.07年のMacBook(2GHz Interl Core 2 Duo)ではさらに早くなって
\begin{MapleInput}
bob% gcc -O3 bob.c -o bob
bob% ./bob
1000
 1000 [dim]     1.7543 [sec] #BOB
bob% gcc -O3 lapack.c -llapack -lblas -o lapack
bob% ./lapack
1000
 1000 [dim]     0.1893 [sec] #LAPACK
\end{MapleInput}
で,3.5GFLOPSが出ている.

ライブラリーは世界中の計算機屋さんがよってたかって検証しているので,バグがほとんど無く,また,高速である.初学者はライブラリーを使うべきである.ただし,下のサンプルプログラムの行列生成の違いのように,ブラックボックス化すると思わぬ間違い(ここではFortranとCでの行列の並び順の違いが原因)をしでかすことがあるので,プログラムに組み込む前に必ず小さい次元(サンプルコード)で検証しておくこと.

添付のコードはちょっと長いが時間があればフォローせよ.コンパイルは,OSXでは
\begin{MapleInput}
> gcc -O3 -UPRINT lapack.c -llapack -lblas
\end{MapleInput}
とすればできる.linuxではLAPACK, BLASがインストールされていれば,
\begin{MapleInput}
> #include <vecLib/vecLib.h>
\end{MapleInput}
をコメントアウトして,
\begin{MapleInput}
> gcc -O3 -DPRINT lapack.c -L/usr/local/lib64 -llapack -lblas -lg2c
\end{MapleInput}
などとすればコンパイルできるはず.

\paragraph{リスト: 西谷製lazy逆行列計算プログラム} 
\begin{MapleInput}
#include <stdio.h>
#include <stdlib.h>
#include <math.h>
#include <time.h>

//#undef PRINT
//#define PRINT

void printMatrix(double *a, double *b, long n);
int MatrixInverse(double *a, double *b, long n);

int main(void){
  clock_t start, end;
  int i,j;
  long n;
  double *a,*b;

  scanf("%ld",&n);

  a=(double *)malloc(n*n*sizeof(double));
  b=(double *)malloc(n*sizeof(double));

  for (i=0;i<n;i++){
    for (j=0;j<n;j++){
      a[i*n+j]= 2*(double) random() / RAND_MAX - 1.0;
    }
  }
  for (i=0;i<n;i++){
    b[i]= 2*(double) random() / RAND_MAX - 1.0;
  }
  printMatrix(a,b,n);

  start = clock();
  MatrixInverse(a,b,n);
  end = clock();
  printf("%5d [dim] %10.4f [sec] #BOB\n", 
	 n,(double)(end-start)/CLOCKS_PER_SEC);
  printMatrix(a,b,n);

  free(a);
  free(b);
  return 0;
}

int MatrixInverse(double *a, double *b, long n){
  double *x;
  double pvt=0.00005,am;
  int i,j,k;

  x=(double *)malloc(n*sizeof(double));

  for(i=0;i<n-1;i++){
    if(fabs(a[i*n+i])<pvt){
      printf("Pivot %3d=%10.5f is too small.\n",i,a[i*n+i]);
      return 1;
    }
    for(j=i+1;j<n;j++){
      am=a[j*n+i]/a[i*n+i];
      for(k=0;k<n;k++) a[j*n+k]-=am*a[i*n+k];
      b[j]-=am*b[i];
    }
  }
  //Backward substitution
  for(j=n-1;j>=0;j--){
    x[j]=b[j];
    for(k=j+1;k<n;k++){
      x[j]-=a[j*n+k]*x[k];
    }
    b[j]=x[j]/=a[j*n+j];
  }
  free(x);
  return 0;
}

void printMatrix(double *a, double *b, long n){
  int i,j;
#ifdef PRINT
  printf("\n");
  for (i=0;i<n;i++){
    for (j=0;j<n;j++){
      printf("%10.5f",a[i*n+j]);
    }
    printf(":%10.5f",b[i]);
    printf("\n");
  }
  printf("\n");
#endif
  return;
}
\end{MapleInput}


\paragraph{リスト : LAPACK謹製smart逆行列計算プログラム} 
\begin{MapleInput}
#include <stdio.h>
#include <stdlib.h>
#include <math.h>
#include <time.h>

//#define PRINT
//#undef PRINT

void printMatrix(double *a, double *b, long n);

int main(void){
  clock_t start, end;
  int i,j;
  double *a,*b;
  long n,nrhs=1, lda,ldb, info, *ipiv;

  scanf("%ld",&n);

  a=(double *)malloc(n*n*sizeof(double));
  b=(double *)malloc(n*sizeof(double));
  lda=ldb=n;
  ipiv=(long *)malloc(n*sizeof(long));
  
  for (i=0;i<n;i++){
    for (j=0;j<n;j++){
      a[j*n+i]= 2*(double) random() / RAND_MAX - 1.0;
    }
  }

  for (i=0;i<n;i++){
    b[i]= 2*(double) random() / RAND_MAX - 1.0;
  }
  printMatrix(a,b,n);

  start = clock();
  dgesv_(&n, &nrhs, a, &lda, ipiv, b, &ldb, &info);
  end = clock();
  printf("%5d [dim] %10.4f [sec] #LAPACK\n", 
	 n, (double)(end-start)/CLOCKS_PER_SEC);
  printMatrix(a,b,n);

  free(a);
  free(b);
  free(ipiv);

  return 0;
}

void printMatrix(double *a, double *b, long n){
  int i,j;
#ifdef PRINT
  printf("\n");
  for (i=0;i<n;i++){
    for (j=0;j<n;j++){
      printf("%10.5f",a[i*n+j]);
    }
    printf(":%10.5f",b[i]);
    printf("\n");
  }
  printf("\n");
#endif
  return;
}
\end{MapleInput}



 
\section{課題}
\begin{enumerate}
\item 4x4の行列を適当に作り,Mapleで固有値を求めよ.求め方はマニュアルを参照せよ.
\item Jacobi法によって固有値を求めよ.
\item LAPACKに含まれているdsyev関数を用いて実対称行列の固有値を求めよ.(演習で詳しく取り上げている.研究や今後のために必要と思うときは,テキストを取りにおいで)
\end{enumerate} 
\chapter{補間(interpolation)と数値積分(Integral)}
\section{概要:補間と近似}
数値計算の中心課題の一つである,行列に関する演算について見ていく.多次元,大規模な行列に対する効率のよい計算法が多数開発されており,多くの既存のライブラリが用意
されている.本章ではそれらの中心をなす,逆行列(matrix inverse)と固有値(Eigen
values)に関して具体的な計算方法を示す.現実的な問題には既存のライブラリを使うのが上策であるが,それでも基礎となる原理の理解や,ちょっとした計算,ライブラ
リの結果の検証に使えるルーチンを示す.

逆行列は連立一次方程式を解くことと等価である.ルーチン的なやり方にガウスの消去法がある.これは上三角行列になれば代入を適宜おこなうことで解が容易に求まることを利
用する.さらに,初期値から始めて次々に解に近づけていく反復法がある.この代表例であるJacobi(ヤコビ)法と,収束性を高めたGauss-Seidel(ガウス-
ザイデル)法を紹介する.

上記の手法をより高速にした修正コレスキー分解と共役傾斜(共役勾配)法があるが,少し複雑になるので割愛する.必要ならばNumRecipeを読め. 
\section{多項式補間(polynomial interpolation)}
データを単純に多項式で補間する方法を先ず示そう.$N+1$点をN次の多項式でつなぐ.この場合の補間関数は,
\begin{equation*}
F \left(x \right)={\sum_{i=0}^{N} } a _{i }x ^{i }=a_{0}+a_{1}x +a_{2}x^{2}+\cdots +a_{N}x^{N}
\end{equation*}
である.データの点を$(x_{i},\,y_{i}),i=0..N$とすると
\begin{equation*}
\begin{array}{cl}
a _{0}+a _{1}x _{0}+a _{2}x _{0}^{2}+\cdots +a _{N }x _{0}^{N }& =y _{0}\\
a _{0}+a _{1}x _{1}+a _{2}x _{1}^{2}+\cdots +a _{N }x _{1}^{N }& =y _{1}\\
\vdots& \\
a _{0}+a _{1}x _{N}+a _{2}x _{N}^{2}+\cdots +a _{N }x _{N}^{N }& =y _{N}
\end{array}
\end{equation*}
が,係数 $a_i$を未知数と見なした線形の連立方程式となっている.係数行列は
\begin{equation*}
A=\left[
\begin{array}{ccccc}
1&x_0&x_0^2&\cdots&x_0^N \\
1&x_1&x_1^2&\cdots&x_1^N \\
\vdots& & & & \vdots \\
1&x_N&x_N^2&\cdots&x_N^N 
\end{array} \right]
\end{equation*}
となる.$a_i$と$y_i$をそれぞれベクトルとみなすと
\ifHIKI %%%%
||               ||
\else %%%%
\begin{equation*}
\setlength{\unitlength}{1cm}
\begin{picture}(10,3.5)
\put(0,0){\framebox(10,3.5){}}
\end{picture}
\end{equation*}
\fi %%%%
により未知数ベクトル$a_i$が求まる.これは単純に,前に紹介した Gauss の消去法や LU 分解で解ける.

\subsection{Mapleによる多項式補間の実例}
\begin{MapleInput}
> restart; X:=[0,1,2,3]: Y:=[1,2,3,-2]:
> with(LinearAlgebra):
> list1:=[X,Y];
\end{MapleInput}
\begin{MapleOutput}
{\it list1}\, := \,[[0,1,2,3],[1,2,3,-2]]
\end{MapleOutput}
\begin{MapleInput}
> with(plots):
  l1p:=listplot(Transpose(Matrix(list1))):
  display(l1p);
\end{MapleInput}
\MaplePlot{50mm}{./figures/C7_InterpolationIntegralplot2d1.eps}

\begin{MapleInput}
> A:=Matrix(4,4): 
  for i from 1 to 4 do 
    for j from 1 to 4 do 
      A[i,j]:=X[i]^(j-1);
    end do; 
  end do:
  A;
\end{MapleInput}
\begin{MapleOutput}
\left[ \begin {array}{cccc} 1&0&0&0\\ 1&1&1&1\\ 1&2&4&8\\ 1&3&9&27\end {array} \right]
\end{MapleOutput}

\begin{MapleInput}
> a1:=MatrixInverse(A).Vector(Y);
\end{MapleInput}
\begin{MapleOutput}
{\it a1}\, := \, \left[ \begin {array}{c} 1\\ -1\\ 3\\ -1\end {array} \right]
\end{MapleOutput}
\begin{MapleInput}
> f1:=unapply(add(a1[ii]*x^(ii-1),ii=1..4),x);
\end{MapleInput}
\begin{MapleOutput}
{\it f1}\, := \,x\mapsto 1-x+3\,{x}^{2}-{x}^{3}
\end{MapleOutput}
\begin{MapleInput}
> f1p:=plot(f1(x),x=0..3): 
  l1p:=listplot(Transpose(Matrix(list1))):
  display(f1p,l1p);
\end{MapleInput}
\MaplePlot{50mm}{./figures/C7_InterpolationIntegralplot2d2.eps}
 
\section{Lagrange(ラグランジュ) の内挿公式}
多項式補間は手続きが簡単であるため,計算間違いが少なく,プログラムとして組むのに適している.しかし,あまり"みとうし"のよい方法とはいえない.その点,Lagrange(ラグランジュ)の内挿公式は見通しがよい.これは
\begin{equation*}
F(x)= \sum_{k=0}^N \frac{\displaystyle \prod_{j \ne k} (x-x_j)}
{\displaystyle \prod_{j \ne k} (x_k-x_j)} y_k
=\sum_{k=0}^N \frac{ \frac{\displaystyle (x-x_0)(x-x_1)\cdots(x-x_N)}{\displaystyle (x-x_k)}}
{\frac{\displaystyle (x_k-x_0)(x_k-x_1)\cdots(x_k-x_N)}{\displaystyle (x_k-x_k)}} y_k
\end{equation*}
と表わされる.数学的に 2つ目の表記は間違っているが,先に割り算を実行すると読み取って欲しい.これは一見複雑に見えるが,単純な発想から出発している.求めたい関数$F(x)$を
\begin{equation*}
F(x) = y_0 L_0(x)+y_1 L_1(x)+y_2 L_2(x)
\end{equation*}
とすると
\begin{equation*}
\begin{array}{ccc}
 L_0(x_0) = 1 &L_0(x_1) = 0 &L_0(x_2) = 0 \\
 L_1(x_0) = 0 &L_1(x_1) = 1 &L_1(x_2) = 0 \\
 L_2(x_0) = 0 &L_2(x_1) = 0 &L_2(x_2) = 1 
\end{array}
\end{equation*}
となるように関数$L_i(x)$を決めればよい.これを以下のようにとればLagrangeの内挿公式となる.
  
\ifHIKI %%%%
||               ||
\else %%%%
\begin{equation*}
\setlength{\unitlength}{1cm}
\begin{picture}(10,10)
\put(0,0){\framebox(10,10){}}
\end{picture}
\end{equation*}
\fi %%%%
 
\section{Newton(ニュートン) の差分商公式}
もう一つ有名なNewton(ニュートン) の内挿公式は,
\begin{equation*}
\begin{array}{rc}
F (x )&=F (x _{0})+
(x -x _{0})f _{1}\lfloor x_0,x_1\rfloor+
(x -x _{0})(x -x _{1})
f _{2}\lfloor x_0,x_1,x_2\rfloor + \\
& \cdots + \displaystyle \prod_{i=0}^{n-1} (x-x_i) \, 
f_n \lfloor x_0,x_1,\cdots,x_n \rfloor
\end{array}
\end{equation*}
となる.ここで$f_i \lfloor\, \rfloor$ は次のような関数を意味していて,
\begin{equation*}
\begin{array}{rl}
f _{1}\lfloor x_0,x_1\rfloor &= \displaystyle \frac{y_1-y_0}{x_1-x_0} \\
f _{2}\lfloor x_0,x_1,x_2\rfloor &= \displaystyle \frac{f _{1}\lfloor x_1,x_2\rfloor-
f _{1}\lfloor x_0,x_1\rfloor}{x_2-x_0} \\
\vdots & \\
f _{n}\lfloor x_0,x_1,\cdots,x_n\rfloor &= \displaystyle \frac{f_{n-1}\lfloor x_1,x_2\cdots,x_{n}\rfloor-
f _{n-1}\lfloor x_0,x_1,\cdots,x_{n-1}\rfloor}{x_n-x_0} \\
\end{array}
\end{equation*}
差分商と呼ばれる.得られた多項式は,Lagrange の内挿公式で得られたものと当然一致する.Newtonの内挿公式の利点は,新たなデータ点が増えたときに,新たな項を加えるだけで,内挿式が得られる点である.

\subsection{Newton補間と多項式補間の一致の検証}
関数$F(x)$を$x$の多項式として展開.その時の,係数の取るべき値と,差分商で得られる値が一致.
\begin{MapleInput}
> restart: F:=x->f0+(x-x0)*f1p+(x-x0)*(x-x1)*f2p;
\end{MapleInput}
\begin{MapleOutput}
F\, := \,x\mapsto f0+ ( x-x0 ) f1p+ ( x-x0 )  ( x-x1 ) f2p
\end{MapleOutput}
\begin{MapleInput}
> F(x1); 
  sf1p:=solve(F(x1)=f1,f1p);
\end{MapleInput}
\begin{MapleOutputGather}
f0+ ( x1-x0 ) f1p \notag \\
\displaystyle sf1p\, := \,{\frac {f0-f1}{-x1+x0}} \notag
\end{MapleOutputGather}
f20の取るべき値の導出
\begin{MapleInput}
> sf2p:=solve(F(x2)=f2,f2p); 
  fac_f2p:=factor(subs(f1p=sf1p,sf2p));
\end{MapleInput}
\begin{MapleOutputGather}
sf2p\, := \displaystyle -{\frac {f0+f1p\,x2-f1p\,x0-f2}{ ( -x2+x0 ) 
 ( -x2+x1 ) }} \notag \\
\displaystyle {\it fac\_f2p}\, := \displaystyle {\frac {f0\,x1-x2\,f0+x2\,f1-x0\,f1-f2\,x1+f2\,x0}{ ( -x1+x0 )  ( -x2+x0 )  ( -x2+x1 ) }} \notag
\end{MapleOutputGather}
ニュートンの差分商公式を変形
\begin{MapleInput}
> ff11:=(f0-f1)/(x0-x1); 
  ff12:=(f1-f2)/(x1-x2); 
  ff2:=(ff11-ff12)/(x0-x2);
  fac_newton:=factor(ff2);
\end{MapleInput}
\begin{MapleOutputGather}
ff11:= {\displaystyle \frac {f0-f1}{-x1+x0}} \notag \\
ff12 := {\displaystyle \frac {f1-f2}{-x2+x1}} \notag \\
ff2 :=  \frac{ {\displaystyle \frac {f0-f1}{-x1+x0}}-{\displaystyle \frac {f1-f2}{-x2+x1}} }{-x2+x0 }\notag \\
{\it fac\_newton} := {\displaystyle \frac {f0\,x1-x2\,f0+x2\,f1-x0\,f1-f2\,x1+f2\,x0}{ ( -x1+x0 )  ( -x2+x0 )  ( -x2+x1 ) }} \notag 
\end{MapleOutputGather}

二式が等しいかどうかをevalbで判定
\begin{MapleInput}
> evalb(fac_f2p=fac_newton);
\end{MapleInput}
\begin{MapleOutput}
true
\end{MapleOutput}
 
\section{数値積分 (Numerical integration)}
\documentclass[10pt,a4j]{jreport}
\usepackage[dvips]{graphicx,color}
\usepackage{verbatim}
\usepackage{amsmath,amsthm,amssymb}
\topmargin -15mm\oddsidemargin -4mm\evensidemargin\oddsidemargin
\textwidth 170mm\textheight 257mm\columnsep 7mm
\setlength{\fboxrule}{0.2ex}
\setlength{\fboxsep}{0.6ex}

\pagestyle{empty}

\newcommand{\MaplePlot}[2]{{\begin{center}
    \includegraphics[width=#1,clip]{#2}
                     \end{center}
%
} }

\newenvironment{MapleInput}{%
    \color{red}\verbatim
}{%
    \endverbatim
}

\newenvironment{MapleError}{%
    \color{blue}\verbatim
}{%
    \endverbatim
}

\newenvironment{MapleOutput}{%
    \color{blue}\begin{equation*}
}{%
    \end{equation*}
}

\newenvironment{MapleOutputGather}{%
    \color{blue}\gather
}{%
    \endgather
}
\newcommand{\ChartElement}[1]{{
	\color{magenta}\begin{flushleft}$\left[\left[\left[\textbf{\large #1}\right]\right]\right]$
	\end{flushleft}\vspace{-10mm}
} }

\newcommand{\ChartElementTwo}[1]{{
	\color{magenta}\begin{flushleft}$\left[\left[\left[\textbf{\large #1}\right]\right]\right]$
	\end{flushleft}
} }
\begin{document}
\pagebreak
\section{積分(int)}
\ChartElement{解説}

\subsection{\textbf{�ۑ�}}
\subsubsection{\textbf{\textit{1. ���̕s��ϕ������߂�D}}}
\begin{maplegroup}
\begin{Maple Normal}{
i)\mapleinline{inert}{2d}{int(4*x+3, x)}{$\displaystyle \int \!4\,x+3{dx}$}
�Cii)\mapleinline{inert}{2d}{int(1/(1+e^x), x)}{$\displaystyle \int \! \left( 1+{e}^{x} \right) ^{-1}{dx}$}
�Ciii)\mapleinline{inert}{2d}{int(1/(e^(-x)+e^x), x)}{$\displaystyle \int \! \left( {e}^{-x}+{e}^{x} \right) ^{-1}{dx}$}
�Civ)\mapleinline{inert}{2d}{int(sqrt(1-x^2), x)}{$\displaystyle \int \! \sqrt{1-{x}^{2}}{dx}$}
}\end{Maple Normal}

\end{maplegroup}
\subsubsection{\textbf{\textit{2. ���̒�ϕ������߂�D}}}
\begin{maplegroup}
\begin{Maple Normal}{
i)\mapleinline{inert}{2d}{int(sin*x, x = 0 .. Pi)}{$\displaystyle \int _{0}^{\pi }\! \left( \sin \right) \,x{dx}$}
�Cii)\mapleinline{inert}{2d}{int(arctan*x, x = 0 .. 1)}{$\displaystyle \int _{0}^{1}\! \left( \arctan \right) \,x{dx}$}
�Ciii)\mapleinline{inert}{2d}{int(1/sqrt(4-x^2), x = -2 .. 2)}{$\displaystyle \int _{-2}^{2}\! \left(  \sqrt{4-{x}^{2}} \right) ^{-1}{dx}$}
�Civ)\mapleinline{inert}{2d}{int(1/(x^2+x+1), x = 0 .. 1)}{$\displaystyle \int _{0}^{1}\! \left( {x}^{2}+x+1 \right) ^{-1}{dx}$}
}\end{Maple Normal}

\end{maplegroup}
\subsubsection{\textbf{\textit{3.(���W�ۑ�j����2�d�ϕ������߂�D}}}
\begin{maplegroup}
\begin{Maple Normal}{
\mapleinline{inert}{2d}{Typesetting:-mrow(Typesetting:-msubsup(Typesetting:-mo("&Integral;", mathvariant = "normal", fence = "unset", separator = "unset", stretchy = "true", symmetric = "unset", largeop = "true", movablelimits = "unset", accent = "unset", lspace = "0.0em", rspace = "0.0em"), Typesetting:-mi(""), Typesetting:-mi(""), superscriptshift = "0", subscriptshift = "0"), Typesetting:-msubsup(Typesetting:-mo("&Integral;", mathvariant = "normal", fence = "unset", separator = "unset", stretchy = "true", symmetric = "unset", largeop = "true", movablelimits = "unset", accent = "unset", lspace = "0.0em", rspace = "0.0em"), Typesetting:-mrow(Typesetting:-mi("D", italic = "false", mathvariant = "normal")), Typesetting:-mi(""), superscriptshift = "0", subscriptshift = "0"), Typesetting:-msqrt(Typesetting:-mrow(Typesetting:-msup(Typesetting:-mi("x", italic = "true", mathvariant = "italic"), Typesetting:-mrow(Typesetting:-mn("2", mathvariant = "normal")), superscriptshift = "0"), Typesetting:-mo("+", mathvariant = "normal", fence = "false", separator = "false", stretchy = "false", symmetric = "false", largeop = "false", movablelimits = "false", accent = "false", lspace = "0.2222222em", rspace = "0.2222222em"), Typesetting:-msup(Typesetting:-mi("y", italic = "true", mathvariant = "italic"), Typesetting:-mrow(Typesetting:-mn("2", mathvariant = "normal")), superscriptshift = "0"))), Typesetting:-mi(""), Typesetting:-mspace(height = "0.0ex", width = "0.5em", depth = "0.0ex", linebreak = "auto"), Typesetting:-mo("&DifferentialD;", mathvariant = "normal", fence = "unset", separator = "unset", stretchy = "unset", symmetric = "unset", largeop = "unset", movablelimits = "unset", accent = "unset", lspace = "0.0em", rspace = "0.0em"), Typesetting:-mi("x", italic = "true", mathvariant = "italic"), Typesetting:-mo(" ", mathvariant = "normal", fence = "false", separator = "false", stretchy = "false", symmetric = "false", largeop = "false", movablelimits = "false", accent = "false", lspace = "0.0em", rspace = "0.0em"), Typesetting:-mo("&DifferentialD;", mathvariant = "normal", fence = "unset", separator = "unset", stretchy = "unset", symmetric = "unset", largeop = "unset", movablelimits = "unset", accent = "unset", lspace = "0.0em", rspace = "0.0em"), Typesetting:-mi("y", italic = "true", mathvariant = "italic"), Typesetting:-mo(" ", mathvariant = "normal", fence = "false", separator = "false", stretchy = "false", symmetric = "false", largeop = "false", movablelimits = "false", accent = "false", lspace = "0.0em", rspace = "0.0em"), Typesetting:-mi(""), Typesetting:-mi("D", italic = "false", mathvariant = "normal"), Typesetting:-mo(":", mathvariant = "normal", fence = "false", separator = "false", stretchy = "false", symmetric = "false", largeop = "false", movablelimits = "false", accent = "false", lspace = "0.2777778em", rspace = "0.2777778em"), Typesetting:-mn("0", mathvariant = "normal"), Typesetting:-mo("&leq;", mathvariant = "normal", fence = "false", separator = "false", stretchy = "false", symmetric = "false", largeop = "false", movablelimits = "false", accent = "false", lspace = "0.2777778em", rspace = "0.2777778em"), Typesetting:-mi("y", italic = "true", mathvariant = "italic"), Typesetting:-mo("&leq;", mathvariant = "normal", fence = "false", separator = "false", stretchy = "false", symmetric = "false", largeop = "false", movablelimits = "false", accent = "false", lspace = "0.2777778em", rspace = "0.2777778em"), Typesetting:-mi("x", italic = "true", mathvariant = "italic"), Typesetting:-mo("&leq;", mathvariant = "normal", fence = "false", separator = "false", stretchy = "false", symmetric = "false", largeop = "false", movablelimits = "false", accent = "false", lspace = "0.2777778em", rspace = "0.2777778em"), Typesetting:-mn("1", mathvariant = "normal"))}{$\displaystyle \int _{}^{}\int _{\mathop{\rm D}}^{}\sqrt{x ^{2}+y ^{2}}d x \mathop{\rm  }d y \mathop{\rm  }\mathop{\rm D}:0\le y \le x \le 1$}
}\end{Maple Normal}

\end{maplegroup}
\begin{maplegroup}
\newpage
\end{maplegroup}
\subsection{\textbf{�𓚗�}}
\subsubsection{\textbf{\textit{1.}}}
\begin{maplegroup}
\begin{mapleinput}
\mapleinline{active}{1d}{int(4*x+3,x);
int( 1/(1+exp(x)),x);
int(1/(exp(-x)+exp(x)),x);
int(sqrt(1-x\symbol{94}2),x);
}{}
\end{mapleinput}
\mapleresult
\begin{maplelatex}
\mapleinline{inert}{2d}{2*x^2+3*x}{\[\displaystyle 2\,{x}^{2}+3\,x\]}
\end{maplelatex}
\mapleresult
\begin{maplelatex}
\mapleinline{inert}{2d}{-ln(1+exp(x))+ln(exp(x))}{\[\displaystyle -\ln  \left( 1+{{\rm e}^{x}} \right) +\ln  \left( {{\rm e}^{x}} \right) \]}
\end{maplelatex}
\mapleresult
\begin{maplelatex}
\mapleinline{inert}{2d}{arctan(exp(x))}{\[\displaystyle \arctan \left( {{\rm e}^{x}} \right) \]}
\end{maplelatex}
\mapleresult
\begin{maplelatex}
\mapleinline{inert}{2d}{(1/2)*x*sqrt(1-x^2)+(1/2)*arcsin(x)}{\[\displaystyle 1/2\,x \sqrt{1-{x}^{2}}+1/2\,\arcsin \left( x \right) \]}
\end{maplelatex}
\end{maplegroup}
\subsubsection{\textbf{\textit{2.}}}
\begin{maplegroup}
\begin{mapleinput}
\mapleinline{active}{1d}{int(sin(x),x=0..Pi);
int(arctan(x),x=0..1);
int(1/(sqrt(4-x\symbol{94}(2))),x=-2..2);
int(1/(x\symbol{94}2+x+1),x=0..1);
}{}
\end{mapleinput}
\mapleresult
\begin{maplelatex}
\mapleinline{inert}{2d}{2}{\[\displaystyle 2\]}
\end{maplelatex}
\mapleresult
\begin{maplelatex}
\mapleinline{inert}{2d}{(1/4)*Pi-(1/2)*ln(2)}{\[\displaystyle 1/4\,\pi -1/2\,\ln  \left( 2 \right) \]}
\end{maplelatex}
\mapleresult
\begin{maplelatex}
\mapleinline{inert}{2d}{Pi}{\[\displaystyle \pi \]}
\end{maplelatex}
\mapleresult
\begin{maplelatex}
\mapleinline{inert}{2d}{(1/9)*Pi*sqrt(3)}{\[\displaystyle 1/9\,\pi \, \sqrt{3}\]}
\end{maplelatex}
\end{maplegroup}
\subsubsection{\textbf{\textit{3.}}}
\begin{maplegroup}
\begin{mapleinput}
\mapleinline{active}{1d}{with(plots):
}{}
\end{mapleinput}
\end{maplegroup}
\begin{maplegroup}
\begin{mapleinput}
\mapleinline{active}{1d}{inequal(\{x-y>=0,x>=0,x<=1,y>=0\},x=-0.5..1.5,y=-0.5..1.5,optionsexcluded=(color=white));
}{}
\end{mapleinput}
\mapleresult
\mapleplot{Intplot2d1.eps}
\end{maplegroup}
\begin{maplegroup}
\begin{mapleinput}
\mapleinline{active}{1d}{f:=unapply(sqrt(x\symbol{94}2+y\symbol{94}2),(x,y)):
plot3d(f(x,y),x=0..1,y=0..1,axes=box);}{}
\end{mapleinput}
\mapleresult
\mapleplot{Intplot3d2.eps}
\end{maplegroup}
\begin{maplegroup}
\begin{mapleinput}
\mapleinline{active}{1d}{int(int(f(x,y),y=0..x),x=0..1);
}{}
\end{mapleinput}
\mapleresult
\begin{maplelatex}
\mapleinline{inert}{2d}{(1/6)*sqrt(2)+(1/6)*ln(1+sqrt(2))}{\[\displaystyle 1/6\, \sqrt{2}+1/6\,\ln  \left( 1+ \sqrt{2} \right) \]}
\end{maplelatex}
\end{maplegroup}
\begin{maplegroup}
\begin{mapleinput}
\mapleinline{active}{1d}{}{}
\end{mapleinput}
\end{maplegroup}
\end{document}
 
\subsection{studentパッケージいろいろ}
ちょっとぐらい難しい積分も,Mapleは単純にintコマンドだけで実行してくれる.しかし,時には,途中の計算法である部分積分,置換積分,部分分数展開が必要になる.このような計算はstudentパッケージに用意さている.
\begin{MapleInput}
> with(student):
\end{MapleInput}
\paragraph{部分積分(integration by parts)} 
\begin{MapleInput}
> intparts(Int(x*exp(x),x),x);
\end{MapleInput}
\begin{MapleOutput}
x\exp(x)-\int \exp(x){dx}
\end{MapleOutput}
\paragraph{置換(change of variables)による積分} 
\begin{MapleInput}
> Int((cos(x)+1)^3*sin(x), x);
> changevar(cos(x)+1=u, Int((cos(x)+1)^3*sin(x), x=a..b), u);
> changevar(cos(x)+1=u, int((cos(x)+1)^3*sin(x), x), u);
\end{MapleInput}
\begin{MapleOutputGather}
\int \left( \cos \left( x \right) +1 \right) ^{3}\sin \left( x \right) {dx} \notag \\
\int _{\cos \left( a \right) +1}^{\cos \left( b \right) +1}-{u}^{3}{du} \notag \\
 -\frac{1}{4}\,{u}^{4} \notag \\
\end{MapleOutputGather}
\paragraph{部分分数(partial fraction)展開による積分}
部分分数(partial fraction)展開による積分では,convertコマンドを用いる.
\begin{MapleInput}
> pf1:=convert(1/(1+x^3),parfrac,x);
  int(pf1,x);
\end{MapleInput}
\begin{MapleOutputGather}
 {\it pf1}\, := \frac{1}{3}\,{\frac {-x+2}{{x}^{2}-x+1}}+ \frac{1}{3\,\left( x+1 \right) } \notag \\
 -\frac{1}{6}\,\ln  \left( {x}^{2}-x+1 \right) +\frac{1}{3}\, \sqrt{3}\arctan \left( 1/3\, \left( 2\,x-1 \right)  \sqrt{3} \right) +\frac{1}{3}\,\ln  \left( x+1 \right) \notag
\end{MapleOutputGather}

 
\ChartElementTwo{課題}
\begin{enumerate}
\item 不定積分:次の不定積分を求めよ.

i) $\int 4\,x+3{dx}$
,ii)$\displaystyle \int  \frac{1}{ 1+\mbox{e}^{x} }{dx}$
,iii)$\displaystyle \int  \frac{1}{ \mbox{e}^{-x}+\mbox{e}^{x} }{dx}$
,iv)$\displaystyle \int  \sqrt{1-{x}^{2}}{dx}$


\item 定積分:次の定積分を求めよ.

i)$\displaystyle \int _{0}^{\pi } \sin x{dx}$
,ii)$\displaystyle \int _{0}^{1} \arctan x{dx}$
,iii)$\displaystyle \int _{-2}^{2} \frac{1}{ \sqrt{4-{x}^{2}} }{dx}$
,iv)$\displaystyle \int _{0}^{1} \frac{1}{ {x}^{2}+x+1 }{dx}$

\item (発展課題,重積分)次の2重積分を求めよ.
\begin{equation*}
\int \int_{D} \sqrt{x^2+y^2}dxdy\,\, D:0\leq y \leq x \leq 1
\end{equation*}
\end{enumerate} 
\ChartElementTwo{解答例}
\begin{enumerate}

\item 
\begin{MapleInput}
> int(4*x+3,x);
> int( 1/(1+exp(x)),x);
> int(1/(exp(-x)+exp(x)),x);
> int(sqrt(1-x^2),x);
\end{MapleInput}
\begin{MapleOutputGather}
\displaystyle 2\,{x}^{2}+3\,x  \notag \\
\displaystyle -\ln  \left( 1+ \mbox{e}^x \right) +\ln  \left( \mbox{e}^x \right)  \notag \\
\displaystyle \arctan \left( \mbox{e}^x \right)  \notag \\
\displaystyle \frac{1}{2}\,x \sqrt{1-{x}^{2}}+\frac{1}{2}\,\arcsin \left( x \right)  \notag
\end{MapleOutputGather}

\item
\begin{MapleInput}
> int(sin(x),x=0..Pi);
> int(arctan(x),x=0..1);
> int(1/(sqrt(4-x^(2))),x=-2..2);
> int(1/(x^2+x+1),x=0..1);
\end{MapleInput}
\begin{MapleOutputGather}
\displaystyle 2   \notag \\
\displaystyle \frac{1}{4}\,\pi -\frac{1}{2}\,\ln  \left( 2 \right)   \notag \\
\displaystyle \pi   \notag \\
\displaystyle \frac{1}{9}\,\pi \, \sqrt{3}  \notag
\end{MapleOutputGather}

\item
\begin{MapleInput}
> with(plots):
> inequal({x-y>=0,x>=0,x<=1,y>=0},x=-0.5..1.5,y=-0.5..1.5,optionsexcluded=(color=white));
\end{MapleInput}

\MaplePlot{40mm}{./figures/Intplot2d1.eps}

\begin{MapleInput}
> f:=unapply(sqrt(x^2+y^2),(x,y)):
> plot3d(f(x,y),x=0..1,y=0..1,axes=box);
\end{MapleInput}

\MaplePlot{40mm}{./figures/Intplot3d2.eps}
\begin{MapleInput}
> int(int(f(x,y),y=0..x),x=0..1);
\end{MapleInput}
\begin{MapleOutput}
\displaystyle \frac{1}{6} \sqrt{2}+\frac{1}{6}\,\ln  \left( 1+ \sqrt{2} \right) 
\end{MapleOutput}

\end{enumerate} 

\end{document}
 
\section{数値積分のコード}
次の積分を例に,Mapleのコードを示す.
\begin{equation*}
\int_0^1 \frac{4}{1+x^2} \, dx
\end{equation*}
先ずは問題が与えられたらできるだけMapleで解いてしまう.答えをあらかじめ知っておくと間違いを見つけるのが容易.プロットしてみる.
\begin{MapleInput}
> restart; 
  f1:=x->4/(1+x^2); 
  plot(f1(x),x=0..5);
\end{MapleInput}
\begin{MapleOutput}
{\it f1}\, := \,x\mapsto \frac{4}{1+{x}^{2}}
\end{MapleOutput}
\MaplePlot{50mm}{./figures/C7_InterpolationIntegralplot2d3.eps}
Mapleで解いてみる.
\begin{MapleInput}
>int(f1(x),x=0..1);
\end{MapleInput}
\begin{MapleOutput}
\pi
\end{MapleOutput}
えっと思うかも知れないが,
\begin{MapleInput}
>int(1/(1+x^2),x);
\end{MapleInput}
\begin{MapleOutput}
arctan(x)
\end{MapleOutput}
となるので,納得できるでしょう.

具体的にMapleでコードを示す.先ずは初期設定.
\begin{MapleInput}
>N:=8: x0:=0: xn:=1: Digits:=20:
\end{MapleInput}

\paragraph{Midpoint rule(中点法)} 
\begin{MapleInput}
> h:=(xn-x0)/N: S:=0: 
  for i from 0 to N-1 do 
    xi:=x0+(i+1/2)*h; 
    dS:=h*f1(xi);
    S:=S+dS; 
  end do: 
  evalf(S);
\end{MapleInput}
\begin{MapleOutput}
3.1428947295916887799
\end{MapleOutput}

\paragraph{Trapezoidal rule(台形公式)} 
\begin{MapleInput}
> h:=(xn-x0)/N: S:=f1(x0)/2: 
  for i from 1 to N-1 do 
    xi:=x0+i*h; 
    dS:=f1(xi);
    S:=S+dS; 
  end do: 
  S:=S+f1(xn)/2: 
  evalf(h*S);
\end{MapleInput}
\begin{MapleOutput}
3.1389884944910890093
\end{MapleOutput}

\paragraph{Simpson's rule(シンプソンの公式)} 
\begin{MapleInput}
> M:=N/2: h:=(xn-x0)/(2*M): Seven:=0: Sodd:=0: 
  for i from 1 to 2*M-1 by 2 do
    xi:=x0+i*h; 
    Sodd:=Sodd+f1(xi); 
  end do: 
  for i from 2 to 2*M-1 by 2 do
    xi:=x0+i*h; 
    Seven:=Seven+f1(xi); 
  end do:
  evalf(h*(f1(x0)+4*Sodd+2*Seven+f1(xn))/3);
\end{MapleInput}
\begin{MapleOutput}
3.1415925024587069144
\end{MapleOutput}

 
\section{課題}
\begin{enumerate}
\item 4x4の行列を適当に作り,Mapleで固有値を求めよ.求め方はマニュアルを参照せよ.
\item Jacobi法によって固有値を求めよ.
\item LAPACKに含まれているdsyev関数を用いて実対称行列の固有値を求めよ.(演習で詳しく取り上げている.研究や今後のために必要と思うときは,テキストを取りにおいで)
\end{enumerate} 

\chapter{線形最小2乗法(LeastSquareFit)}
\section{Mapleによる最小2乗法}
前章では,データに多項式を完全にフィットする補間についてみた.今回は,近似的にフィットする最小二乗法について詳しくみていく.図のようなデータに直線をフィットする場合を考えよう.

\MaplePlot{50mm}{./figures/C8_LeastSquareFitplot2d1.eps}

コマンドleastsquareによるfitting(2変数の例)
\begin{MapleInput}
> restart: X:=[1,2,3,4]: Y:=[0,5,15,24]:
> with(plots):with(linalg):with(stats):
> l1:=pointplot(transpose([X,Y]),symbolsize=30):
> eq_fit:=fit[leastsquare[[x, y], y = a0+a1*x, {a0,a1}]]([X, Y]);
\end{MapleInput}
\begin{MapleOutput}
eq\_fit\, := \,y=-\frac{19}{2}+{\frac {41}{5}}\,x
\end{MapleOutput}
\begin{MapleInput}
> f1:=unapply(rhs(eq_fit),x);
\end{MapleInput}
\begin{MapleOutput}
f1\, := \,x\mapsto -\frac{19}{2}+{\frac {41}{5}}\,x
\end{MapleOutput}
\begin{MapleInput}
> p1:=plot(f1(x),x=0..4):
> display(p1,l1);
\end{MapleInput}
\MaplePlot{50mm}{./figures/C8_LeastSquareFitplot2d2.eps}
 
\section{最小2乗法の原理}
もっとも簡単な例で原理を解説する.近似関数として,
\begin{equation*}
F(x) = a_0+a_1\,x
\end{equation*}
という直線近似を考える.もっともらしい関数は$N$点の測定データとの差$d_i = F(x_i)-y_i$を最小にすればよさそうであるが,これはプラスマイナスですぐに消えて不定になる.そこで,
\begin{equation*}
\chi^{2}=\sum_i^N d_i^2=\sum_i^N\left(a_0+a_1\,x_i-y_i\right)^2
\end{equation*}
という関数を考える.この$\chi^2$(カイ二乗)関数が,$a_0, a_1$をパラメータとして変えた時に最小となる$a_0, a_1$を求める.これは,それらの微分がそれぞれ0となる場合である.これは$\chi^2$の和$\sum$(sum)の中身を展開し,
\ifHIKI %%%%
||$\chi^2=$||            ||

\else %%%%

\begin{table}[htbp]
\begin{center}
\begin{tabular}{cc}
$\chi^2=$& \\
&
\setlength{\unitlength}{1cm}
\begin{picture}(10,6.5)
\put(0,0){\framebox(10,6.5){}}
\end{picture}

\end{tabular}
\end{center}
\end{table}%
\fi %%%%

$a_0, a_1$でそれぞれ微分すれば
\ifHIKI %%%%
||$\displaystyle \frac{\partial}{\partial a_0} \chi^2 =$||            ||
||$\displaystyle \frac{\partial}{\partial a_1} \chi^2 =$||            ||
\else %%%%

\begin{table}[htbp]
\begin{center}
\begin{tabular}{cc}
$\displaystyle \frac{\partial}{\partial a_0} \chi^2 =$& 
\setlength{\unitlength}{1cm}
\begin{picture}(10,3.5)
\put(0,0){\framebox(10,3.5){}}
\end{picture}  \\
$\displaystyle \frac{\partial}{\partial a_1} \chi^2 =$&
\setlength{\unitlength}{1cm}
\begin{picture}(10,3.5)
\put(0,0){\framebox(10,3.5){}}
\end{picture}  \\
\end{tabular}
\end{center}
\end{table}%
\fi %%%%

という$a_0, a_1$を未知変数とする2元の連立方程式が得られる.これは前に説明した通り逆行列で解くことができる.
 
\section{$\chi^2$の極小値から(2変数の例)}
\begin{MapleInput}
> restart; X:=[1,2,3,4]: Y:=[0,5,15,24]: f1:=x->a0+a1*x:
  S:=0: 
  for i from 1 to 4 do 
    S:=S+(f1(X[i])-Y[i])^2; 
  end do:
> fS:=unapply(S,(a0,a1));
\end{MapleInput}
\begin{MapleOutput}
{\it fS}\, := \,( {{\it a0},{\it a1}} )\mapsto  \left( {\it a0}+{\it a1} \right) ^{2}+ \left( {\it a0}+2\,{\it a1}-5 \right) ^{2}+ \left( {\it a0}+3\,{\it a1}-15 \right) ^{2}+ \left( {\it a0}+4\,{\it a1}-24 \right) ^{2}
\end{MapleOutput}
\begin{MapleInput}
> expand(fS(a0,a1));
\end{MapleInput}
\begin{MapleOutput}
4\,{{\it a0}}^{2}+20\,{\it a0}\,{\it a1}+30\,{{\it a1}}^{2}-88\,{\it a0}-302\,{\it a1}+826
\end{MapleOutput}

\begin{MapleInput}
> plot3d(fS(a0,a1),a0=-20..20,a1=0..20);
\end{MapleInput}
\MaplePlot{50mm}{./figures/C8_LeastSquareFitplot3d3.eps}


\begin{MapleInput}
> eqs:={diff(expand(S),a0)=0, diff(expand(S),a1)=0};
\end{MapleInput}
\begin{MapleOutput}
{\it eqs}\, := \, \left\{ 8\,{\it a0}+20\,{\it a1}-88=0,20\,{\it a0}+60\,{\it a1}-302=0 \right\}
\end{MapleOutput}

\begin{MapleInput}
> solve(eqs,{a0,a1});
\end{MapleInput}
\begin{MapleOutput}
\left\{ {\it a0}=-\frac{19}{2},{\it a1}={\frac {41}{5}} \right\}
\end{MapleOutput}
 
\section{正規方程式(Normal Equations)による解}
より一般的な場合の最小二乗法の解法を説明する.先程の例では1次の多項式を近似関数とした.これをより一般的な関数,例えば,$\sin, \cos, \tan, \exp, \sinh$などとする.これを線形につないだ関数を
\begin{equation*}
F \left(x \right)=a _{0}\sin \left(x \right)+a _{1}\cos \left(x \right)+a _{2}\exp \left(-x \right)+a _{3}\sinh \left(x \right)+\cdots ={\sum_{k=1}^{M}}a _{k }X _{k }\left(x \right)
\end{equation*}
ととる.実際には,$X_k(x)$はモデルや,多項式の高次項など論拠のある関数列をとる.これらを基底関数(base functions)と呼ぶ.ここで線形といっているのは,パラメータ$a_k$について線形という意味である.このような,より一般的な基底関数を使っても,$\chi^2$関数は
\begin{equation*}
{\chi}^{2}=\sum _{i=1}^{N} \left( F \left( x_{{i}} \right) -y_{{i}} \right) ^{2}
=\sum _{i=1}^{N} \left( \sum _{k=1}^{M}a_{{k}}X_{{k}} \left( x_{{i}} \right) -y_{{i}} \right) ^{2}
\end{equation*}
と求めることができる.この関数を,$a_k$を変数とする関数とみなす.この関数が最小値を取るのは,$\chi^2$を$M$個の$a_k$で偏微分した式がすべて0となる場合であ
る.これを実際に求めてみると,
\begin{equation*}
\sum _{i=1}^{N} \left( \sum _{j=1}^{M}a_{{j}}X_{{j}} \left( x_{{i}} \right) -y_{{i}} \right) X_{{k}} \left( x_{{i}} \right) =0
\end{equation*}
となる.ここで,$k = 1..M$の$M$個の連立方程式である.この連立方程式を最小二乗法の正規方程式(normal equations)と呼ぶ.

上記の記法のままでは,ややこしいので,行列形式で書き直す.$N \times M$で,各要素を
\begin{equation*}
A_{ij} = X_j(x_i)
\end{equation*}
とする行列$A$を導入する.この行列は,
\begin{equation*}
A=\left[
\begin{array}{cccc}
X_1(x_1) & X_2(x_1) & \cdots & X_M(x_1) \\
\vdots & \vdots & \cdots & \vdots \\
\vdots & \vdots & \cdots & \vdots \\
\vdots & \vdots & \cdots & \vdots \\
X_1(x_N) & X_2(x_N) & \cdots & X_M(x_N) 
\end{array}
\right]
\end{equation*}
となる.これをデザイン行列と呼ぶ.すると先程の正規方程式は,
\begin{equation*}
A^t . A . a = A^t . y
\end{equation*}
で与えられる.$A^t$は行列$A$の転置(transpose)
\begin{equation*}
A^t = A_{ij}^t = A_{ji}
\end{equation*}
を意味し,得られた行列は,$M \times N$である.$a, y$はそれぞれ,
\begin{equation*}
a=\left[
\begin{array}{c}
a_1\\a_2\\\vdots\\a_M
\end{array}
\right],\,
y=\left[
\begin{array}{c}
y_1\\y_2\\\vdots\\y_N
\end{array}
\right]
\end{equation*}
である.

$M = 3, N = 25$として行列の次元だけで表現すると,
\begin{equation*}
\left[
\begin{array}{ccccc}
 &  & \cdots & &\\
\cdots & \cdots &  \cdots & \cdots & \cdots \\
 &  & \cdots & &\\
\end{array}
\right]
\left[
\begin{array}{ccc}
& \vdots &\\
& \vdots &\\
\cdots & \cdots &  \cdots\\
& \vdots &\\
& \vdots &\\
\end{array}
\right]
\left[
\begin{array}{c}
\vdots\\
\vdots\\
\vdots
\end{array}
\right]
=
\left[
\begin{array}{ccccc}
 &  & \cdots & &\\
\cdots & \cdots &  \cdots & \cdots & \cdots \\
 &  & \cdots & &\\
\end{array}
\right]
\left[
\begin{array}{c}
\vdots\\
\vdots\\
\vdots\\
\vdots\\
\vdots
\end{array}
\right]
\end{equation*}
となる.これは少しの計算で$3 \times 3$の逆行列を解く問題に変形できる.

\subsection{Mapleによる具体例}
\begin{MapleInput}
> restart; X:=[1,2,3,4]: Y:=[0,5,15,24]: 
  f1:=x->a[1]+a[2]*x+a[3]*x^2:
  with(LinearAlgebra): Av:=Matrix(1..4,1..3):
  ff:=(x,i)->x^(i-1):
  for i from 1 to 3 do 
    for j from 1 to 4 do
      Av[j,i]:=ff(X[j],i); 
    end do; 
  end do;
  Av;
\end{MapleInput}
\begin{MapleOutput}
\left[ \begin{array}{ccc} 1&1&1\\1&2&4\\1&3&9\\1&4&16\end {array} \right]
\end{MapleOutput}
\begin{MapleInput}
> Ai:=MatrixInverse(Transpose(Av).Av);
\end{MapleInput}
\begin{MapleOutput}
{\it Ai}\, := \, \left[ \begin {array}{ccc} 
{\displaystyle \frac {31}{4}}&-{\displaystyle \frac {27}{4}}&\displaystyle \frac{5}{4}\\
-{\displaystyle \frac {27}{4}}&{\displaystyle \frac {129}{20}}&\displaystyle -\frac{5}{4}\\
\displaystyle \frac{5}{4}&\displaystyle -\frac{5}{4}&\displaystyle \frac{1}{4}
\end {array} \right]
\end{MapleOutput}
\begin{MapleInput}
> b:=Transpose(Av).Vector(Y);
\end{MapleInput}
\begin{MapleOutput}
b\, := \, \left[ \begin {array}{c} 44\\151\\539\end {array} \right]
\end{MapleOutput}

\begin{MapleInput}
> Ai.b;
\end{MapleInput}
\begin{MapleOutput}
\left[ \begin {array}{c}\displaystyle -\frac{9}{2}\\
\displaystyle {\frac {16}{5}}\\
1\end {array} \right]
\end{MapleOutput}
 
\section{特異値分解(Singular Value Decomposition)による解}
正規方程式を解くときには,少し注意が必要である.正規方程式での共分散行列,特異値分解の導出や標準偏差との関係はNumRecipeを参照せよ.
\begin{MapleInput}
> restart; X:=[1,2,3,4]: Y:=[0,5,15,24]: f1:=x->a[1]+a[2]*x+a[3]*x^2:
> with(LinearAlgebra): Av:=Matrix(1..4,1..3):
> ff:=(x,i)->x^(i-1): 
  for i from 1 to 3 do 
    for j from 1 to 4 do
      Av[j,i]:=ff(X[j],i); 
    end do; 
  end do; 
  Av;
\end{MapleInput}
\begin{MapleOutput}
\left[ \begin {array}{ccc} 1&1&1\\1&2&4\\1&3&9\\1&4&16\end {array} \right] 
\end{MapleOutput}

\begin{MapleInput}
> U,S,Vt:=evalf(SingularValues(Av,output=['U','S','Vt'])):
> DiagonalMatrix(S[1..3],4,3); U.DiagonalMatrix(S[1..3],4,3).Vt:
\end{MapleInput}
\begin{MapleOutput}
\left[ \begin {array}{ccc}  19.6213640200000015&0&0\\0& 1.71206987399999999&0\\0&0& 0.266252879300000022\\0&0&0\end {array} \right]
\end{MapleOutput}

\begin{MapleInput}
> iS:=Vector(3): 
  for i from 1 to 3 do 
    iS[i]:=1/S[i]; 
  end do:
> DiS:=DiagonalMatrix(iS[1..3],3,4);
\end{MapleInput}
\begin{MapleOutput}
{\it DiS}\, := \, \left[ \begin {array}{cccc}  0.05096485642&0&0&0\\0& 0.5840883104&0&0\\0&0& 3.755827928&0\end {array} \right]
\end{MapleOutput}

\begin{MapleInput}
> Transpose(Vt).DiS.(Transpose(U).Vector(Y));
\end{MapleInput}
\begin{MapleOutput}
\left[ \begin {array}{c} - 4.50000000198176498\\ 3.20000000035008324\\ 1.00000000040565196\end {array} \right]
\end{MapleOutput}

 
\section{2次元曲面へのフィット}
先程の一般化をより発展させると,3次元$(x_i, y_i, z_i)$で提供されるデータへの,2次元平面でのフィットも可能となる.2次元の単純な曲面は,方程式を使って,
\begin{equation*}
F(x, y) = a_1+a_2\,x+a_3\,y+a_4\,xy+a_5\,x^2+a_6\,y^2
\end{equation*}
となる.デザイン行列の$i$行目の要素は,
\begin{equation*}
[1, x_i, y_i, x_i\,y_i, x_i^2, y_i^2]
\end{equation*}
として,それぞれ求める.このデータの変換の様子をMapleスクリプトで詳しく示した.後は,通常の正規方程式を解くようにすれば,このデータを近似する曲面を定めるパラメータ$a_1, a_2, \cdots,a_6$が求まる.最小二乗法はパラメータ$a_k$について線形であればよい.

\subsection{Mapleによる具体例}
実際のデータ解析での例.データの座標をx,y,zで用意して,Mapleの埋め込み関数のleastsquareでfitしている.
\begin{MapleInput}
> with(plots):with(plottools): 
  z:=[0.000046079702088, 0.000029479057275,
  0.000025769637830, 0.000034951410953, 0.000057024385455, 0.000029485453808,
  0.000011519913869, 0.000006442404299, 0.000014252898382, 0.000034951410953,
  0.000025769637773, 0.000006442404242, 0.000000000000057, 0.000006442404242,
  0.000025769637773, 0.000034932221524, 0.000014246501905, 0.000006442404299,
  0.000011519913926, 0.000029479057332, 0.000056973214100, 0.000034932221467,
  0.000025769637773, 0.000029485453808, 0.000046079702031]:
>  x:=[]:
  y:=[]:
  p1:=2:
  for i from -p1 to p1 do
    for j from -p1 to p1 do
      x:=[op(x),i*0.0005];
      y:=[op(y),j*0.0005];
    end do;
  end do;
> with(LinearAlgebra): p2:=convert(Transpose(Matrix([x,y,z])),listlist):
  pp2:=pointplot3d(p2,symbol=circle,symbolsize=30,color=black):
  with(stats): data:=[x,y,z]: 
  fit1:=fit[leastsquare[[s,t,u], 
    u=a1+a2*t+a3*s+a4*t*s+a5*t^2+a6*s^2, 
    {a1,a2,a3,a4,a5,a6}]](data);
\end{MapleInput}
\begin{MapleOutputGather}
  {\it fit1}\, := \,u= 25.76962838\,{s}^{2}- 5.459553587\,st+ 25.76962835\,{t}^{2} \\ \notag
  - 0.000006396456800\,s+ 0.000006396438400\,t-{ 8.657142857\times 10^{-13}} \notag
\end{MapleOutputGather}
\begin{MapleInput}
> f1:=unapply(rhs(fit1),(s,t)):
> pf1:=plot3d(f1(ss,tt),ss=-0.001..0.001,tt=-0.001..0.001,color=gray):
> display(pf1,pp2,axes=boxed);
\end{MapleInput}
\MaplePlot{50mm}{./figures/C8_LeastSquareFitplot3d4.eps}

\subsection{正規方程式による解法}
デザイン行列へのデータ変換
\begin{MapleInput}
> bb:=Vector(25): A:=Matrix(25,6): 
  p1:=2: 
  for i from 1 to 25 do 
    A[i,1]:=1;
    A[i,2]:=x_i; 
    A[i,3]:=y_i; 
    A[i,4]:=x_i*y_i; 
    A[i,5]:=x_i^2; 
    A[i,6]:=y_i^2;
    bb_i:=z_i; 
  end do:
\end{MapleInput}
正規方程式の解
\begin{MapleInput}
> MatrixInverse(Transpose(A).A).(Transpose(A).bb);
\end{MapleInput}
\begin{MapleOutput}
\left[ \begin {array}{c} -{ 9.185257196\times 10^{-13}}\\   - 0.00000639644675999994798\\    0.00000639644220000032532\\   - 5.45955358336000173\\    25.7696284050857187\\    25.7696284050857543\end {array} \right]
\end{MapleOutput}
 
\section{課題}
\begin{enumerate}
\item 4x4の行列を適当に作り,Mapleで固有値を求めよ.求め方はマニュアルを参照せよ.
\item Jacobi法によって固有値を求めよ.
\item LAPACKに含まれているdsyev関数を用いて実対称行列の固有値を求めよ.(演習で詳しく取り上げている.研究や今後のために必要と思うときは,テキストを取りにおいで)
\end{enumerate} 


\chapter{非線形最小2乗法(NonLinearFit)}
\section{非線形最小2乗法の原理}
もっとも簡単な例で原理を解説する.近似関数として,
\begin{equation*}
F(x) = a_0+a_1\,x
\end{equation*}
という直線近似を考える.もっともらしい関数は$N$点の測定データとの差$d_i = F(x_i)-y_i$を最小にすればよさそうであるが,これはプラスマイナスですぐに消えて不定になる.そこで,
\begin{equation*}
\chi^{2}=\sum_i^N d_i^2=\sum_i^N\left(a_0+a_1\,x_i-y_i\right)^2
\end{equation*}
という関数を考える.この$\chi^2$(カイ二乗)関数が,$a_0, a_1$をパラメータとして変えた時に最小となる$a_0, a_1$を求める.これは,それらの微分がそれぞれ0となる場合である.これは$\chi^2$の和$\sum$(sum)の中身を展開し,
\ifHIKI %%%%
||$\chi^2=$||            ||

\else %%%%

\begin{table}[htbp]
\begin{center}
\begin{tabular}{cc}
$\chi^2=$& \\
&
\setlength{\unitlength}{1cm}
\begin{picture}(10,6.5)
\put(0,0){\framebox(10,6.5){}}
\end{picture}

\end{tabular}
\end{center}
\end{table}%
\fi %%%%

$a_0, a_1$でそれぞれ微分すれば
\ifHIKI %%%%
||$\displaystyle \frac{\partial}{\partial a_0} \chi^2 =$||            ||
||$\displaystyle \frac{\partial}{\partial a_1} \chi^2 =$||            ||
\else %%%%

\begin{table}[htbp]
\begin{center}
\begin{tabular}{cc}
$\displaystyle \frac{\partial}{\partial a_0} \chi^2 =$& 
\setlength{\unitlength}{1cm}
\begin{picture}(10,3.5)
\put(0,0){\framebox(10,3.5){}}
\end{picture}  \\
$\displaystyle \frac{\partial}{\partial a_1} \chi^2 =$&
\setlength{\unitlength}{1cm}
\begin{picture}(10,3.5)
\put(0,0){\framebox(10,3.5){}}
\end{picture}  \\
\end{tabular}
\end{center}
\end{table}%
\fi %%%%

という$a_0, a_1$を未知変数とする2元の連立方程式が得られる.これは前に説明した通り逆行列で解くことができる.
 
\section{具体的な手順}

パラメータの初期値を
\begin{equation*}
a_{{0}}+\Delta\,a,\,b_{{0}}+\Delta\,b,\,c_{{0}}+\Delta\,c,\,d_{{0}}+\Delta\,d
\end{equation*}
とする.このとき関数$f$を真値$a_0, b_0, c_0, d_0$のまわりでテイラー展開し,高次項を無視すると
\begin{equation*}
\Delta\,f=f \left( a_{{0}}+\Delta\,a_{{1}},b_{{0}}+\Delta\,b_{{1}},c_{{0}}+\Delta\,c_{{1}},d_{{0}}+\Delta\,d_{{1}} \right) -f \left( a_{{0}},b_{{0}},c_{{0}},d_{{0}} \right)
\end{equation*}

\begin{equation*}
=\left(\frac{\partial }{\partial a }f \right)_{0}\Delta a _{1}+\left(\frac{\partial }{\partial b }f \right)_{0}\Delta b _{1}+\left(\frac{\partial }{\partial c }f \right)_{0}\Delta c _{1}+\left(\frac{\partial }{\partial d }f \right)_{0}\Delta d _{1}
\end{equation*}
となる.

課題でつくったデータはt = 1からt = 256までの時刻に対応したデータ点$f_{1},\,f_{2},\,\cdots  f_{256}$とする.各測定値とモデル関数から予想される値との差$\Delta f_1,\Delta f_2,\cdots,\Delta f_{256}$は,
\begin{equation}
\left(\begin{array}{c}\Delta f _{1} \\\Delta f _{2} \\ \vdots \\\Delta f _{256} \\\end{array}\right)=J \left(\begin{array}{c}\Delta a _{1} \\\Delta b _{1} \\\Delta c _{1} \\\Delta d _{1} \\\end{array}\right)
\end{equation}
となる.ここで$J$はヤコビ行列と呼ばれる行列で,4列256行
\begin{equation}
\displaystyle J =\left(\begin{array}{cccc}\left(\frac{\partial }{\partial a }f \right)_{1} & \left(\frac{\partial }{\partial b }f \right)_{1} & \left(\frac{\partial }{\partial c }f \right)_{1} & \left(\frac{\partial }{\partial d }f \right)_{1} \\ \vdots & \vdots  &  \vdots & \vdots  \\\left(\frac{\partial }{\partial a }f \right)_{256} & \left(\frac{\partial }{\partial b }f \right)_{256} & \left(\frac{\partial }{\partial c }f \right)_{256} & \left(\frac{\partial }{\partial d }f \right)_{256} \\\end{array}\right)
\end{equation}
である.このような矩形行列の逆行列は転置行列$J^T$を用いて,`
\begin{equation}
J ^{-1}=\left(J ^{T }J \right)^{-1}J ^{T }
\end{equation}
と表わされる.したがって,真値からのずれは
\begin{equation}
\left(\begin{array}{c}\Delta a_2 \\\Delta b_2 \\\Delta c_2 \\\Delta d_2 \\\end{array}\right)
=\left(J ^{T }J \right)^{-1}J ^{T }
\left(\begin{array}{c}\Delta f _{1} \\\Delta f _{2} \\ \vdots \\\Delta f _{256} \\\end{array}\right)
\end{equation}
で求められる.理想的には$(\Delta a_2,\,\Delta b_2,\,\Delta c_2,\,\Delta d_2)$は$(\Delta a,\,\Delta b,\,\Delta c,\,\Delta d)$に一致するはずだが,測定誤差と高次項のために一致しない.初期値に比べ,より真値に近づくだけ.そこで,新たに得られたパラメータの組を新たな初期値に用いて,より良いパラメータに近付けていくという操作を繰り返す.新たに得られたパラメータと前のパラメータとの差がある誤差以下になったところで計算を打ち切り,フィッティングの終了となる.
 
\section{Mapleによる解法の指針}
線形代数計算のためにサブパッケージとしてLinearAlgebraを呼びだしておく.
\begin{MapleInput}
> restart; 
  with(plots): 
  with(LinearAlgebra):
\end{MapleInput}


データを読み込む.
\begin{MapleInput}
> ndata:=8: 
  f1:=t->subs({a1=1,a2=10,a3=1,a4=4},a1+a2/(a3+(t-a4)^2) );
\end{MapleInput}
\begin{MapleOutput}
{\it f1}\, := \,t\mapsto 1+10\, \left( 1+ \left( t-4 \right) ^{2} \right) ^{-1}
\end{MapleOutput}
データの表示
\begin{MapleInput}
> T:=[seq(f1(i),i=1..ndata)]:
  listplot(T); 
  l1:=listplot(T):
\end{MapleInput}
\MaplePlot{80mm}{./figures/C9_NonLinearFitplot2d4.eps}


ローレンツ型の関数を仮定し,関数として定義.
\begin{MapleInput}
> f:=t->a1+a2/(a3+(t-a4)^2); nparam:=4:
\end{MapleInput}
\begin{MapleOutput}
f\, := \,t\mapsto {\it a1}+{\frac {{\it a2}}{{\it a3}+ \left( t-{\it a4} \right) ^{2}}}\end{MapleOutput}
ヤコビアンの中の微分を新たな関数として定義.
\begin{MapleInput}
> for i from 1 to nparam do 
    dfda||i:=unapply(diff(f(x),a||i),x); 
  end do;
\end{MapleInput}
\begin{MapleOutputGather}
{\it dfda1}\, := \,x\mapsto 1 \notag \\
{\it dfda2}\, := \,x\mapsto  \left( {\it a3}+ \left( x-{\it a4} \right) ^{2} \right) ^{-1}
\notag \\
{\it dfda3}\, := \,x\mapsto -{\frac {{\it a2}}{ \left( {\it a3}+ \left( x-{\it a4} \right) ^{2} \right) ^{2}}} \notag \\
{\it dfda4}\, := \,x\mapsto -{\frac {{\it a2}\, \left( -2\,x+2\,{\it a4} \right) }{ \left( {\it a3}+ \left( x-{\it a4} \right) ^{2} \right) ^{2}}}  \notag
\end{MapleOutputGather}
ここで,"$||$"は連結作用素とよばれるMapleのコマンドで,$dfda||1 \mapsto dfda1$と連結する.
初期値を仮定して,データとともに関数を表示.
\begin{MapleInput}
> g1:=Vector([1,8,1,4.5]): 
  guess1:={}: 
  for i from 1 to nparam do
    guess1:={op(guess1),a||i=g1[i]}; 
  end do: 
  guess1;
\end{MapleInput}
\begin{MapleOutput}
\left\{ {\it a1}=1,{\it a2}=8,{\it a3}=1,{\it a4}= 4.5\right\}
\end{MapleOutput}

\begin{MapleInput}
> p1:=plot(subs(guess1,f(x)),x=1..ndata): 
  display(l1,p1);  
\end{MapleInput}
\MaplePlot{50mm}{./figures/C9_NonLinearFitplot2d5.eps}
見やすいように,小数点以下を3桁表示に制限する.
\begin{MapleInput}
> interface(displayprecision=3):
> df:=Vector([seq(subs(guess1,T[i]-f(i)),i=1..ndata)]);
\end{MapleInput}
\begin{MapleOutput}
{\it df}\, := \, \left[ \begin {array}{c}  0.396\\  0.897\\  2.538\\  3.600\\ - 1.400\\ - 0.462\\ - 0.103\\ - 0.016\end {array} \right]
\end{MapleOutput}

\begin{MapleInput}
> Jac:=Matrix(ndata,nparam): 
  for i from 1 to ndata do 
    for j from 1 to nparam do
      Jac[i,j]:=evalf(subs(guess1,dfda||j(i))); 
    end do: 
  end do:
  Jac;
\end{MapleInput}
\begin{MapleOutput}
\left[ \begin {array}{cccc}  1.0& 0.075&- 0.046&- 0.319\\  1.0& 0.138&- 0.152&- 0.761\\  1.0& 0.308&- 0.757&- 2.272\\  1.0& 0.800&- 5.120&- 5.120\\  1.0& 0.800&- 5.120& 5.120\\  1.0& 0.308&- 0.757& 2.272\\  1.0& 0.138&- 0.152& 0.761\\  1.0& 0.075&- 0.046& 0.319\end {array} \right]
\end{MapleOutput}

\begin{MapleInput}
> tJac:=(MatrixInverse(Transpose(Jac).Jac)).Transpose(Jac);
\end{MapleInput}
\begin{MapleOutput}
{\it tJac}\, := \, \left[ \begin {array}{cccccccc}  0.565& 0.249&- 0.354& 0.040& 0.040&- 0.354& 0.249& 0.565\\ - 2.954&- 0.506& 4.012&- 0.552&- 0.552& 4.012&- 0.506&- 2.954\\ - 0.352&- 0.029& 0.557&- 0.176&- 0.176& 0.557&- 0.029&- 0.352\\ - 0.005&- 0.012&- 0.035&- 0.080& 0.080& 0.035& 0.012& 0.005\end {array} \right] 
\end{MapleOutput}

\begin{MapleInput}
> g2:=tJac.df; 
  g1:=g1+g2;
\end{MapleInput}
\begin{MapleOutputGather}
{\it g2}\, := \, \left[ \begin {array}{c} - 0.235\\  5.592\\  0.613\\ - 0.520\end {array} \right]  \notag \\
{\it g1}\, := \, \left[ \begin {array}{c}  0.765\\  13.592\\  1.613\\  3.980\end {array} \right] \notag
\end{MapleOutputGather}
これをまたもとの近似値(guess)に入れ直して表示させると以下のようになる.カーブがデータに近づいているのが確認できよう.この操作をずれが十分小さくなるまで繰
り返す.
\begin{MapleInput}
> guess1:={seq(a||i=g1[i],i=1..nparam)};
  p1:=plot(subs(guess1,f(x)),x=1..ndata):
  display(l1,p1);
\end{MapleInput}
\begin{MapleOutput}
guess1:=\{a1=0.765, a2=13.592, a3=1.613, a4=3.980\}
\end{MapleOutput}
\MaplePlot{50mm}{./figures/C9_NonLinearFitplot2d6.eps}

4回ほど繰り返すと以下の通り,いい値に収束している.
\begin{MapleOutput}
guess1:=\{a1 = 1.006, a2 = 9.926, a3 = .989, a4 = 4.000\}
\end{MapleOutput}

 
\section{Gauss-Newton法に関するメモ}
このGauss-Newton法と呼ばれる非線形最小二乗法は線形問題から拡張した方法として論理的に簡明であり,広く使われている.しかし,収束性は高くなく,むしろ発散しやすいので注意が必要.2次の項を無視するのでなく,うまく見積もる方法を用いたのがLevenberg-Marquardt法である.明快な解説がNumerical Recipes in C(C 言語による数値計算のレシピ)WilliamH.Press 他著,技術評論社1993にある.
 
\section{課題}
\begin{enumerate}
\item 4x4の行列を適当に作り,Mapleで固有値を求めよ.求め方はマニュアルを参照せよ.
\item Jacobi法によって固有値を求めよ.
\item LAPACKに含まれているdsyev関数を用いて実対称行列の固有値を求めよ.(演習で詳しく取り上げている.研究や今後のために必要と思うときは,テキストを取りにおいで)
\end{enumerate} 
\section{解答例}
2. ふた山ピークへのフィット.
\begin{MapleInput}
> restart; with(plots): with(LinearAlgebra):
> f1:=t->subs({a=10,b=40000,c=380,d=128},a+b/(c+(t-d)^2) );
> f2:=t->subs({a=10,b=40000,c=380,e=90},a+b/(c+(t-e)^2) );
> T:=[seq((f1(i)+f2(i))*(0.6+0.2*evalf(rand()/10^12)),i=1..256)]:
\end{MapleInput}
\begin{MapleOutputGather}
{\it f1}\, := \,t\mapsto 10+40000\, \left( 380+ \left( t-128 \right) ^{2} \right) ^{-1} \notag \\
{\it f2}\, := \,t\mapsto 10+40000\, \left( 380+ \left( t-90 \right) ^{2} \right) ^{-1} \notag
\end{MapleOutputGather}
\begin{MapleInput}
> l1:=listplot(T):
> f:=t->a1+a2/(a3+(t-a4)^2)+a2/(a3+(t-a5)^2); 
  nparam:=5:
\end{MapleInput}
\begin{MapleOutput}
f\, := \,t\mapsto {\it a1}+{\frac {{\it a2}}{{\it a3}+ \left( t-{\it a4} \right) ^{2}}}+{\frac {{\it a2}}{{\it a3}+ \left( t-{\it a5} \right) ^{2}}}
\end{MapleOutput}
\begin{MapleInput}
> for i from 1 to nparam do 
    dfda||i:=unapply(diff(f(x),a||i),x); 
  end do;
\end{MapleInput}
\begin{MapleOutputGather}
{\it dfda1}\, := \,x\mapsto 1 \notag \\
{\it dfda2}\, := \,x\mapsto  \left( {\it a3}+ \left( x-{\it a4} \right) ^{2} \right) ^{-1}+ \left( {\it a3}+ \left( x-{\it a5} \right) ^{2} \right) ^{-1} \notag \\
{\it dfda3}\, := \,x\mapsto -{\frac {{\it a2}}{ \left( {\it a3}+ \left( x-{\it a4} \right) ^{2} \right) ^{2}}}-{\frac {{\it a2}}{ \left( {\it a3}+ \left( x-{\it a5} \right) ^{2} \right) ^{2}}} \notag \\
{\it dfda4}\, := \,x\mapsto -{\frac {{\it a2}\, \left( -2\,x+2\,{\it a4} \right) }{ \left( {\it a3}+ \left( x-{\it a4} \right) ^{2} \right) ^{2}}} \notag \\
{\it dfda5}\, := \,x\mapsto -{\frac {{\it a2}\, \left( -2\,x+2\,{\it a5} \right) }{ \left( {\it a3}+ \left( x-{\it a5} \right) ^{2} \right) ^{2}}} \notag
\end{MapleOutputGather}

\begin{MapleInput}
> g1:=Vector([10,1200,10,125,90]);
\end{MapleInput}
\begin{MapleOutput}
{\it g1}\, := \, \left[ \begin {array}{c} 10\\ 1200\\ 10\\ 125\\ 90\end {array} \right] 
\end{MapleOutput}
\begin{MapleInput}
> guess1:={seq(a||i=g1[i],i=1..nparam)};
\end{MapleInput}
\begin{MapleOutput}
guess1 := \{a1 = 10, a2 = 1200, a3 = 10, a4 = 125, a5 = 90\}
\end{MapleOutput}

\begin{MapleInput}
> p1:=plot(subs(guess1,f(x)),x=1..256): 
  display(l1);
\end{MapleInput}
\MaplePlot{50mm}{./figures/C9_NonLinearFitplot2d9.eps}

\begin{MapleInput}
> df:=Vector([seq(subs(guess1,T[i]-f(i)),i=1..256)]):
  Jac:=Matrix(1..256,1..nparam,sparse):
  for i from 1 to 256 do 
    for j from 1 to nparam do
      Jac[i,j]:=evalf(subs(guess1,dfda||j(i))); 
    end do:
  end do:
  tJac:=(MatrixInverse(Transpose(Jac).Jac)).Transpose(Jac):
  g2:=tJac.df; g1:=g1+g2;
\end{MapleInput}
\begin{MapleOutputGather}
{\it g2}\, := \, \left[ \begin {array}{c} - 0.390553882992161205\\  1584.55290636967129\\  24.9577909601538366\\ - 0.0472041829705451138\\ - 0.00719532042503852940\end {array} \right]  \notag \\
{\it g1}\, := \, \left[ \begin {array}{c}  13.6348019182603064\\  29567.3667677707381\\  410.545681677467769\\  128.512734548828887\\  90.9223109918718678\end {array} \right] \notag
\end{MapleOutputGather}

\begin{MapleInput}
> guess1:={seq(a||i=g1[i],i=1..nparam)};
  p1:=plot(subs(guess1,f(x)),x=1..256):
  display(l1,p1);
\end{MapleInput}
\begin{MapleOutput}
guess1 := \{a1 = 30.251, a2 = 3854.136, a3 = 39.571, a4 = 124.800, a5 = 89.960\}
\end{MapleOutput}
\MaplePlot{50mm}{./figures/C9_NonLinearFitplot2d10.eps}
何回か繰り返せば,データ点に近づいてくるはず.
 

\chapter{FFT(Fast Fourier Transformation)}
\section{FFTの応用}
Fast Fourier Transformation(FFT)高速フーリエ変換(あるいはデジタル(離散)フーリエ変換(DFT))は,周波数分解やフィルターを初め,画像処理などの多くの分野で使われている.基本となる考え方は,直交基底による関数の内挿法である.最初にその応用例を見た後,どのような理屈でFFTが動いているかを解説する.

\subsection{周波数分解}
はじめの例は,周波数分解.先ずは,非整合な波を二つ用意しておく.
\begin{MapleInput}
> restart:
  funcs:=[sin(i/13),sin(i/2)];
  #funcs:=[sin(i*2),2*sin(i/2)];
  plot(funcs,i=0..300);
\end{MapleInput}
\MaplePlot{50mm}{./figures/C10_FFTplot2d1.eps}
これを重ねあわせた波を作る.
\begin{MapleInput}
> data1:=[]:
  for i from 1 to 256 do
    data1:=[op(data1),evalf(funcs[1]+funcs[2]))]; 
  end do:
  with(plots): 
  listplot(data1);
\end{MapleInput}
\MaplePlot{50mm}{./figures/C10_FFTplot2d2.eps}
ゆっくり変化する波に,激しく変化する波が重なっていることが読み取れる.これにFFTを掛ける
\begin{MapleInput}
> X:=array(data1): 
  Y:=array(1..256,sparse):
  FFT(8,X,Y);
\end{MapleInput}
\begin{MapleOutput}
256
\end{MapleOutput}
その強さを求めて,周波数で表示すると,
\begin{MapleInput}
> Data2:=[seq([i,sqrt(X[i]^2+Y[i]^2)],i=1..128)]:
  plot(Data2);
\end{MapleInput}
\MaplePlot{50mm}{./figures/C10_FFTplot2d3.eps}
もとの2つの周波数に対応するところにピークができているのが確認できる.広がりは,誤差のせい.
logplotでも良い.

\subsection{高周波フィルター}
次の例は,高周波フィルター.たとえば次のようなローレンツ関数を考える.
\begin{MapleInput}
> restart;
  f1:=t->subs(a=10,b=40000,c=380,d=128,a+b/(c+(t-d)^2));
\end{MapleInput}
\begin{MapleOutput}
{\it f1}\, := \,t\mapsto 10+\displaystyle \frac{40000}{380+ \left( t-128 \right) ^{2} }
\end{MapleOutput}

\begin{MapleInput}
> plot(f1(t),t=1..256);
\end{MapleInput}
\MaplePlot{50mm}{./figures/C10_FFTplot2d4.eps}

これにノイズがのると,次のようになる.
\begin{MapleInput}
> T:=[seq(f1(i)*(0.6+0.8*evalf(rand()/10^12)),i=1..256)]:
  #T:=[seq(evalf(rand()/10^12),i=1..256)]:  #これはホワイトノイズ
  #T:=[seq(f1(i),i=1..256)]: #これは元の関数そのまま
  with(plots):
  listplot(T);
\end{MapleInput}
\MaplePlot{50mm}{./figures/C10_FFTplot2d5.eps}
これに高周波フィルターを掛けるとノイズが消えるが,その様子を示そう.先ずは,FFTを掛ける.
\begin{MapleInput}
> Idata:=array([seq(0,i=1..256)]):
  Rdata:=convert(T,array):
  FFT(8,Rdata,Idata);
\end{MapleInput}
\begin{MapleOutput}
256
\end{MapleOutput}
これは次のような強度分布をもっている.
\begin{MapleInput}
> Adata:=[seq([i,sqrt(Idata[i]^2+Rdata[i]^2)],i=1..128)]:
> logplot(Adata);
\end{MapleInput}
\MaplePlot{70mm}{./figures/C10_FFTplot2d6.eps}
低周波の部分に,ゆっくりとした変化を表す成分が固まっている.次のような三角フィルターを用意する.これは,低周波ほど影響を大きくするフィルター.
\begin{MapleInput}
> filter:=x->piecewise(x>=0 and x<=20,(1-x/20)): #三角フィルター
  #filter:=x->piecewise(x>=0 and x<=20,1); #方形フィルター
  plot(filter(x),x=0..40);
\end{MapleInput}
\MaplePlot{30mm}{./figures/C10_FFTplot2d7.eps}
これとデータを各点で掛けあわせる事によって,フィルターを通したことになる.
\begin{MapleInput}
> FRdata:=array([seq(Rdata[i]*filter(i),i=1..256)]):
> FIdata:=array([seq(Idata[i]*filter(i),i=1..256)]):
\end{MapleInput}
先ほどと同様に表示すると
\begin{MapleInput}
> Bdata:=[seq([i,sqrt(FIdata[i]^2+FRdata[i]^2)],i=1..128)]:
> logplot(Adata);
\end{MapleInput}
\MaplePlot{70mm}{./figures/C10_FFTplot3d6.eps}
$i=20$以上の領域がフィルターによってちょん切られていることが確認できる.これを逆フーリエ変換する.
\begin{MapleInput}
> iFFT(8,FRdata,FIdata);
\end{MapleInput}
\begin{MapleOutput}
256
\end{MapleOutput}
これを表示すると,
\begin{MapleInput}
> listplot(FRdata);
\end{MapleInput}
\MaplePlot{50mm}{./figures/C10_FFTplot2d8.eps}
となる.ノイズが取り除かれているのが確認できる.元の関数に加えたホワイトノイズにFFTを掛ければ分かるが,全周波数域にわたって均質に広がった関数となる.これを三角フィルターなどで高周波成分をカットすることで,ノイズが取り除かれていくのが理解されよう.
 
\section{FFTの動作原理}
このように便利なFFTであるが,どのような理屈で導かれるのか? Fourier変換法は,この課題だけでも何回ものコマ数が必要なほどの内容を含んでいる.ここでは,
その基本となる考え方(のひとつ)だけを提示する.
\begin{enumerate}
\item 関数の内挿で導入した基底関数を直交関数系でとる.ところが,展開係数を逆行列で求める手法では計算が破綻.
\item 直交関係からの積分による係数決定.
\item 選点直交性による計算の簡素化.
\item 高速フーリエ変換アルゴリズムによる高速化.
\end{enumerate}
 
\section{関数内挿としてのFourier関数系}
一連の関数系による関数の内挿は,基底関数を$\varphi_n(x)$として
\begin{equation*}
F(x) = \sum^N_{n=1}a_n \varphi_n(x)
\end{equation*}
で得られることを見た.
Fourier変換では基底関数として$\varphi _{n} \left( x \right) =\sin \left( 2\,\pi {\it nx} \right) ,\,\cos \left( 2\,\pi {\it nx} \right)$をとる.関数の内挿法で示したように,この$x_i$での値$f_i, i=1 \cdots M$と,近似の次数($N$)とでつくる係数行列,
\begin{equation*}
A=\left[ \begin{array}{cccc}
\varphi_0(x_0)&\varphi_1(x_0)& \cdots &\varphi_N(x_0) \\
\vdots & \vdots & \vdots & \vdots \\
\varphi_0(x_M)&\varphi_1(x_M)& \cdots &\varphi_N(x_M) 
\end{array}\right]
\end{equation*}
を求めて,係数$a_i$とデータ点$f_i$をそれぞれベクトルと考えると,
\begin{equation*}
\boldsymbol{A}.\boldsymbol{a} = \boldsymbol{f}
\end{equation*}
から,通常の逆行列を求める手法で係数を決定することもできる.しかし,この強引な方法はデータ数,関数の次数が多い,フーリエ変換が対象としようとする問題では破綻する.もっといい方法が必要で,それが直交関数系では存在する.
 
\section{直交関係からの積分による係数決定}
関数の直交関係は,
\begin{equation}
\int _{a}^{b}\varphi _{n} \left( x \right) \varphi _{m} \left( x \right) {dx}=
\delta_{\it mn}C_{n}= \left\{\begin{array}{lr}
C_m & at\, n=m \\
0 & at\, n\neq m
\end{array}\right.
\label{Eq:Orthogonal}
\end{equation}
である.定数$C_m$は,$\sin,\cos$の三角関数系では次の通り.
\begin{MapleInput}
> plot([sin(x),sin(3*x)],x=0..2*Pi);
\end{MapleInput}
\MaplePlot{50mm}{./figures/C10_FFTplot2d9.eps}

\begin{MapleInput}
> plot([sin(x)*sin(3*x)],x=0..2*Pi, color=black);
\end{MapleInput}
\MaplePlot{50mm}{./figures/C10_FFTplot2d10.eps}

\begin{MapleInput}
> int(sin(x)*sin(3*x),x=0..2*Pi);
\end{MapleInput}
\begin{MapleOutput}
0
\end{MapleOutput}
\begin{MapleInput}
> for i from 1 to 3 do for j from 1 to 3 do S:=int(sin(i*x)*sin(j*x),x=0..2*Pi);
> print(i,j,S); end do; end do:
\end{MapleInput}
\begin{MapleOutputGather}
1, 1, \pi \notag \\
1, 2, 0 \notag \\
1, 3, 0 \notag \\
2, 1, 0 \notag \\
2, 2, \pi \notag \\
2, 3, 0 \notag \\
3, 1, 0 \notag \\
3, 2, 0 \notag \\
3, 3, \pi \notag
\end{MapleOutputGather}
\begin{equation*}
\int _{a}^{b}F \left( x \right) \varphi _{m} \left( x \right) {dx}
\end{equation*}
を考える.先程の\ref{Eq:Orthogonal}式をいれると
\begin{equation}
\int _{a }^{b }F \left(x \right)\varphi _{m }\left(x \right) dx =\int _{a }^{b }{\sum^N_{n=1} }a _{n }\varphi _{n }\left(x \right)\varphi _{m }\left(x \right)d x = 
\left\{\begin{array}{lr}
a_m C_m & at\, n=m \\
0 & at\, n\neq m
\end{array}\right.
\end{equation}
となる.こうして,係数$a_n$が
\begin{equation*}
a_{{n}}=\frac {1}{C_n}\int _{a}^{b} F \left( x \right) \varphi _{{n}} \left( x \right) {dx}
\end{equation*}
で決定できる.
 
\section{直接積分によるフーリエ係数}
対象とする関数をまず作る.
\begin{MapleInput}
> restart;
> #F:=x->piecewise(x=0,1/2,x>0,x); 
  #F:=x->piecewise(x<1/2,x,x>=1/2,1-x);
  #F:=x->piecewise(x<1/2,-1,x>1/2,1); 
  F:=x->piecewise(x<1/2,-1,x>=1/2,1);
  #F:=x->piecewise(x>0 and x<1/2,-1,x>1/2,1); 
  #F:=x->x-1/2; 
  plot(F(x),x=0..1);
\end{MapleInput}
\begin{MapleOutput}
F := x \mapsto piecewise(x < \frac{1}{2}, -1, \frac{1}{2} \leq x, 1) 
\end{MapleOutput}
\MaplePlot{50mm}{./figures/C10_FFTplot2d11.eps}
piecewise関数は階段関数で,振る舞いはコメント(\#)を適当に外して確認せよ.
初期設定.
\begin{MapleInput}
> KK:=3; N:=2^KK;L:=1-0;
> 2*Pi*1/L*x;
\end{MapleInput}
\begin{MapleOutput}
2 \pi x
\end{MapleOutput}
\begin{MapleInput}
> int(F(x)*cos(2*Pi*1/L*x),x=0..L);
\end{MapleInput}
\begin{MapleOutput}
0
\end{MapleOutput}

\begin{MapleInput}
> for n from 0 to N do
    a[n]:=2/L*int(F(x)*cos(2*Pi*n/L*x),x=0..L); 
  end do;
\end{MapleInput}
\begin{MapleOutputGather}
a_0:=0 \notag \\
a_1:=0 \notag \\
a_2:=0 \notag \\
a_3:=0 \notag \\
a_4:=0 \notag \\
a_5:=0 \notag \\
a_6:=0 \notag \\
a_7:=0 \notag \\
a_8:=0 \notag 
\end{MapleOutputGather}

\begin{MapleInput}
> for n from 0 to N do 
    b[n]:=2/L*int(F(x)*sin(2*Pi*n/L*x),x=0..L); 
  end do;
\end{MapleInput}
\begin{MapleOutputGather}
b_0:=0 \notag \\
b_1:=\frac{4}{\pi} \notag \\
b_2:=0 \notag \\
b_3:=\frac{4}{3\pi} \notag \\
b_4:=0 \notag \\
b_5:=\frac{4}{5\pi} \notag \\
b_6:=0 \notag \\
b_7:=\frac{4}{7\pi} \notag \\
b_8:=0 \notag 
\end{MapleOutputGather}
ここで,オイラーの関係
\begin{MapleOutputGather}
a[n]=c[n]+c[-n],\, b[n]=I (c[n]-c[-n]) \notag \\
c[-n]= \frac{1}{2} (a[n] + b[n]),\, c[n]=\frac{1}{2} (a[n] - I b[n])
\end{MapleOutputGather}
を使って,三角関数系から$\exp$へ変換する.

\begin{MapleInput}
> for n from 0 to N do c[n]:=1/L*int(F(x)*exp(-I*2*Pi*n/L*x),x=0..L); end do;
> for n from 1 to N do c[-n]:=1/L*int(F(x)*exp(I*2*Pi*n/L*x),x=0..L); end do;
\end{MapleInput}
\begin{MapleOutputGather}
c_0:=0 \notag \\
c_1:=\frac{2I}{\pi} \notag \\
c_2:=0 \notag \\
c_3:=\frac{2I}{3\pi} \notag \\
c_4:=0 \notag \\
c_5:=\frac{2I}{5\pi} \notag \\
c_6:=0 \notag \\
c_7:=\frac{2I}{7\pi} \notag \\
c_8:=0 \notag \\
c_{-1}:=-\frac{2I}{\pi} \notag \\
c_{-2}:=0 \notag \\
c_{-3}:=-\frac{2I}{3\pi} \notag \\
c_{-4}:=0 \notag \\
c_{-5}:=-\frac{2I}{5\pi} \notag \\
c_{-6}:=0 \notag \\
c_{-7}:=-\frac{2I}{7\pi} \notag \\
c_{-8}:=0 \notag 
\end{MapleOutputGather}

\begin{MapleInput}
> F1:=unapply(sum(evalf(c[i]*exp(I*2*Pi*i/L*x)),i=-(N-1)..(N-1)),x):
> plot({Re(F1(x)),F(x)},x=0..1);
\end{MapleInput}
\MaplePlot{50mm}{./figures/C10_FFTplot2d12.eps}

\begin{MapleInput}
> evalf(2/3/Pi);
\end{MapleInput}
\begin{MapleOutput}
0.2122065907
\end{MapleOutput}
                                
 
\section{選点直交性による計算の簡素化}
ところが,実際に積分していては,時間がかかりすぎる.直交関数系の選点直交性を使うとより簡単になる.

\paragraph{直交関数系の選点直交性}
\begin{quotation}
直交多項式は,
\begin{equation*}
\varphi _{n} \left( x \right) =0 \,at\,x_1,\,x_2,\,\cdots x_{n}
\end{equation*}
である.$n-1$以下の次数$m,\,l$では,
\begin{equation*}
\sum _{i=1}^{n}\phi_{l} \left( x_{i} \right) \varphi _{m} \left( x_{i} \right) =\delta_{\it ml}C_{l}
\end{equation*}
が成り立つ.これは,直交関係と違い積分でないことに注意.証明は略.
\end{quotation}
これを使えば,この先程の直交関数展開
\begin{equation*}
F \left( x \right) =\sum _{l=1}^{N}a_{l}\varphi _{l} \left( x \right)
\end{equation*}
の両辺に$\varphi _{m} \left( x_{i} \right)$を掛けて$i$について和をとれば,
\begin{gather}
\sum _{i=1}^{n} F \left(x _{i }\right)\phi _{m }\left(x _{i }\right) = \\
\sum _{i=1}^{n}  \sum _{l=1}^{N}a_{l}\varphi _{l} \left( x_{i} \right) \varphi _{m} \left( x_{i} \right)  \notag \\
=\sum _{l=1}^{N}  a_{l}\sum _{i=1}^{n}\varphi _{l} \left( x_{i} \right) \varphi _{m} \left( x_{i} \right) \notag \\
=\sum _{l=1}^{N}a_{l}\delta_{\it ml}C_{m}=a_{m}C_{m} \notag
\end{gather}
となる.つまり,
\begin{equation*}
a_{m}=\frac{1}{C_{m}} {\sum _{i=1}^{n}F \left( x_{i} \right) \varphi _{m} \left( x_{i} \right) }
\end{equation*}
となり,単純な関数の代入とかけ算で係数が決定される.

\subsection{選点直交性を用いた結果}
\begin{MapleInput}
> KK:=4; N:=2^KK;L:=1-0;
> for k from 0 to N-1 do 
    c[k]:=evalf(sum(F(i*L/N)*exp(-I*2*Pi*k*i/N),i=0..N-1));
  end do;
\end{MapleInput}
\begin{MapleError}
c_0:=0. 
c_1:=-2.000000000 + 10.05467898 I 
c_2:=0. 
c_3:=-2.000000000 + 2.993211524 I 
c_4:=0. 
c_5:=-2.000000001 + 1.336357276 I 
c_6:=0. 
c_7:=-2.000000001 + 0.3978247331 I 
c_8:=0. 
c_9:=-2.000000001 - 0.3978247331 I 
c_10:=0. 
c_11:=-2.000000001 - 1.336357276 I 
c_12:=0. 
c_13:=-2.000000000 - 2.993211524 I 
c_14:=0. 
c_15:=-2.000000000 - 10.05467898 I
\end{MapleError}

\begin{MapleInput}
> F1:=unapply(sum(evalf(c[i]*exp(I*2*Pi*i/L*x)/N),i=0..(N/2-1))+
> sum(evalf(c[N-i]*exp(-I*2*Pi*i/L*x)/N),i=1..(N/2-1)),x):
> plot({Re(F1(x)),F(x)},x=0..1);
\end{MapleInput}
\MaplePlot{50mm}{./figures/C10_FFTplot2d13.eps}

 
\section{高速フーリエ変換アルゴリズムによる高速化}
$\sin, \cos$と$\exp$関数を結びつけるオイラーの関係を使うと,
\begin{equation*}
{\exp\left({\frac {2\pi}{N}}I\right)}=\cos \left({\frac { 2\,\pi }{N}} \right) +I\,\sin \left( {\frac {2\,\pi }{N}} \right) \end{equation*}
と変換できる.これを使うと,
\begin{equation*}
c_{{k}}=\frac{1}{C_{{m}}}\sum _{i=0}^{N-1}F \left( x_{{i}} \right) {{\exp}\left({{\frac {-2\,\pi I}{N}}}\right)}
\end{equation*}
となる.$N = 8$の場合を実際に計算すると,$\displaystyle z = \exp(-\frac{2\pi}{8}I)$として,$z^8=1,z^9=z, \cdots$を使うと,
\begin{equation*}
\left[\begin{array}{c}
c_0\\c_1\\c_2\\c_3\\c_4\\c_5\\c_6\\c_7
\end{array}\right] =
\left[\begin{array}{cccccccc}
1&1&1&1&1&1&1&1\\
1&z&z^2&z^3&z^4&z^5&z^6&z^7\\
1&z^2&z^4&z^6&1&z^2&z^4&z^6\\
1&z^3&z^6&z&z^4&z^7&z^2&z^5\\
1&z^4&1&z^4&1&z^4&1&z^4\\
1&z^5&z^2&z^7&z^4&z^1&z^6&z^3\\
\vdots & & & & & & &  \\
\vdots & & & & & & & 
\end{array}\right]
\left[\begin{array}{c}
F_0\\F_1\\F_2\\F_3\\F_4\\F_5\\F_6\\F_7
\end{array}\right]
\end{equation*}
となる.この行列計算を素直に実行すると,8×8=64回の演算が必要となる.これを減らせないかと考えたのが,高速フーリエ変換の始まりである.
この行列をよく見ると同じ計算を重複しておこなっていることが分かる.そこで,行列の左側と右側で同じ計算をしている部分をまとめると,
\begin{equation*}
\left[\begin{array}{c}
c_0\\c_1\\c_2\\c_3\\c_4\\c_5\\c_6\\c_7
\end{array}\right] =
\left[\begin{array}{cccccccc}
1&1&1&1&0&0&0&0\\
0&0&0&0&1&z&z^2&z^3\\
1&z^2&z^4&z^6&0&0&0&0\\
0&0&0&0&1&z^3&z^6&z\\
1&z^4&1&z^4&0&0&0&0\\
0&0&0&0&1&z^5&z^2&z^7\\
1&z^6&z^4&z^2&0&0&0&0\\
0&0&0&0&1&z^7&z^6&z^5\\
\end{array}\right]
\left[\begin{array}{c}
F_0+F_4\\F_1+F_5\\F_2+F_6\\F_3+F_7\\F_0-F_4\\F_1-F_5\\F_{[ \,\,]}-F_6\\F_{[\,\,]}-F_7
\end{array}\right]
\end{equation*}
とすることができる.ここで,$z^4 =-1$などを使っている.右側のベクトルの計算でロスするが,行列の中の計算の回数を半分に減らすことができる.再度できあがった行列を見れば,同じ計算をさらにまとめるこ
とができそうである.こうして,次々と計算回数を減らしていくことが可能で,最終的に行列部分の計算が一切なくなる.残るのは,右側のベクトルの足し算引き算だけになる.

このベクトルの組み合わせは,一見相当複雑そうで,その条件分岐で時間がかかりそうに思われる.しかし,よく調べてみれば,単純なビット演算で処理することが可能であるこ
とが判明した.こうして,2の整数乗のデータの組に対しては,極めて高速にフーリエ変換を実行することが可能となった.
FFTでの演算回数は,データ数をNとすると
\begin{equation*}
N\log_2 N
\end{equation*}
となる.単純な場合の$N^2$と比較すると,以下のようになり,どれだけ高速化されているかが理解されよう.
\begin{MapleInput}
> dN2:=[]; dFft:=[]; for i from 2 to 16 do N:=2^i; n2:=N*N; Fft:=N/2*log[2](N);
> Fft/n2; printf("%10d %12d %12d %10.5f\n",N,n2,Fft,evalf(Fft/n2));
> dN2:=[op(dN2),[N,n2]]; dFft:=[op(dFft),[N,Fft]]; end do:
\end{MapleInput}
\begin{MapleError}
         4           16            4    0.25000
         8           64           12    0.18750
        16          256           32    0.12500
        32         1024           80    0.07812
        64         4096          192    0.04688
       128        16384          448    0.02734
       256        65536         1024    0.01562
       512       262144         2304    0.00879
      1024      1048576         5120    0.00488
      2048      4194304        11264    0.00269
      4096     16777216        24576    0.00146
      8192     67108864        53248    0.00079
     16384    268435456       114688    0.00043
     32768   1073741824       245760    0.00023
     65536   4294967296       524288    0.00012
\end{MapleError}
\begin{MapleInput}
> with(plots):
> l1:=plot(dN2): l2:=plot(dFft):
> display(l1,l2);
\end{MapleInput}
\MaplePlot{50mm}{./figures/C10_FFTplot2d14.eps}

\begin{MapleInput}
> l1:=logplot(dN2): l2:=logplot(dFft):
> display(l1,l2);
\end{MapleInput}
\MaplePlot{50mm}{./figures/C10_FFTplot2d15.eps}

\begin{MapleInput}
> l1:=loglogplot(dN2): l2:=loglogplot(dFft):
> display(l1,l2);
\end{MapleInput}
\MaplePlot{50mm}{./figures/C10_FFTplot2d16.eps}
 
\section{FFT関数を用いた結果}
\begin{MapleInput}
> KK:=4;
  N:=2^KK;i:='i';L:=1-0;
  x1:=array([evalf(seq(F(i/N),i=0..N-1))]);
  y1:=array([evalf(seq(0,i=0..N-1))]);
\end{MapleInput}
\begin{MapleOutputGather}
{\it x1}\, := \, \left[ \begin {array}{cccccccccccccccc} - 1.0&- 1.0&- 1.0&- 1.0&- 1.0&- 1.0&- 1.0&- 1.0& 1.0& 1.0& 1.0& 1.0& 1.0& 1.0& 1.0& 1.0\end {array} \right] \notag \\
 {\it y1}\, := \, \left[ \begin {array}{cccccccccccccccc}  0.0& 0.0& 0.0& 0.0& 0.0& 0.0& 0.0& 0.0& 0.0& 0.0& 0.0& 0.0& 0.0& 0.0& 0.0& 0.0\end {array} \right] \notag
\end{MapleOutputGather}
\begin{MapleInput}
> FFT(KK,x1,y1);
\end{MapleInput}
\begin{MapleOutput}
16
\end{MapleOutput}

\begin{MapleInput}                                      
> interface(displayprecision=2):
  print(x1);print(y1);
\end{MapleInput}
\begin{MapleOutputGather}
\left[ \begin {array}{cccccccccccccccc}  0.0&- 2.0& 0.0&- 2.0& 0.0&- 2.0& 0.0&- 2.0& 0.0&- 2.0& 0.0&- 2.0& 0.0&- 2.0& 0.0&- 2.0\end {array} \right] \notag \\
\left[ \begin {array}{cccccccccccccccc}  0.0& 10.05& 0.0& 2.99& 0.0& 1.34& 0.0& 0.40& 0.0&- 0.40& 0.0&- 1.34& 0.0&- 2.99& 0.0&- 10.05\end {array} \right] \notag
\end{MapleOutputGather}

\begin{MapleInput}
> F2:=unapply(sum(evalf((x1[i]+I*y1[i])*exp(I*2*Pi*(i-1)/L*x)/N),i=1..N/2)+
         sum(evalf((x1[N-i+2]+I*y1[N-i+2])*exp(-I*2*Pi*(i-1)/L*x)/N),i=2..N/2),x):
> plot({Re(F2(x)),F(x)},x=0..1);
\end{MapleInput}
\MaplePlot{50mm}{./figures/C10_FFTplot2d17.eps}

 








