行列Aの行列式($\left|A\right|$あるいはdet$A$と表記)はDeterminantで求まります.
\begin{MapleInput}
> Determinant(A);
\end{MapleInput}
\begin{MapleOutput}
-18
\end{MapleOutput}
では次のクイズ.先ほど求めた,行列Aの行列式は,どこに対応するでしょう?
以下の(1,0),(0,1)の点を変換した点に原点からベクトルを結んでその意味を説明してください.さらに,そのマイナスの意味は?.
\begin{MapleInput}
> point3:=[]:line3:=[]: XX:=Matrix([[1,0],[0,1]]): 
  for i from 1 to 2 do
    x0:=Column(XX,i); x1:=A.x0; 
    p0:=convert(x0,list): 
    p1:=convert(x1,list):
    point3:=[op(point3),disk(p0,0.2,color=red),disk(p1,0.2,color=blue)]:
    line3:=[op(line3),arrow([0,0],p0,.05,.3,.1 ),arrow([0,0],p1,.05,.3,.1 )]:
  end do:
  display(point3,line3,view=[-1..8,-1..8],gridlines=true);
\end{MapleInput}
\MaplePlot{80mm}{./figures/LAFundamentalsplot2d4.eps}
