\begin{enumerate}
\item 西暦を代入したら,明治,大正,昭和,平成で答えてくれるプログラムを作成せよ.西暦1868, 1912,1926,1989年をそれぞれの元年とする.
\item 整数を代入したら,それ以下の素数をすべて表示するプログラムを作れ.素数かどうかの判定はMapleコマンドのisprimeを用いよ.
\item pが素数でp+2も素数のとき,これらは双子の素数と呼ばれる.10以上,100以下の双子の素数を全部見つけて出力せよ.
\item 素数判定を原理から実現せよ.
ある数nが素数かどうか(自分自身の数nと1以外の数で割りきれないかどうか)を判定せよ.割り算の余り(剰余)はiremで求まる.例えば
\begin{MapleInput}
> residue:=irem(9,2);
\end{MapleInput}
として変数residue(余りの英語)をprintfしてみよ.番兵を置いておいて,n-1から2までの数でnを次々と割っていき,一度でも割り切れれば番兵にマークをつける.ループが終わった後に番兵のマークを見て素数(prime number)かどうかを判定すればよい.
\item うるう年かどうかを表示するプログラムをかけ.\\
うるう年は4で割り切れる数の年.ただし,100で割り切れる年はうるう年でなく,400で割り切れる年はうるう年.
\item ゴールドバッハの予想 \\
「6以上の偶数は二つの素数の和として表わされる」という予想を100以下の偶数について検証せよ.あらかじめ100までの素数をリストアップしておいてそのなかから組み合わせを探すと便利.
\end{enumerate}